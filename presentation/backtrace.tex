\subsection{Backtrace}

%\begin{frame}[fragile]
%\frametitle{-g flag} % FIXME
%\begin{verbatim}
%-g Record debugging information for exception backtrace
%\end{verbatim}
%\end{frame}

\begin{frame}{Source code}
  \begin{columns}[c]
    \begin{column}{0.5\textwidth}
      \begin{itemize}
        \item Same code but this time we compile using debugging information (-g)
      \end{itemize}
    \end{column}
    \begin{column}{0.5\textwidth}
      \listocaml[firstline=9,lastline=34]{../src/backtrace/backtrace.ml}{backtrace/backtrace.ml}
    \end{column}
  \end{columns}
\end{frame}

\begin{frame}{CMM and disassembly of \funcname{definitely\_raise}}
  \begin{columns}[c]
    \begin{column}{0.5\textwidth}
      \begin{itemize}
        \item<1-> Impact of compiling with debugging information (-g): the CMM no longer uses \funcname{raise\_notrace} to raise the exception but \funcname{raise}
        \item<2-> Disassembly shows \funcname{raise} is actually a call to \funcname{caml\_raise\_exn}, an assembly function provided by the runtime
      \end{itemize}
      \bigskip
      \mintocaml[firstline=9,lastline=13,highlightlines=12]{../src/backtrace/backtrace.ml}
    \end{column}
    \begin{column}{0.5\textwidth}
      \onslide*<1>{
        \mintcmm[highlightlines=5]{../src/backtrace/camlBacktrace\_\_definitely\_raise\_347.cmm}
      }
      \onslide*<2>{
        \mintobjdump[firstline=2,highlightlines=14]{../src/backtrace/camlBacktrace\_\_definitely\_raise\_347.objdump}
      }
    \end{column}
  \end{columns}
\end{frame}

\begin{frame}{Raise}
  \begin{columns}[c]
    \begin{column}{0.5\textwidth}
      \onslide*<1>{
        \begin{itemize}
          \item raising the exceptino is performed using \funcname{caml\_raise\_exn}
          \item \funcname{caml\_raise\_exn} is an assembly function provided by the runtime
          \item \funcname{caml\_raise\_exn} is architecture specific
        \end{itemize}
      }
      \onslide*<2>{
        \mintobjdump[firstline=2,highlightlines=14]{../src/backtrace/camlBacktrace\_\_definitely\_raise\_347.objdump}
      }
      \bigskip
      \mintocaml[firstline=9,lastline=13,highlightlines=12]{../src/backtrace/backtrace.ml}
    \end{column}
    \begin{column}{0.5\textwidth}
      \providecommand\step{1}\begin{tikzpicture}[font=\sffamily\tiny]
\input{../src/backtrace/gdb.out}

\newcommand\trapExnAEnd{\xinteval{\trapExnABegin - 2 * \wordSize}}
\newcommand\trapExnBEnd{\xinteval{\trapExnBBegin - 2 * \wordSize}}
\newcommand\trapExnCEnd{\xinteval{\trapExnCBegin - 2 * \wordSize}}

\begin{stackDiagram}[yFactor=0.35,showAddress=true]{\frameMainBegin}{\frameDefinitelyEnd}
\callStack

\begin{funFrame}[name=main,color=GreenYellow]{\frameMainBegin}{\frameMainEnd}
\funReturnAddr{\frameMainRetaddr}
\frameLocal{1}{\frameMainLocalOne}
\end{funFrame}

\ifthenelse{\step<13}{
\trapFrame[name=ExnC,handler=\trapExnCHandler,next=\trapExnCNext,color=WildStrawberry]{\trapExnCBegin}
\coordAtRight{exnCtrap}{\exnHandlerAfterExnCTrap}
}{}


\ifthenelse{\step<10}{
\trapFrame[name=ExnB,handler=\trapExnBHandler,next=\trapExnBNext,color=OrangeRed]{\trapExnBBegin}
\coordAtRight{exnBtrap}{\exnHandlerAfterExnBTrap}
\draw [->,>=latex] (exnBtrap.east) to [out=0,in=0] (exnCtrap.east);

\begin{funFrame}[name=maybe,color=YellowGreen]{\frameMaybeBegin}{\frameMaybeEnd}
\funReturnAddr{\frameMaybeRetaddr}
\frameLocal{1}{\frameMaybeLocalOne}
\end{funFrame}

\begin{funFrame}[name=probably,color=Green]{\frameProbablyBegin}{\frameProbablyEnd}
\funReturnAddr{\frameProbablyRetaddr}
\frameLocal{1}{\frameProbablyLocalOne}
\end{funFrame}
}{}


\ifthenelse{\step<5}{
\trapFrame[name=ExnA,handler=\trapExnAHandler,next=\trapExnANext,color=Red]{\trapExnABegin}
\coordAtRight{exnAtrap}{\exnHandlerAfterExnATrap}
\draw [->,>=latex] (exnAtrap.east) to [out=0,in=0] (exnBtrap.east);

\begin{funFrame}[name=definitely,color=JungleGreen]{\frameDefinitelyBegin}{\frameDefinitelyEnd}
\funReturnAddr{\frameDefinitelyRetaddr}
\frameLocal{1}{\frameDefinitelyLocalOne}
\end{funFrame}

\begin{funFrame}[color=SeaGreen]{\frameCamlRaiseExnBegin}{\frameCamlRaiseExnEnd}
\funReturnAddr{\frameCamlRaiseExnRetaddr}
\end{funFrame}
}{}


\ifthenelse{\step>5 \and \step<10}{
\begin{funFrame}[name=reraise,color=JungleGreen]{\frameCamlReraiseExnBeginvB}{\frameCamlReraiseExnEndvB}
\funReturnAddr{\frameCamlReraiseExnRetaddrvB}
\end{funFrame}
}{}

\ifthenelse{\step>10 \and \step<13}{
\begin{funFrame}[name=reraise,color=JungleGreen]{\frameCamlReraiseExnBeginvC}{\frameCamlReraiseExnEndvC}
\funReturnAddr{\frameCamlReraiseExnRetaddrvC}
\end{funFrame}
}{}


\coordinate (bottom) at ($(stackBL) + (0,-0.25)$);


\ifthenelse{\step=1 \or \step=2}{
\domainStateBacktrace[pos=(bottom),yFactor=0.03]{\exnHandlerAfterExnATrap}{}{0}
\draw [->,>=latex] (exnHandlerRight.east) to [out=0,in=0] (exnAtrap.east);
}{}

\ifthenelse{\step>4 \and \step<7}{
\domainStateBacktrace[pos=(bottom),yFactor=0.03]{\exnHandlerAfterExnPopTrapvA}{\nextFrameDescrPCvB}{\stashBacktraceBacktracePosvA}
}{}

\ifthenelse{\step>9 \and \step<12}{
\domainStateBacktrace[pos=(bottom),yFactor=0.03]{\exnHandlerAfterExnPopTrapvB}{\nextFrameDescrPCvE}{\stashBacktraceBacktracePosvB}
}{}

\ifthenelse{\step=5}{
\draw [->,>=latex] (exnHandlerRight.east) to [out=0,in=0] (exnBtrap.east);
}{}

\ifthenelse{\step=10}{
\draw [->,>=latex] (exnHandlerRight.east) to [out=0,in=0] (exnCtrap.east);
}{}

\ifthenelse{\step=2}{
\showRegister{sp}{\stashBacktraceArgSPvA}
}{}

\ifthenelse{\step=3}{
\domainStateBacktrace[pos=(bottom),yFactor=0.03]{\exnHandlerAfterExnATrap}{\nextFrameDescrPCvA}{1}
\showRegister{sp}{\nextFrameDescrSPvA}
}{}

\ifthenelse{\step=4}{
\domainStateBacktrace[pos=(bottom),yFactor=0.03]{\exnHandlerAfterExnATrap}{\nextFrameDescrPCvB}{2}
\showRegister[opacity=0.25]{sp}{\nextFrameDescrSPvA}
\showRegister[offset=-0.8]{sp}{\nextFrameDescrSPvB}
}{}

% Drawing order matter for z-index trapsp/sp
\ifthenelse{\step>1 \and \step<5}{
\showRegister{trapsp}{\stashBacktraceArgTrapSPvA}
\showRegister{pc}{\frameCamlRaiseExnEnd}
}{}

\ifthenelse{\step>5 \and \step<10}{
\draw [->,>=latex] (exnHandlerRight.east) to [out=0,in=0] (exnBtrap.east);
}{}

\ifthenelse{\step=7}{
\domainStateBacktrace[pos=(bottom),yFactor=0.03]{\exnHandlerAfterExnPopTrapvA}{\nextFrameDescrPCvC}{3}
\showRegister[opacity=0.25]{sp}{\stashBacktraceArgSPvB}
\showRegister{sp}{\nextFrameDescrSPvC}
}{}

\ifthenelse{\step=8}{
\domainStateBacktrace[pos=(bottom),yFactor=0.03]{\exnHandlerAfterExnPopTrapvA}{\nextFrameDescrPCvD}{4}
\showRegister[opacity=0.25]{sp}{\nextFrameDescrSPvC}
\showRegister{sp}{\nextFrameDescrSPvD}
}{}

\ifthenelse{\step=9}{
\domainStateBacktrace[pos=(bottom),yFactor=0.03]{\exnHandlerAfterExnPopTrapvA}{\nextFrameDescrPCvE}{5}
\showRegister[opacity=0.25]{sp}{\nextFrameDescrSPvD}
\showRegister[offset=-0.8]{sp}{\nextFrameDescrSPvE}
}{}

% Drawing order matter for z-index trapsp/sp
\ifthenelse{\step>6 \and \step<10}{
\showRegister{pc}{\frameCamlReraiseExnEndvB}
\showRegister{trapsp}{\stashBacktraceArgTrapSPvB}
}{}


\ifthenelse{\step>10 \and \step<13}{
\draw [->,>=latex] (exnHandlerRight.east) to [out=0,in=0] (exnCtrap.east);
}{}

\ifthenelse{\step=12}{
\domainStateBacktrace[pos=(bottom),yFactor=0.03]{\exnHandlerAfterExnPopTrapvB}{\nextFrameDescrPCvF}{\stashBacktraceBacktracePosvC}
}{}

% Drawing order matter for z-index trapsp/sp
\ifthenelse{\step>10 \and \step<13}{
\showRegister{pc}{\frameCamlReraiseExnEndvC}
\showRegister{trapsp}{\stashBacktraceArgTrapSPvC}
}{}

\end{stackDiagram}
\end{tikzpicture}

    \end{column}
  \end{columns}
\end{frame}

\begin{frame}{caml\_domain\_state}
\begin{columns}
\begin{column}{0.5\textwidth}
\begin{listing}
\onslide*<1>{\mintc[lastline=32,highlightlines={7}]{src/ocaml/domain\_state.h}}
\onslide*<2>{
\begin{itemize}
\item \localname{backtrace\_active} ...
\end{itemize}
}
\end{listing}
\end{column}
\begin{column}{0.5\textwidth}
\begin{listing}
\onslide*<1>{\mintc[firstline=32]{src/ocaml/domain\_state.h}}
\onslide*<2->{\mintc[highlightlines={7}]{src/ocaml/domain\_state\_abbr.h}}
\caption{runtime/caml/domain\_state.h}
\end{listing}
\end{column}
\end{columns}
\end{frame}

\newcommand\listAmdSixtyFour[1][]{
  \listasm[firstline=702,lastline=725,#1]{../ocaml/runtime/amd64.S}{runtime/amd64.S}}
\begin{frame}{Introducing \funcname{caml\_raise\_exn}}
  \begin{columns}[c]
    \begin{column}{0.5\textwidth}
      \begin{itemize}
        \item<1-> First checks if exception backtrace should be recorded
          \begin{itemize}
            \item \funcname{OCAMLRUNPARAM}=b environment variable
            \item \mintinline{ocaml}{Printexc.record_backtrace : bool -> unit} \footnotemark
          \end{itemize}
        \item<2-> If not, standard trap behaviour
          \begin{itemize}
            \item Sets \regname{RSP} to \localname{domain\_state->exn\_handler} value
            \item Restores \localname{domain\_state->exn\_handler} from trap
            \item Jumps with \funcname{ret} (same effect as using \funcname{pop} and \funcname{jmp})
          \end{itemize}
        \item<3-> Switches stack to execute C code
        \item<4-> Calls \funcname{caml\_stash\_backtrace}, a C function of the runtime, with
          \begin{itemize}
            \item \funcarg{exn}: exception value
            \item \funcarg{pc}: code address of raising point
            \item \funcarg{sp}: lower stack address of raising function
            \item \funcarg{trapsp}: stack address of next handler
          \end{itemize}
      \end{itemize}
      \bigskip
      \mintocaml[firstline=9,lastline=13,highlightlines=12]{../src/backtrace/backtrace.ml}
    \end{column}
    \begin{column}{0.5\textwidth}
      \onslide*<1,3-4>{
        \mintc[firstline=190,lastline=192]{../ocaml/runtime/amd64.S}
        \mintc[firstline=234,lastline=237]{../ocaml/runtime/amd64.S}
      }
      \onslide*<2>{
        \mintc[firstline=190,lastline=192,highlightlines={191-192}]{../ocaml/runtime/amd64.S}
        \mintc[firstline=234,lastline=237,highlightlines={235-237}]{../ocaml/runtime/amd64.S}
      }
      \onslide*<1>{\listAmdSixtyFour[highlightlines=705]{}}
      \onslide*<2>{\listAmdSixtyFour[highlightlines={707-708}]{}}
      \onslide*<3>{\listAmdSixtyFour[highlightlines={712-714}]{}}
      \onslide*<4>{\listAmdSixtyFour[highlightlines={715-720}]{}}
      \onslide*<5>{\providecommand\step{2}\begin{tikzpicture}[font=\sffamily\tiny]
\input{../src/backtrace/gdb.out}

\newcommand\trapExnAEnd{\xinteval{\trapExnABegin - 2 * \wordSize}}
\newcommand\trapExnBEnd{\xinteval{\trapExnBBegin - 2 * \wordSize}}
\newcommand\trapExnCEnd{\xinteval{\trapExnCBegin - 2 * \wordSize}}

\begin{stackDiagram}[yFactor=0.35,showAddress=true]{\frameMainBegin}{\frameDefinitelyEnd}
\callStack

\begin{funFrame}[name=main,color=GreenYellow]{\frameMainBegin}{\frameMainEnd}
\funReturnAddr{\frameMainRetaddr}
\frameLocal{1}{\frameMainLocalOne}
\end{funFrame}

\ifthenelse{\step<13}{
\trapFrame[name=ExnC,handler=\trapExnCHandler,next=\trapExnCNext,color=WildStrawberry]{\trapExnCBegin}
\coordAtRight{exnCtrap}{\exnHandlerAfterExnCTrap}
}{}


\ifthenelse{\step<10}{
\trapFrame[name=ExnB,handler=\trapExnBHandler,next=\trapExnBNext,color=OrangeRed]{\trapExnBBegin}
\coordAtRight{exnBtrap}{\exnHandlerAfterExnBTrap}
\draw [->,>=latex] (exnBtrap.east) to [out=0,in=0] (exnCtrap.east);

\begin{funFrame}[name=maybe,color=YellowGreen]{\frameMaybeBegin}{\frameMaybeEnd}
\funReturnAddr{\frameMaybeRetaddr}
\frameLocal{1}{\frameMaybeLocalOne}
\end{funFrame}

\begin{funFrame}[name=probably,color=Green]{\frameProbablyBegin}{\frameProbablyEnd}
\funReturnAddr{\frameProbablyRetaddr}
\frameLocal{1}{\frameProbablyLocalOne}
\end{funFrame}
}{}


\ifthenelse{\step<5}{
\trapFrame[name=ExnA,handler=\trapExnAHandler,next=\trapExnANext,color=Red]{\trapExnABegin}
\coordAtRight{exnAtrap}{\exnHandlerAfterExnATrap}
\draw [->,>=latex] (exnAtrap.east) to [out=0,in=0] (exnBtrap.east);

\begin{funFrame}[name=definitely,color=JungleGreen]{\frameDefinitelyBegin}{\frameDefinitelyEnd}
\funReturnAddr{\frameDefinitelyRetaddr}
\frameLocal{1}{\frameDefinitelyLocalOne}
\end{funFrame}

\begin{funFrame}[color=SeaGreen]{\frameCamlRaiseExnBegin}{\frameCamlRaiseExnEnd}
\funReturnAddr{\frameCamlRaiseExnRetaddr}
\end{funFrame}
}{}


\ifthenelse{\step>5 \and \step<10}{
\begin{funFrame}[name=reraise,color=JungleGreen]{\frameCamlReraiseExnBeginvB}{\frameCamlReraiseExnEndvB}
\funReturnAddr{\frameCamlReraiseExnRetaddrvB}
\end{funFrame}
}{}

\ifthenelse{\step>10 \and \step<13}{
\begin{funFrame}[name=reraise,color=JungleGreen]{\frameCamlReraiseExnBeginvC}{\frameCamlReraiseExnEndvC}
\funReturnAddr{\frameCamlReraiseExnRetaddrvC}
\end{funFrame}
}{}


\coordinate (bottom) at ($(stackBL) + (0,-0.25)$);


\ifthenelse{\step=1 \or \step=2}{
\domainStateBacktrace[pos=(bottom),yFactor=0.03]{\exnHandlerAfterExnATrap}{}{0}
\draw [->,>=latex] (exnHandlerRight.east) to [out=0,in=0] (exnAtrap.east);
}{}

\ifthenelse{\step>4 \and \step<7}{
\domainStateBacktrace[pos=(bottom),yFactor=0.03]{\exnHandlerAfterExnPopTrapvA}{\nextFrameDescrPCvB}{\stashBacktraceBacktracePosvA}
}{}

\ifthenelse{\step>9 \and \step<12}{
\domainStateBacktrace[pos=(bottom),yFactor=0.03]{\exnHandlerAfterExnPopTrapvB}{\nextFrameDescrPCvE}{\stashBacktraceBacktracePosvB}
}{}

\ifthenelse{\step=5}{
\draw [->,>=latex] (exnHandlerRight.east) to [out=0,in=0] (exnBtrap.east);
}{}

\ifthenelse{\step=10}{
\draw [->,>=latex] (exnHandlerRight.east) to [out=0,in=0] (exnCtrap.east);
}{}

\ifthenelse{\step=2}{
\showRegister{sp}{\stashBacktraceArgSPvA}
}{}

\ifthenelse{\step=3}{
\domainStateBacktrace[pos=(bottom),yFactor=0.03]{\exnHandlerAfterExnATrap}{\nextFrameDescrPCvA}{1}
\showRegister{sp}{\nextFrameDescrSPvA}
}{}

\ifthenelse{\step=4}{
\domainStateBacktrace[pos=(bottom),yFactor=0.03]{\exnHandlerAfterExnATrap}{\nextFrameDescrPCvB}{2}
\showRegister[opacity=0.25]{sp}{\nextFrameDescrSPvA}
\showRegister[offset=-0.8]{sp}{\nextFrameDescrSPvB}
}{}

% Drawing order matter for z-index trapsp/sp
\ifthenelse{\step>1 \and \step<5}{
\showRegister{trapsp}{\stashBacktraceArgTrapSPvA}
\showRegister{pc}{\frameCamlRaiseExnEnd}
}{}

\ifthenelse{\step>5 \and \step<10}{
\draw [->,>=latex] (exnHandlerRight.east) to [out=0,in=0] (exnBtrap.east);
}{}

\ifthenelse{\step=7}{
\domainStateBacktrace[pos=(bottom),yFactor=0.03]{\exnHandlerAfterExnPopTrapvA}{\nextFrameDescrPCvC}{3}
\showRegister[opacity=0.25]{sp}{\stashBacktraceArgSPvB}
\showRegister{sp}{\nextFrameDescrSPvC}
}{}

\ifthenelse{\step=8}{
\domainStateBacktrace[pos=(bottom),yFactor=0.03]{\exnHandlerAfterExnPopTrapvA}{\nextFrameDescrPCvD}{4}
\showRegister[opacity=0.25]{sp}{\nextFrameDescrSPvC}
\showRegister{sp}{\nextFrameDescrSPvD}
}{}

\ifthenelse{\step=9}{
\domainStateBacktrace[pos=(bottom),yFactor=0.03]{\exnHandlerAfterExnPopTrapvA}{\nextFrameDescrPCvE}{5}
\showRegister[opacity=0.25]{sp}{\nextFrameDescrSPvD}
\showRegister[offset=-0.8]{sp}{\nextFrameDescrSPvE}
}{}

% Drawing order matter for z-index trapsp/sp
\ifthenelse{\step>6 \and \step<10}{
\showRegister{pc}{\frameCamlReraiseExnEndvB}
\showRegister{trapsp}{\stashBacktraceArgTrapSPvB}
}{}


\ifthenelse{\step>10 \and \step<13}{
\draw [->,>=latex] (exnHandlerRight.east) to [out=0,in=0] (exnCtrap.east);
}{}

\ifthenelse{\step=12}{
\domainStateBacktrace[pos=(bottom),yFactor=0.03]{\exnHandlerAfterExnPopTrapvB}{\nextFrameDescrPCvF}{\stashBacktraceBacktracePosvC}
}{}

% Drawing order matter for z-index trapsp/sp
\ifthenelse{\step>10 \and \step<13}{
\showRegister{pc}{\frameCamlReraiseExnEndvC}
\showRegister{trapsp}{\stashBacktraceArgTrapSPvC}
}{}

\end{stackDiagram}
\end{tikzpicture}
}
    \end{column}
  \end{columns}
  \footnotetext[1]{https://v2.ocaml.org/api/Printexc.html}
\end{frame}


\newcommand\listStashBacktrace[1][]{
  \listc[firstline=91,lastline=120,#1]{../ocaml/runtime/backtrace\_nat.c}{runtime/backtrace\_nat.c:91}}
\begin{frame}{Introducing \funcname{caml\_stash\_backtrace}}
  \begin{columns}[c]
    \begin{column}{0.5\textwidth}
      \begin{itemize}
        \item<1-> A C function provided by the runtime (native code specific)
        \item<1-> Purpose: scans the stack between two addresses in order to collect informations on discovered function frames
        \item<3-> Uses the same technique and informations that the GC use to scan the roots on the stack
        \item<3-> \typename{frame\_descr} holds lot of informations, only need the frame size here
        \item<4-> Store a pointer \typename{frame\_descr} to each discovered in \localname{domain\_state->backtrace\_buffer}
      \end{itemize}
      \onslide*<2>{
        \smallskip
        \listc[firstline=85,lastline=90]{../ocaml/runtime/backtrace\_nat.c}{runtime/backtrace\_nat.c:85}
      }
      \onslide*<3>{
        \smallskip
        \listocaml[firstline=111,lastline=124]{../ocaml/asmcomp/emitaux.ml}{asmcomp/emitaux.ml:111}
      }
      \onslide*<1-2,4->{
        \bigskip
        \mintocaml[firstline=9,lastline=13,highlightlines=12]{../src/backtrace/backtrace.ml}
      }
    \end{column}
    \begin{column}{0.5\textwidth}
      \onslide*<1-2>{\listStashBacktrace{}}
      \onslide*<3>{\listStashBacktrace[highlightlines={94,106,109-110}]{}}
      \onslide*<4>{\providecommand\step{2}\begin{tikzpicture}[font=\sffamily\tiny]
\input{../src/backtrace/gdb.out}

\newcommand\trapExnAEnd{\xinteval{\trapExnABegin - 2 * \wordSize}}
\newcommand\trapExnBEnd{\xinteval{\trapExnBBegin - 2 * \wordSize}}
\newcommand\trapExnCEnd{\xinteval{\trapExnCBegin - 2 * \wordSize}}

\begin{stackDiagram}[yFactor=0.35,showAddress=true]{\frameMainBegin}{\frameDefinitelyEnd}
\callStack

\begin{funFrame}[name=main,color=GreenYellow]{\frameMainBegin}{\frameMainEnd}
\funReturnAddr{\frameMainRetaddr}
\frameLocal{1}{\frameMainLocalOne}
\end{funFrame}

\ifthenelse{\step<13}{
\trapFrame[name=ExnC,handler=\trapExnCHandler,next=\trapExnCNext,color=WildStrawberry]{\trapExnCBegin}
\coordAtRight{exnCtrap}{\exnHandlerAfterExnCTrap}
}{}


\ifthenelse{\step<10}{
\trapFrame[name=ExnB,handler=\trapExnBHandler,next=\trapExnBNext,color=OrangeRed]{\trapExnBBegin}
\coordAtRight{exnBtrap}{\exnHandlerAfterExnBTrap}
\draw [->,>=latex] (exnBtrap.east) to [out=0,in=0] (exnCtrap.east);

\begin{funFrame}[name=maybe,color=YellowGreen]{\frameMaybeBegin}{\frameMaybeEnd}
\funReturnAddr{\frameMaybeRetaddr}
\frameLocal{1}{\frameMaybeLocalOne}
\end{funFrame}

\begin{funFrame}[name=probably,color=Green]{\frameProbablyBegin}{\frameProbablyEnd}
\funReturnAddr{\frameProbablyRetaddr}
\frameLocal{1}{\frameProbablyLocalOne}
\end{funFrame}
}{}


\ifthenelse{\step<5}{
\trapFrame[name=ExnA,handler=\trapExnAHandler,next=\trapExnANext,color=Red]{\trapExnABegin}
\coordAtRight{exnAtrap}{\exnHandlerAfterExnATrap}
\draw [->,>=latex] (exnAtrap.east) to [out=0,in=0] (exnBtrap.east);

\begin{funFrame}[name=definitely,color=JungleGreen]{\frameDefinitelyBegin}{\frameDefinitelyEnd}
\funReturnAddr{\frameDefinitelyRetaddr}
\frameLocal{1}{\frameDefinitelyLocalOne}
\end{funFrame}

\begin{funFrame}[color=SeaGreen]{\frameCamlRaiseExnBegin}{\frameCamlRaiseExnEnd}
\funReturnAddr{\frameCamlRaiseExnRetaddr}
\end{funFrame}
}{}


\ifthenelse{\step>5 \and \step<10}{
\begin{funFrame}[name=reraise,color=JungleGreen]{\frameCamlReraiseExnBeginvB}{\frameCamlReraiseExnEndvB}
\funReturnAddr{\frameCamlReraiseExnRetaddrvB}
\end{funFrame}
}{}

\ifthenelse{\step>10 \and \step<13}{
\begin{funFrame}[name=reraise,color=JungleGreen]{\frameCamlReraiseExnBeginvC}{\frameCamlReraiseExnEndvC}
\funReturnAddr{\frameCamlReraiseExnRetaddrvC}
\end{funFrame}
}{}


\coordinate (bottom) at ($(stackBL) + (0,-0.25)$);


\ifthenelse{\step=1 \or \step=2}{
\domainStateBacktrace[pos=(bottom),yFactor=0.03]{\exnHandlerAfterExnATrap}{}{0}
\draw [->,>=latex] (exnHandlerRight.east) to [out=0,in=0] (exnAtrap.east);
}{}

\ifthenelse{\step>4 \and \step<7}{
\domainStateBacktrace[pos=(bottom),yFactor=0.03]{\exnHandlerAfterExnPopTrapvA}{\nextFrameDescrPCvB}{\stashBacktraceBacktracePosvA}
}{}

\ifthenelse{\step>9 \and \step<12}{
\domainStateBacktrace[pos=(bottom),yFactor=0.03]{\exnHandlerAfterExnPopTrapvB}{\nextFrameDescrPCvE}{\stashBacktraceBacktracePosvB}
}{}

\ifthenelse{\step=5}{
\draw [->,>=latex] (exnHandlerRight.east) to [out=0,in=0] (exnBtrap.east);
}{}

\ifthenelse{\step=10}{
\draw [->,>=latex] (exnHandlerRight.east) to [out=0,in=0] (exnCtrap.east);
}{}

\ifthenelse{\step=2}{
\showRegister{sp}{\stashBacktraceArgSPvA}
}{}

\ifthenelse{\step=3}{
\domainStateBacktrace[pos=(bottom),yFactor=0.03]{\exnHandlerAfterExnATrap}{\nextFrameDescrPCvA}{1}
\showRegister{sp}{\nextFrameDescrSPvA}
}{}

\ifthenelse{\step=4}{
\domainStateBacktrace[pos=(bottom),yFactor=0.03]{\exnHandlerAfterExnATrap}{\nextFrameDescrPCvB}{2}
\showRegister[opacity=0.25]{sp}{\nextFrameDescrSPvA}
\showRegister[offset=-0.8]{sp}{\nextFrameDescrSPvB}
}{}

% Drawing order matter for z-index trapsp/sp
\ifthenelse{\step>1 \and \step<5}{
\showRegister{trapsp}{\stashBacktraceArgTrapSPvA}
\showRegister{pc}{\frameCamlRaiseExnEnd}
}{}

\ifthenelse{\step>5 \and \step<10}{
\draw [->,>=latex] (exnHandlerRight.east) to [out=0,in=0] (exnBtrap.east);
}{}

\ifthenelse{\step=7}{
\domainStateBacktrace[pos=(bottom),yFactor=0.03]{\exnHandlerAfterExnPopTrapvA}{\nextFrameDescrPCvC}{3}
\showRegister[opacity=0.25]{sp}{\stashBacktraceArgSPvB}
\showRegister{sp}{\nextFrameDescrSPvC}
}{}

\ifthenelse{\step=8}{
\domainStateBacktrace[pos=(bottom),yFactor=0.03]{\exnHandlerAfterExnPopTrapvA}{\nextFrameDescrPCvD}{4}
\showRegister[opacity=0.25]{sp}{\nextFrameDescrSPvC}
\showRegister{sp}{\nextFrameDescrSPvD}
}{}

\ifthenelse{\step=9}{
\domainStateBacktrace[pos=(bottom),yFactor=0.03]{\exnHandlerAfterExnPopTrapvA}{\nextFrameDescrPCvE}{5}
\showRegister[opacity=0.25]{sp}{\nextFrameDescrSPvD}
\showRegister[offset=-0.8]{sp}{\nextFrameDescrSPvE}
}{}

% Drawing order matter for z-index trapsp/sp
\ifthenelse{\step>6 \and \step<10}{
\showRegister{pc}{\frameCamlReraiseExnEndvB}
\showRegister{trapsp}{\stashBacktraceArgTrapSPvB}
}{}


\ifthenelse{\step>10 \and \step<13}{
\draw [->,>=latex] (exnHandlerRight.east) to [out=0,in=0] (exnCtrap.east);
}{}

\ifthenelse{\step=12}{
\domainStateBacktrace[pos=(bottom),yFactor=0.03]{\exnHandlerAfterExnPopTrapvB}{\nextFrameDescrPCvF}{\stashBacktraceBacktracePosvC}
}{}

% Drawing order matter for z-index trapsp/sp
\ifthenelse{\step>10 \and \step<13}{
\showRegister{pc}{\frameCamlReraiseExnEndvC}
\showRegister{trapsp}{\stashBacktraceArgTrapSPvC}
}{}

\end{stackDiagram}
\end{tikzpicture}
}
      \onslide*<5>{\providecommand\step{3}\begin{tikzpicture}[font=\sffamily\tiny]
\input{../src/backtrace/gdb.out}

\newcommand\trapExnAEnd{\xinteval{\trapExnABegin - 2 * \wordSize}}
\newcommand\trapExnBEnd{\xinteval{\trapExnBBegin - 2 * \wordSize}}
\newcommand\trapExnCEnd{\xinteval{\trapExnCBegin - 2 * \wordSize}}

\begin{stackDiagram}[yFactor=0.35,showAddress=true]{\frameMainBegin}{\frameDefinitelyEnd}
\callStack

\begin{funFrame}[name=main,color=GreenYellow]{\frameMainBegin}{\frameMainEnd}
\funReturnAddr{\frameMainRetaddr}
\frameLocal{1}{\frameMainLocalOne}
\end{funFrame}

\ifthenelse{\step<13}{
\trapFrame[name=ExnC,handler=\trapExnCHandler,next=\trapExnCNext,color=WildStrawberry]{\trapExnCBegin}
\coordAtRight{exnCtrap}{\exnHandlerAfterExnCTrap}
}{}


\ifthenelse{\step<10}{
\trapFrame[name=ExnB,handler=\trapExnBHandler,next=\trapExnBNext,color=OrangeRed]{\trapExnBBegin}
\coordAtRight{exnBtrap}{\exnHandlerAfterExnBTrap}
\draw [->,>=latex] (exnBtrap.east) to [out=0,in=0] (exnCtrap.east);

\begin{funFrame}[name=maybe,color=YellowGreen]{\frameMaybeBegin}{\frameMaybeEnd}
\funReturnAddr{\frameMaybeRetaddr}
\frameLocal{1}{\frameMaybeLocalOne}
\end{funFrame}

\begin{funFrame}[name=probably,color=Green]{\frameProbablyBegin}{\frameProbablyEnd}
\funReturnAddr{\frameProbablyRetaddr}
\frameLocal{1}{\frameProbablyLocalOne}
\end{funFrame}
}{}


\ifthenelse{\step<5}{
\trapFrame[name=ExnA,handler=\trapExnAHandler,next=\trapExnANext,color=Red]{\trapExnABegin}
\coordAtRight{exnAtrap}{\exnHandlerAfterExnATrap}
\draw [->,>=latex] (exnAtrap.east) to [out=0,in=0] (exnBtrap.east);

\begin{funFrame}[name=definitely,color=JungleGreen]{\frameDefinitelyBegin}{\frameDefinitelyEnd}
\funReturnAddr{\frameDefinitelyRetaddr}
\frameLocal{1}{\frameDefinitelyLocalOne}
\end{funFrame}

\begin{funFrame}[color=SeaGreen]{\frameCamlRaiseExnBegin}{\frameCamlRaiseExnEnd}
\funReturnAddr{\frameCamlRaiseExnRetaddr}
\end{funFrame}
}{}


\ifthenelse{\step>5 \and \step<10}{
\begin{funFrame}[name=reraise,color=JungleGreen]{\frameCamlReraiseExnBeginvB}{\frameCamlReraiseExnEndvB}
\funReturnAddr{\frameCamlReraiseExnRetaddrvB}
\end{funFrame}
}{}

\ifthenelse{\step>10 \and \step<13}{
\begin{funFrame}[name=reraise,color=JungleGreen]{\frameCamlReraiseExnBeginvC}{\frameCamlReraiseExnEndvC}
\funReturnAddr{\frameCamlReraiseExnRetaddrvC}
\end{funFrame}
}{}


\coordinate (bottom) at ($(stackBL) + (0,-0.25)$);


\ifthenelse{\step=1 \or \step=2}{
\domainStateBacktrace[pos=(bottom),yFactor=0.03]{\exnHandlerAfterExnATrap}{}{0}
\draw [->,>=latex] (exnHandlerRight.east) to [out=0,in=0] (exnAtrap.east);
}{}

\ifthenelse{\step>4 \and \step<7}{
\domainStateBacktrace[pos=(bottom),yFactor=0.03]{\exnHandlerAfterExnPopTrapvA}{\nextFrameDescrPCvB}{\stashBacktraceBacktracePosvA}
}{}

\ifthenelse{\step>9 \and \step<12}{
\domainStateBacktrace[pos=(bottom),yFactor=0.03]{\exnHandlerAfterExnPopTrapvB}{\nextFrameDescrPCvE}{\stashBacktraceBacktracePosvB}
}{}

\ifthenelse{\step=5}{
\draw [->,>=latex] (exnHandlerRight.east) to [out=0,in=0] (exnBtrap.east);
}{}

\ifthenelse{\step=10}{
\draw [->,>=latex] (exnHandlerRight.east) to [out=0,in=0] (exnCtrap.east);
}{}

\ifthenelse{\step=2}{
\showRegister{sp}{\stashBacktraceArgSPvA}
}{}

\ifthenelse{\step=3}{
\domainStateBacktrace[pos=(bottom),yFactor=0.03]{\exnHandlerAfterExnATrap}{\nextFrameDescrPCvA}{1}
\showRegister{sp}{\nextFrameDescrSPvA}
}{}

\ifthenelse{\step=4}{
\domainStateBacktrace[pos=(bottom),yFactor=0.03]{\exnHandlerAfterExnATrap}{\nextFrameDescrPCvB}{2}
\showRegister[opacity=0.25]{sp}{\nextFrameDescrSPvA}
\showRegister[offset=-0.8]{sp}{\nextFrameDescrSPvB}
}{}

% Drawing order matter for z-index trapsp/sp
\ifthenelse{\step>1 \and \step<5}{
\showRegister{trapsp}{\stashBacktraceArgTrapSPvA}
\showRegister{pc}{\frameCamlRaiseExnEnd}
}{}

\ifthenelse{\step>5 \and \step<10}{
\draw [->,>=latex] (exnHandlerRight.east) to [out=0,in=0] (exnBtrap.east);
}{}

\ifthenelse{\step=7}{
\domainStateBacktrace[pos=(bottom),yFactor=0.03]{\exnHandlerAfterExnPopTrapvA}{\nextFrameDescrPCvC}{3}
\showRegister[opacity=0.25]{sp}{\stashBacktraceArgSPvB}
\showRegister{sp}{\nextFrameDescrSPvC}
}{}

\ifthenelse{\step=8}{
\domainStateBacktrace[pos=(bottom),yFactor=0.03]{\exnHandlerAfterExnPopTrapvA}{\nextFrameDescrPCvD}{4}
\showRegister[opacity=0.25]{sp}{\nextFrameDescrSPvC}
\showRegister{sp}{\nextFrameDescrSPvD}
}{}

\ifthenelse{\step=9}{
\domainStateBacktrace[pos=(bottom),yFactor=0.03]{\exnHandlerAfterExnPopTrapvA}{\nextFrameDescrPCvE}{5}
\showRegister[opacity=0.25]{sp}{\nextFrameDescrSPvD}
\showRegister[offset=-0.8]{sp}{\nextFrameDescrSPvE}
}{}

% Drawing order matter for z-index trapsp/sp
\ifthenelse{\step>6 \and \step<10}{
\showRegister{pc}{\frameCamlReraiseExnEndvB}
\showRegister{trapsp}{\stashBacktraceArgTrapSPvB}
}{}


\ifthenelse{\step>10 \and \step<13}{
\draw [->,>=latex] (exnHandlerRight.east) to [out=0,in=0] (exnCtrap.east);
}{}

\ifthenelse{\step=12}{
\domainStateBacktrace[pos=(bottom),yFactor=0.03]{\exnHandlerAfterExnPopTrapvB}{\nextFrameDescrPCvF}{\stashBacktraceBacktracePosvC}
}{}

% Drawing order matter for z-index trapsp/sp
\ifthenelse{\step>10 \and \step<13}{
\showRegister{pc}{\frameCamlReraiseExnEndvC}
\showRegister{trapsp}{\stashBacktraceArgTrapSPvC}
}{}

\end{stackDiagram}
\end{tikzpicture}
}
      \onslide*<6>{\providecommand\step{4}\begin{tikzpicture}[font=\sffamily\tiny]
\input{../src/backtrace/gdb.out}

\newcommand\trapExnAEnd{\xinteval{\trapExnABegin - 2 * \wordSize}}
\newcommand\trapExnBEnd{\xinteval{\trapExnBBegin - 2 * \wordSize}}
\newcommand\trapExnCEnd{\xinteval{\trapExnCBegin - 2 * \wordSize}}

\begin{stackDiagram}[yFactor=0.35,showAddress=true]{\frameMainBegin}{\frameDefinitelyEnd}
\callStack

\begin{funFrame}[name=main,color=GreenYellow]{\frameMainBegin}{\frameMainEnd}
\funReturnAddr{\frameMainRetaddr}
\frameLocal{1}{\frameMainLocalOne}
\end{funFrame}

\ifthenelse{\step<13}{
\trapFrame[name=ExnC,handler=\trapExnCHandler,next=\trapExnCNext,color=WildStrawberry]{\trapExnCBegin}
\coordAtRight{exnCtrap}{\exnHandlerAfterExnCTrap}
}{}


\ifthenelse{\step<10}{
\trapFrame[name=ExnB,handler=\trapExnBHandler,next=\trapExnBNext,color=OrangeRed]{\trapExnBBegin}
\coordAtRight{exnBtrap}{\exnHandlerAfterExnBTrap}
\draw [->,>=latex] (exnBtrap.east) to [out=0,in=0] (exnCtrap.east);

\begin{funFrame}[name=maybe,color=YellowGreen]{\frameMaybeBegin}{\frameMaybeEnd}
\funReturnAddr{\frameMaybeRetaddr}
\frameLocal{1}{\frameMaybeLocalOne}
\end{funFrame}

\begin{funFrame}[name=probably,color=Green]{\frameProbablyBegin}{\frameProbablyEnd}
\funReturnAddr{\frameProbablyRetaddr}
\frameLocal{1}{\frameProbablyLocalOne}
\end{funFrame}
}{}


\ifthenelse{\step<5}{
\trapFrame[name=ExnA,handler=\trapExnAHandler,next=\trapExnANext,color=Red]{\trapExnABegin}
\coordAtRight{exnAtrap}{\exnHandlerAfterExnATrap}
\draw [->,>=latex] (exnAtrap.east) to [out=0,in=0] (exnBtrap.east);

\begin{funFrame}[name=definitely,color=JungleGreen]{\frameDefinitelyBegin}{\frameDefinitelyEnd}
\funReturnAddr{\frameDefinitelyRetaddr}
\frameLocal{1}{\frameDefinitelyLocalOne}
\end{funFrame}

\begin{funFrame}[color=SeaGreen]{\frameCamlRaiseExnBegin}{\frameCamlRaiseExnEnd}
\funReturnAddr{\frameCamlRaiseExnRetaddr}
\end{funFrame}
}{}


\ifthenelse{\step>5 \and \step<10}{
\begin{funFrame}[name=reraise,color=JungleGreen]{\frameCamlReraiseExnBeginvB}{\frameCamlReraiseExnEndvB}
\funReturnAddr{\frameCamlReraiseExnRetaddrvB}
\end{funFrame}
}{}

\ifthenelse{\step>10 \and \step<13}{
\begin{funFrame}[name=reraise,color=JungleGreen]{\frameCamlReraiseExnBeginvC}{\frameCamlReraiseExnEndvC}
\funReturnAddr{\frameCamlReraiseExnRetaddrvC}
\end{funFrame}
}{}


\coordinate (bottom) at ($(stackBL) + (0,-0.25)$);


\ifthenelse{\step=1 \or \step=2}{
\domainStateBacktrace[pos=(bottom),yFactor=0.03]{\exnHandlerAfterExnATrap}{}{0}
\draw [->,>=latex] (exnHandlerRight.east) to [out=0,in=0] (exnAtrap.east);
}{}

\ifthenelse{\step>4 \and \step<7}{
\domainStateBacktrace[pos=(bottom),yFactor=0.03]{\exnHandlerAfterExnPopTrapvA}{\nextFrameDescrPCvB}{\stashBacktraceBacktracePosvA}
}{}

\ifthenelse{\step>9 \and \step<12}{
\domainStateBacktrace[pos=(bottom),yFactor=0.03]{\exnHandlerAfterExnPopTrapvB}{\nextFrameDescrPCvE}{\stashBacktraceBacktracePosvB}
}{}

\ifthenelse{\step=5}{
\draw [->,>=latex] (exnHandlerRight.east) to [out=0,in=0] (exnBtrap.east);
}{}

\ifthenelse{\step=10}{
\draw [->,>=latex] (exnHandlerRight.east) to [out=0,in=0] (exnCtrap.east);
}{}

\ifthenelse{\step=2}{
\showRegister{sp}{\stashBacktraceArgSPvA}
}{}

\ifthenelse{\step=3}{
\domainStateBacktrace[pos=(bottom),yFactor=0.03]{\exnHandlerAfterExnATrap}{\nextFrameDescrPCvA}{1}
\showRegister{sp}{\nextFrameDescrSPvA}
}{}

\ifthenelse{\step=4}{
\domainStateBacktrace[pos=(bottom),yFactor=0.03]{\exnHandlerAfterExnATrap}{\nextFrameDescrPCvB}{2}
\showRegister[opacity=0.25]{sp}{\nextFrameDescrSPvA}
\showRegister[offset=-0.8]{sp}{\nextFrameDescrSPvB}
}{}

% Drawing order matter for z-index trapsp/sp
\ifthenelse{\step>1 \and \step<5}{
\showRegister{trapsp}{\stashBacktraceArgTrapSPvA}
\showRegister{pc}{\frameCamlRaiseExnEnd}
}{}

\ifthenelse{\step>5 \and \step<10}{
\draw [->,>=latex] (exnHandlerRight.east) to [out=0,in=0] (exnBtrap.east);
}{}

\ifthenelse{\step=7}{
\domainStateBacktrace[pos=(bottom),yFactor=0.03]{\exnHandlerAfterExnPopTrapvA}{\nextFrameDescrPCvC}{3}
\showRegister[opacity=0.25]{sp}{\stashBacktraceArgSPvB}
\showRegister{sp}{\nextFrameDescrSPvC}
}{}

\ifthenelse{\step=8}{
\domainStateBacktrace[pos=(bottom),yFactor=0.03]{\exnHandlerAfterExnPopTrapvA}{\nextFrameDescrPCvD}{4}
\showRegister[opacity=0.25]{sp}{\nextFrameDescrSPvC}
\showRegister{sp}{\nextFrameDescrSPvD}
}{}

\ifthenelse{\step=9}{
\domainStateBacktrace[pos=(bottom),yFactor=0.03]{\exnHandlerAfterExnPopTrapvA}{\nextFrameDescrPCvE}{5}
\showRegister[opacity=0.25]{sp}{\nextFrameDescrSPvD}
\showRegister[offset=-0.8]{sp}{\nextFrameDescrSPvE}
}{}

% Drawing order matter for z-index trapsp/sp
\ifthenelse{\step>6 \and \step<10}{
\showRegister{pc}{\frameCamlReraiseExnEndvB}
\showRegister{trapsp}{\stashBacktraceArgTrapSPvB}
}{}


\ifthenelse{\step>10 \and \step<13}{
\draw [->,>=latex] (exnHandlerRight.east) to [out=0,in=0] (exnCtrap.east);
}{}

\ifthenelse{\step=12}{
\domainStateBacktrace[pos=(bottom),yFactor=0.03]{\exnHandlerAfterExnPopTrapvB}{\nextFrameDescrPCvF}{\stashBacktraceBacktracePosvC}
}{}

% Drawing order matter for z-index trapsp/sp
\ifthenelse{\step>10 \and \step<13}{
\showRegister{pc}{\frameCamlReraiseExnEndvC}
\showRegister{trapsp}{\stashBacktraceArgTrapSPvC}
}{}

\end{stackDiagram}
\end{tikzpicture}
}
    \end{column}
  \end{columns}
\end{frame}

\begin{frame}{Control returns to \funcname{caml\_raise\_exn}}
  \begin{columns}[c]
    \begin{column}{0.5\textwidth}
      \onslide*<1>{
        \begin{itemize}
          \item \funcname{caml\_stash\_backtrace} completed
          \item Pops the trap and jump to \typename{ExnA} handler's code
        \end{itemize}
      }
      \onslide*<2>{\listAmdSixtyFour[highlightlines={722-723}]}
      \bigskip
      \mintocaml[firstline=9,lastline=13,highlightlines=12]{../src/backtrace/backtrace.ml}
    \end{column}
    \begin{column}{0.5\textwidth}
      \providecommand\step{5}\begin{tikzpicture}[font=\sffamily\tiny]
\input{../src/backtrace/gdb.out}

\newcommand\trapExnAEnd{\xinteval{\trapExnABegin - 2 * \wordSize}}
\newcommand\trapExnBEnd{\xinteval{\trapExnBBegin - 2 * \wordSize}}
\newcommand\trapExnCEnd{\xinteval{\trapExnCBegin - 2 * \wordSize}}

\begin{stackDiagram}[yFactor=0.35,showAddress=true]{\frameMainBegin}{\frameDefinitelyEnd}
\callStack

\begin{funFrame}[name=main,color=GreenYellow]{\frameMainBegin}{\frameMainEnd}
\funReturnAddr{\frameMainRetaddr}
\frameLocal{1}{\frameMainLocalOne}
\end{funFrame}

\ifthenelse{\step<13}{
\trapFrame[name=ExnC,handler=\trapExnCHandler,next=\trapExnCNext,color=WildStrawberry]{\trapExnCBegin}
\coordAtRight{exnCtrap}{\exnHandlerAfterExnCTrap}
}{}


\ifthenelse{\step<10}{
\trapFrame[name=ExnB,handler=\trapExnBHandler,next=\trapExnBNext,color=OrangeRed]{\trapExnBBegin}
\coordAtRight{exnBtrap}{\exnHandlerAfterExnBTrap}
\draw [->,>=latex] (exnBtrap.east) to [out=0,in=0] (exnCtrap.east);

\begin{funFrame}[name=maybe,color=YellowGreen]{\frameMaybeBegin}{\frameMaybeEnd}
\funReturnAddr{\frameMaybeRetaddr}
\frameLocal{1}{\frameMaybeLocalOne}
\end{funFrame}

\begin{funFrame}[name=probably,color=Green]{\frameProbablyBegin}{\frameProbablyEnd}
\funReturnAddr{\frameProbablyRetaddr}
\frameLocal{1}{\frameProbablyLocalOne}
\end{funFrame}
}{}


\ifthenelse{\step<5}{
\trapFrame[name=ExnA,handler=\trapExnAHandler,next=\trapExnANext,color=Red]{\trapExnABegin}
\coordAtRight{exnAtrap}{\exnHandlerAfterExnATrap}
\draw [->,>=latex] (exnAtrap.east) to [out=0,in=0] (exnBtrap.east);

\begin{funFrame}[name=definitely,color=JungleGreen]{\frameDefinitelyBegin}{\frameDefinitelyEnd}
\funReturnAddr{\frameDefinitelyRetaddr}
\frameLocal{1}{\frameDefinitelyLocalOne}
\end{funFrame}

\begin{funFrame}[color=SeaGreen]{\frameCamlRaiseExnBegin}{\frameCamlRaiseExnEnd}
\funReturnAddr{\frameCamlRaiseExnRetaddr}
\end{funFrame}
}{}


\ifthenelse{\step>5 \and \step<10}{
\begin{funFrame}[name=reraise,color=JungleGreen]{\frameCamlReraiseExnBeginvB}{\frameCamlReraiseExnEndvB}
\funReturnAddr{\frameCamlReraiseExnRetaddrvB}
\end{funFrame}
}{}

\ifthenelse{\step>10 \and \step<13}{
\begin{funFrame}[name=reraise,color=JungleGreen]{\frameCamlReraiseExnBeginvC}{\frameCamlReraiseExnEndvC}
\funReturnAddr{\frameCamlReraiseExnRetaddrvC}
\end{funFrame}
}{}


\coordinate (bottom) at ($(stackBL) + (0,-0.25)$);


\ifthenelse{\step=1 \or \step=2}{
\domainStateBacktrace[pos=(bottom),yFactor=0.03]{\exnHandlerAfterExnATrap}{}{0}
\draw [->,>=latex] (exnHandlerRight.east) to [out=0,in=0] (exnAtrap.east);
}{}

\ifthenelse{\step>4 \and \step<7}{
\domainStateBacktrace[pos=(bottom),yFactor=0.03]{\exnHandlerAfterExnPopTrapvA}{\nextFrameDescrPCvB}{\stashBacktraceBacktracePosvA}
}{}

\ifthenelse{\step>9 \and \step<12}{
\domainStateBacktrace[pos=(bottom),yFactor=0.03]{\exnHandlerAfterExnPopTrapvB}{\nextFrameDescrPCvE}{\stashBacktraceBacktracePosvB}
}{}

\ifthenelse{\step=5}{
\draw [->,>=latex] (exnHandlerRight.east) to [out=0,in=0] (exnBtrap.east);
}{}

\ifthenelse{\step=10}{
\draw [->,>=latex] (exnHandlerRight.east) to [out=0,in=0] (exnCtrap.east);
}{}

\ifthenelse{\step=2}{
\showRegister{sp}{\stashBacktraceArgSPvA}
}{}

\ifthenelse{\step=3}{
\domainStateBacktrace[pos=(bottom),yFactor=0.03]{\exnHandlerAfterExnATrap}{\nextFrameDescrPCvA}{1}
\showRegister{sp}{\nextFrameDescrSPvA}
}{}

\ifthenelse{\step=4}{
\domainStateBacktrace[pos=(bottom),yFactor=0.03]{\exnHandlerAfterExnATrap}{\nextFrameDescrPCvB}{2}
\showRegister[opacity=0.25]{sp}{\nextFrameDescrSPvA}
\showRegister[offset=-0.8]{sp}{\nextFrameDescrSPvB}
}{}

% Drawing order matter for z-index trapsp/sp
\ifthenelse{\step>1 \and \step<5}{
\showRegister{trapsp}{\stashBacktraceArgTrapSPvA}
\showRegister{pc}{\frameCamlRaiseExnEnd}
}{}

\ifthenelse{\step>5 \and \step<10}{
\draw [->,>=latex] (exnHandlerRight.east) to [out=0,in=0] (exnBtrap.east);
}{}

\ifthenelse{\step=7}{
\domainStateBacktrace[pos=(bottom),yFactor=0.03]{\exnHandlerAfterExnPopTrapvA}{\nextFrameDescrPCvC}{3}
\showRegister[opacity=0.25]{sp}{\stashBacktraceArgSPvB}
\showRegister{sp}{\nextFrameDescrSPvC}
}{}

\ifthenelse{\step=8}{
\domainStateBacktrace[pos=(bottom),yFactor=0.03]{\exnHandlerAfterExnPopTrapvA}{\nextFrameDescrPCvD}{4}
\showRegister[opacity=0.25]{sp}{\nextFrameDescrSPvC}
\showRegister{sp}{\nextFrameDescrSPvD}
}{}

\ifthenelse{\step=9}{
\domainStateBacktrace[pos=(bottom),yFactor=0.03]{\exnHandlerAfterExnPopTrapvA}{\nextFrameDescrPCvE}{5}
\showRegister[opacity=0.25]{sp}{\nextFrameDescrSPvD}
\showRegister[offset=-0.8]{sp}{\nextFrameDescrSPvE}
}{}

% Drawing order matter for z-index trapsp/sp
\ifthenelse{\step>6 \and \step<10}{
\showRegister{pc}{\frameCamlReraiseExnEndvB}
\showRegister{trapsp}{\stashBacktraceArgTrapSPvB}
}{}


\ifthenelse{\step>10 \and \step<13}{
\draw [->,>=latex] (exnHandlerRight.east) to [out=0,in=0] (exnCtrap.east);
}{}

\ifthenelse{\step=12}{
\domainStateBacktrace[pos=(bottom),yFactor=0.03]{\exnHandlerAfterExnPopTrapvB}{\nextFrameDescrPCvF}{\stashBacktraceBacktracePosvC}
}{}

% Drawing order matter for z-index trapsp/sp
\ifthenelse{\step>10 \and \step<13}{
\showRegister{pc}{\frameCamlReraiseExnEndvC}
\showRegister{trapsp}{\stashBacktraceArgTrapSPvC}
}{}

\end{stackDiagram}
\end{tikzpicture}

    \end{column}
  \end{columns}
\end{frame}

\begin{frame}{\typename{ExnA} handler doesn't match}
  \begin{columns}[c]
    \begin{column}{0.5\textwidth}
      \onslide*<1,3>{
        \begin{itemize}
          \item<1-> Fancy match\_with
          \item<1-> No match
          \item<3-> Reraise
        \end{itemize}
      }
      \onslide*<2>{\mintcmm[highlightlines={10-18}]{../src/backtrace/camlBacktrace\_\_probably\_raise\_351.cmm}}
      \medskip
      \onslide*<1->{\mintocaml[firstline=15,lastline=21,highlightlines={18-21}]{../src/backtrace/backtrace.ml}}
    \end{column}
    \begin{column}{0.5\textwidth}
      \onslide*<1>{\providecommand\step{5}\begin{tikzpicture}[font=\sffamily\tiny]
\input{../src/backtrace/gdb.out}

\newcommand\trapExnAEnd{\xinteval{\trapExnABegin - 2 * \wordSize}}
\newcommand\trapExnBEnd{\xinteval{\trapExnBBegin - 2 * \wordSize}}
\newcommand\trapExnCEnd{\xinteval{\trapExnCBegin - 2 * \wordSize}}

\begin{stackDiagram}[yFactor=0.35,showAddress=true]{\frameMainBegin}{\frameDefinitelyEnd}
\callStack

\begin{funFrame}[name=main,color=GreenYellow]{\frameMainBegin}{\frameMainEnd}
\funReturnAddr{\frameMainRetaddr}
\frameLocal{1}{\frameMainLocalOne}
\end{funFrame}

\ifthenelse{\step<13}{
\trapFrame[name=ExnC,handler=\trapExnCHandler,next=\trapExnCNext,color=WildStrawberry]{\trapExnCBegin}
\coordAtRight{exnCtrap}{\exnHandlerAfterExnCTrap}
}{}


\ifthenelse{\step<10}{
\trapFrame[name=ExnB,handler=\trapExnBHandler,next=\trapExnBNext,color=OrangeRed]{\trapExnBBegin}
\coordAtRight{exnBtrap}{\exnHandlerAfterExnBTrap}
\draw [->,>=latex] (exnBtrap.east) to [out=0,in=0] (exnCtrap.east);

\begin{funFrame}[name=maybe,color=YellowGreen]{\frameMaybeBegin}{\frameMaybeEnd}
\funReturnAddr{\frameMaybeRetaddr}
\frameLocal{1}{\frameMaybeLocalOne}
\end{funFrame}

\begin{funFrame}[name=probably,color=Green]{\frameProbablyBegin}{\frameProbablyEnd}
\funReturnAddr{\frameProbablyRetaddr}
\frameLocal{1}{\frameProbablyLocalOne}
\end{funFrame}
}{}


\ifthenelse{\step<5}{
\trapFrame[name=ExnA,handler=\trapExnAHandler,next=\trapExnANext,color=Red]{\trapExnABegin}
\coordAtRight{exnAtrap}{\exnHandlerAfterExnATrap}
\draw [->,>=latex] (exnAtrap.east) to [out=0,in=0] (exnBtrap.east);

\begin{funFrame}[name=definitely,color=JungleGreen]{\frameDefinitelyBegin}{\frameDefinitelyEnd}
\funReturnAddr{\frameDefinitelyRetaddr}
\frameLocal{1}{\frameDefinitelyLocalOne}
\end{funFrame}

\begin{funFrame}[color=SeaGreen]{\frameCamlRaiseExnBegin}{\frameCamlRaiseExnEnd}
\funReturnAddr{\frameCamlRaiseExnRetaddr}
\end{funFrame}
}{}


\ifthenelse{\step>5 \and \step<10}{
\begin{funFrame}[name=reraise,color=JungleGreen]{\frameCamlReraiseExnBeginvB}{\frameCamlReraiseExnEndvB}
\funReturnAddr{\frameCamlReraiseExnRetaddrvB}
\end{funFrame}
}{}

\ifthenelse{\step>10 \and \step<13}{
\begin{funFrame}[name=reraise,color=JungleGreen]{\frameCamlReraiseExnBeginvC}{\frameCamlReraiseExnEndvC}
\funReturnAddr{\frameCamlReraiseExnRetaddrvC}
\end{funFrame}
}{}


\coordinate (bottom) at ($(stackBL) + (0,-0.25)$);


\ifthenelse{\step=1 \or \step=2}{
\domainStateBacktrace[pos=(bottom),yFactor=0.03]{\exnHandlerAfterExnATrap}{}{0}
\draw [->,>=latex] (exnHandlerRight.east) to [out=0,in=0] (exnAtrap.east);
}{}

\ifthenelse{\step>4 \and \step<7}{
\domainStateBacktrace[pos=(bottom),yFactor=0.03]{\exnHandlerAfterExnPopTrapvA}{\nextFrameDescrPCvB}{\stashBacktraceBacktracePosvA}
}{}

\ifthenelse{\step>9 \and \step<12}{
\domainStateBacktrace[pos=(bottom),yFactor=0.03]{\exnHandlerAfterExnPopTrapvB}{\nextFrameDescrPCvE}{\stashBacktraceBacktracePosvB}
}{}

\ifthenelse{\step=5}{
\draw [->,>=latex] (exnHandlerRight.east) to [out=0,in=0] (exnBtrap.east);
}{}

\ifthenelse{\step=10}{
\draw [->,>=latex] (exnHandlerRight.east) to [out=0,in=0] (exnCtrap.east);
}{}

\ifthenelse{\step=2}{
\showRegister{sp}{\stashBacktraceArgSPvA}
}{}

\ifthenelse{\step=3}{
\domainStateBacktrace[pos=(bottom),yFactor=0.03]{\exnHandlerAfterExnATrap}{\nextFrameDescrPCvA}{1}
\showRegister{sp}{\nextFrameDescrSPvA}
}{}

\ifthenelse{\step=4}{
\domainStateBacktrace[pos=(bottom),yFactor=0.03]{\exnHandlerAfterExnATrap}{\nextFrameDescrPCvB}{2}
\showRegister[opacity=0.25]{sp}{\nextFrameDescrSPvA}
\showRegister[offset=-0.8]{sp}{\nextFrameDescrSPvB}
}{}

% Drawing order matter for z-index trapsp/sp
\ifthenelse{\step>1 \and \step<5}{
\showRegister{trapsp}{\stashBacktraceArgTrapSPvA}
\showRegister{pc}{\frameCamlRaiseExnEnd}
}{}

\ifthenelse{\step>5 \and \step<10}{
\draw [->,>=latex] (exnHandlerRight.east) to [out=0,in=0] (exnBtrap.east);
}{}

\ifthenelse{\step=7}{
\domainStateBacktrace[pos=(bottom),yFactor=0.03]{\exnHandlerAfterExnPopTrapvA}{\nextFrameDescrPCvC}{3}
\showRegister[opacity=0.25]{sp}{\stashBacktraceArgSPvB}
\showRegister{sp}{\nextFrameDescrSPvC}
}{}

\ifthenelse{\step=8}{
\domainStateBacktrace[pos=(bottom),yFactor=0.03]{\exnHandlerAfterExnPopTrapvA}{\nextFrameDescrPCvD}{4}
\showRegister[opacity=0.25]{sp}{\nextFrameDescrSPvC}
\showRegister{sp}{\nextFrameDescrSPvD}
}{}

\ifthenelse{\step=9}{
\domainStateBacktrace[pos=(bottom),yFactor=0.03]{\exnHandlerAfterExnPopTrapvA}{\nextFrameDescrPCvE}{5}
\showRegister[opacity=0.25]{sp}{\nextFrameDescrSPvD}
\showRegister[offset=-0.8]{sp}{\nextFrameDescrSPvE}
}{}

% Drawing order matter for z-index trapsp/sp
\ifthenelse{\step>6 \and \step<10}{
\showRegister{pc}{\frameCamlReraiseExnEndvB}
\showRegister{trapsp}{\stashBacktraceArgTrapSPvB}
}{}


\ifthenelse{\step>10 \and \step<13}{
\draw [->,>=latex] (exnHandlerRight.east) to [out=0,in=0] (exnCtrap.east);
}{}

\ifthenelse{\step=12}{
\domainStateBacktrace[pos=(bottom),yFactor=0.03]{\exnHandlerAfterExnPopTrapvB}{\nextFrameDescrPCvF}{\stashBacktraceBacktracePosvC}
}{}

% Drawing order matter for z-index trapsp/sp
\ifthenelse{\step>10 \and \step<13}{
\showRegister{pc}{\frameCamlReraiseExnEndvC}
\showRegister{trapsp}{\stashBacktraceArgTrapSPvC}
}{}

\end{stackDiagram}
\end{tikzpicture}
}
      \onslide*<2->{\providecommand\step{6}\begin{tikzpicture}[font=\sffamily\tiny]
\input{../src/backtrace/gdb.out}

\newcommand\trapExnAEnd{\xinteval{\trapExnABegin - 2 * \wordSize}}
\newcommand\trapExnBEnd{\xinteval{\trapExnBBegin - 2 * \wordSize}}
\newcommand\trapExnCEnd{\xinteval{\trapExnCBegin - 2 * \wordSize}}

\begin{stackDiagram}[yFactor=0.35,showAddress=true]{\frameMainBegin}{\frameDefinitelyEnd}
\callStack

\begin{funFrame}[name=main,color=GreenYellow]{\frameMainBegin}{\frameMainEnd}
\funReturnAddr{\frameMainRetaddr}
\frameLocal{1}{\frameMainLocalOne}
\end{funFrame}

\ifthenelse{\step<13}{
\trapFrame[name=ExnC,handler=\trapExnCHandler,next=\trapExnCNext,color=WildStrawberry]{\trapExnCBegin}
\coordAtRight{exnCtrap}{\exnHandlerAfterExnCTrap}
}{}


\ifthenelse{\step<10}{
\trapFrame[name=ExnB,handler=\trapExnBHandler,next=\trapExnBNext,color=OrangeRed]{\trapExnBBegin}
\coordAtRight{exnBtrap}{\exnHandlerAfterExnBTrap}
\draw [->,>=latex] (exnBtrap.east) to [out=0,in=0] (exnCtrap.east);

\begin{funFrame}[name=maybe,color=YellowGreen]{\frameMaybeBegin}{\frameMaybeEnd}
\funReturnAddr{\frameMaybeRetaddr}
\frameLocal{1}{\frameMaybeLocalOne}
\end{funFrame}

\begin{funFrame}[name=probably,color=Green]{\frameProbablyBegin}{\frameProbablyEnd}
\funReturnAddr{\frameProbablyRetaddr}
\frameLocal{1}{\frameProbablyLocalOne}
\end{funFrame}
}{}


\ifthenelse{\step<5}{
\trapFrame[name=ExnA,handler=\trapExnAHandler,next=\trapExnANext,color=Red]{\trapExnABegin}
\coordAtRight{exnAtrap}{\exnHandlerAfterExnATrap}
\draw [->,>=latex] (exnAtrap.east) to [out=0,in=0] (exnBtrap.east);

\begin{funFrame}[name=definitely,color=JungleGreen]{\frameDefinitelyBegin}{\frameDefinitelyEnd}
\funReturnAddr{\frameDefinitelyRetaddr}
\frameLocal{1}{\frameDefinitelyLocalOne}
\end{funFrame}

\begin{funFrame}[color=SeaGreen]{\frameCamlRaiseExnBegin}{\frameCamlRaiseExnEnd}
\funReturnAddr{\frameCamlRaiseExnRetaddr}
\end{funFrame}
}{}


\ifthenelse{\step>5 \and \step<10}{
\begin{funFrame}[name=reraise,color=JungleGreen]{\frameCamlReraiseExnBeginvB}{\frameCamlReraiseExnEndvB}
\funReturnAddr{\frameCamlReraiseExnRetaddrvB}
\end{funFrame}
}{}

\ifthenelse{\step>10 \and \step<13}{
\begin{funFrame}[name=reraise,color=JungleGreen]{\frameCamlReraiseExnBeginvC}{\frameCamlReraiseExnEndvC}
\funReturnAddr{\frameCamlReraiseExnRetaddrvC}
\end{funFrame}
}{}


\coordinate (bottom) at ($(stackBL) + (0,-0.25)$);


\ifthenelse{\step=1 \or \step=2}{
\domainStateBacktrace[pos=(bottom),yFactor=0.03]{\exnHandlerAfterExnATrap}{}{0}
\draw [->,>=latex] (exnHandlerRight.east) to [out=0,in=0] (exnAtrap.east);
}{}

\ifthenelse{\step>4 \and \step<7}{
\domainStateBacktrace[pos=(bottom),yFactor=0.03]{\exnHandlerAfterExnPopTrapvA}{\nextFrameDescrPCvB}{\stashBacktraceBacktracePosvA}
}{}

\ifthenelse{\step>9 \and \step<12}{
\domainStateBacktrace[pos=(bottom),yFactor=0.03]{\exnHandlerAfterExnPopTrapvB}{\nextFrameDescrPCvE}{\stashBacktraceBacktracePosvB}
}{}

\ifthenelse{\step=5}{
\draw [->,>=latex] (exnHandlerRight.east) to [out=0,in=0] (exnBtrap.east);
}{}

\ifthenelse{\step=10}{
\draw [->,>=latex] (exnHandlerRight.east) to [out=0,in=0] (exnCtrap.east);
}{}

\ifthenelse{\step=2}{
\showRegister{sp}{\stashBacktraceArgSPvA}
}{}

\ifthenelse{\step=3}{
\domainStateBacktrace[pos=(bottom),yFactor=0.03]{\exnHandlerAfterExnATrap}{\nextFrameDescrPCvA}{1}
\showRegister{sp}{\nextFrameDescrSPvA}
}{}

\ifthenelse{\step=4}{
\domainStateBacktrace[pos=(bottom),yFactor=0.03]{\exnHandlerAfterExnATrap}{\nextFrameDescrPCvB}{2}
\showRegister[opacity=0.25]{sp}{\nextFrameDescrSPvA}
\showRegister[offset=-0.8]{sp}{\nextFrameDescrSPvB}
}{}

% Drawing order matter for z-index trapsp/sp
\ifthenelse{\step>1 \and \step<5}{
\showRegister{trapsp}{\stashBacktraceArgTrapSPvA}
\showRegister{pc}{\frameCamlRaiseExnEnd}
}{}

\ifthenelse{\step>5 \and \step<10}{
\draw [->,>=latex] (exnHandlerRight.east) to [out=0,in=0] (exnBtrap.east);
}{}

\ifthenelse{\step=7}{
\domainStateBacktrace[pos=(bottom),yFactor=0.03]{\exnHandlerAfterExnPopTrapvA}{\nextFrameDescrPCvC}{3}
\showRegister[opacity=0.25]{sp}{\stashBacktraceArgSPvB}
\showRegister{sp}{\nextFrameDescrSPvC}
}{}

\ifthenelse{\step=8}{
\domainStateBacktrace[pos=(bottom),yFactor=0.03]{\exnHandlerAfterExnPopTrapvA}{\nextFrameDescrPCvD}{4}
\showRegister[opacity=0.25]{sp}{\nextFrameDescrSPvC}
\showRegister{sp}{\nextFrameDescrSPvD}
}{}

\ifthenelse{\step=9}{
\domainStateBacktrace[pos=(bottom),yFactor=0.03]{\exnHandlerAfterExnPopTrapvA}{\nextFrameDescrPCvE}{5}
\showRegister[opacity=0.25]{sp}{\nextFrameDescrSPvD}
\showRegister[offset=-0.8]{sp}{\nextFrameDescrSPvE}
}{}

% Drawing order matter for z-index trapsp/sp
\ifthenelse{\step>6 \and \step<10}{
\showRegister{pc}{\frameCamlReraiseExnEndvB}
\showRegister{trapsp}{\stashBacktraceArgTrapSPvB}
}{}


\ifthenelse{\step>10 \and \step<13}{
\draw [->,>=latex] (exnHandlerRight.east) to [out=0,in=0] (exnCtrap.east);
}{}

\ifthenelse{\step=12}{
\domainStateBacktrace[pos=(bottom),yFactor=0.03]{\exnHandlerAfterExnPopTrapvB}{\nextFrameDescrPCvF}{\stashBacktraceBacktracePosvC}
}{}

% Drawing order matter for z-index trapsp/sp
\ifthenelse{\step>10 \and \step<13}{
\showRegister{pc}{\frameCamlReraiseExnEndvC}
\showRegister{trapsp}{\stashBacktraceArgTrapSPvC}
}{}

\end{stackDiagram}
\end{tikzpicture}
}
    \end{column}
  \end{columns}
\end{frame}


%\onslide*<2>{\providecommand\step{2}\begin{tikzpicture}[font=\sffamily\tiny]
\input{../src/backtrace/gdb.out}

\newcommand\trapExnAEnd{\xinteval{\trapExnABegin - 2 * \wordSize}}
\newcommand\trapExnBEnd{\xinteval{\trapExnBBegin - 2 * \wordSize}}
\newcommand\trapExnCEnd{\xinteval{\trapExnCBegin - 2 * \wordSize}}

\begin{stackDiagram}[yFactor=0.35,showAddress=true]{\frameMainBegin}{\frameDefinitelyEnd}
\callStack

\begin{funFrame}[name=main,color=GreenYellow]{\frameMainBegin}{\frameMainEnd}
\funReturnAddr{\frameMainRetaddr}
\frameLocal{1}{\frameMainLocalOne}
\end{funFrame}

\ifthenelse{\step<13}{
\trapFrame[name=ExnC,handler=\trapExnCHandler,next=\trapExnCNext,color=WildStrawberry]{\trapExnCBegin}
\coordAtRight{exnCtrap}{\exnHandlerAfterExnCTrap}
}{}


\ifthenelse{\step<10}{
\trapFrame[name=ExnB,handler=\trapExnBHandler,next=\trapExnBNext,color=OrangeRed]{\trapExnBBegin}
\coordAtRight{exnBtrap}{\exnHandlerAfterExnBTrap}
\draw [->,>=latex] (exnBtrap.east) to [out=0,in=0] (exnCtrap.east);

\begin{funFrame}[name=maybe,color=YellowGreen]{\frameMaybeBegin}{\frameMaybeEnd}
\funReturnAddr{\frameMaybeRetaddr}
\frameLocal{1}{\frameMaybeLocalOne}
\end{funFrame}

\begin{funFrame}[name=probably,color=Green]{\frameProbablyBegin}{\frameProbablyEnd}
\funReturnAddr{\frameProbablyRetaddr}
\frameLocal{1}{\frameProbablyLocalOne}
\end{funFrame}
}{}


\ifthenelse{\step<5}{
\trapFrame[name=ExnA,handler=\trapExnAHandler,next=\trapExnANext,color=Red]{\trapExnABegin}
\coordAtRight{exnAtrap}{\exnHandlerAfterExnATrap}
\draw [->,>=latex] (exnAtrap.east) to [out=0,in=0] (exnBtrap.east);

\begin{funFrame}[name=definitely,color=JungleGreen]{\frameDefinitelyBegin}{\frameDefinitelyEnd}
\funReturnAddr{\frameDefinitelyRetaddr}
\frameLocal{1}{\frameDefinitelyLocalOne}
\end{funFrame}

\begin{funFrame}[color=SeaGreen]{\frameCamlRaiseExnBegin}{\frameCamlRaiseExnEnd}
\funReturnAddr{\frameCamlRaiseExnRetaddr}
\end{funFrame}
}{}


\ifthenelse{\step>5 \and \step<10}{
\begin{funFrame}[name=reraise,color=JungleGreen]{\frameCamlReraiseExnBeginvB}{\frameCamlReraiseExnEndvB}
\funReturnAddr{\frameCamlReraiseExnRetaddrvB}
\end{funFrame}
}{}

\ifthenelse{\step>10 \and \step<13}{
\begin{funFrame}[name=reraise,color=JungleGreen]{\frameCamlReraiseExnBeginvC}{\frameCamlReraiseExnEndvC}
\funReturnAddr{\frameCamlReraiseExnRetaddrvC}
\end{funFrame}
}{}


\coordinate (bottom) at ($(stackBL) + (0,-0.25)$);


\ifthenelse{\step=1 \or \step=2}{
\domainStateBacktrace[pos=(bottom),yFactor=0.03]{\exnHandlerAfterExnATrap}{}{0}
\draw [->,>=latex] (exnHandlerRight.east) to [out=0,in=0] (exnAtrap.east);
}{}

\ifthenelse{\step>4 \and \step<7}{
\domainStateBacktrace[pos=(bottom),yFactor=0.03]{\exnHandlerAfterExnPopTrapvA}{\nextFrameDescrPCvB}{\stashBacktraceBacktracePosvA}
}{}

\ifthenelse{\step>9 \and \step<12}{
\domainStateBacktrace[pos=(bottom),yFactor=0.03]{\exnHandlerAfterExnPopTrapvB}{\nextFrameDescrPCvE}{\stashBacktraceBacktracePosvB}
}{}

\ifthenelse{\step=5}{
\draw [->,>=latex] (exnHandlerRight.east) to [out=0,in=0] (exnBtrap.east);
}{}

\ifthenelse{\step=10}{
\draw [->,>=latex] (exnHandlerRight.east) to [out=0,in=0] (exnCtrap.east);
}{}

\ifthenelse{\step=2}{
\showRegister{sp}{\stashBacktraceArgSPvA}
}{}

\ifthenelse{\step=3}{
\domainStateBacktrace[pos=(bottom),yFactor=0.03]{\exnHandlerAfterExnATrap}{\nextFrameDescrPCvA}{1}
\showRegister{sp}{\nextFrameDescrSPvA}
}{}

\ifthenelse{\step=4}{
\domainStateBacktrace[pos=(bottom),yFactor=0.03]{\exnHandlerAfterExnATrap}{\nextFrameDescrPCvB}{2}
\showRegister[opacity=0.25]{sp}{\nextFrameDescrSPvA}
\showRegister[offset=-0.8]{sp}{\nextFrameDescrSPvB}
}{}

% Drawing order matter for z-index trapsp/sp
\ifthenelse{\step>1 \and \step<5}{
\showRegister{trapsp}{\stashBacktraceArgTrapSPvA}
\showRegister{pc}{\frameCamlRaiseExnEnd}
}{}

\ifthenelse{\step>5 \and \step<10}{
\draw [->,>=latex] (exnHandlerRight.east) to [out=0,in=0] (exnBtrap.east);
}{}

\ifthenelse{\step=7}{
\domainStateBacktrace[pos=(bottom),yFactor=0.03]{\exnHandlerAfterExnPopTrapvA}{\nextFrameDescrPCvC}{3}
\showRegister[opacity=0.25]{sp}{\stashBacktraceArgSPvB}
\showRegister{sp}{\nextFrameDescrSPvC}
}{}

\ifthenelse{\step=8}{
\domainStateBacktrace[pos=(bottom),yFactor=0.03]{\exnHandlerAfterExnPopTrapvA}{\nextFrameDescrPCvD}{4}
\showRegister[opacity=0.25]{sp}{\nextFrameDescrSPvC}
\showRegister{sp}{\nextFrameDescrSPvD}
}{}

\ifthenelse{\step=9}{
\domainStateBacktrace[pos=(bottom),yFactor=0.03]{\exnHandlerAfterExnPopTrapvA}{\nextFrameDescrPCvE}{5}
\showRegister[opacity=0.25]{sp}{\nextFrameDescrSPvD}
\showRegister[offset=-0.8]{sp}{\nextFrameDescrSPvE}
}{}

% Drawing order matter for z-index trapsp/sp
\ifthenelse{\step>6 \and \step<10}{
\showRegister{pc}{\frameCamlReraiseExnEndvB}
\showRegister{trapsp}{\stashBacktraceArgTrapSPvB}
}{}


\ifthenelse{\step>10 \and \step<13}{
\draw [->,>=latex] (exnHandlerRight.east) to [out=0,in=0] (exnCtrap.east);
}{}

\ifthenelse{\step=12}{
\domainStateBacktrace[pos=(bottom),yFactor=0.03]{\exnHandlerAfterExnPopTrapvB}{\nextFrameDescrPCvF}{\stashBacktraceBacktracePosvC}
}{}

% Drawing order matter for z-index trapsp/sp
\ifthenelse{\step>10 \and \step<13}{
\showRegister{pc}{\frameCamlReraiseExnEndvC}
\showRegister{trapsp}{\stashBacktraceArgTrapSPvC}
}{}

\end{stackDiagram}
\end{tikzpicture}
}
%\onslide*<3>{\providecommand\step{3}\begin{tikzpicture}[font=\sffamily\tiny]
\input{../src/backtrace/gdb.out}

\newcommand\trapExnAEnd{\xinteval{\trapExnABegin - 2 * \wordSize}}
\newcommand\trapExnBEnd{\xinteval{\trapExnBBegin - 2 * \wordSize}}
\newcommand\trapExnCEnd{\xinteval{\trapExnCBegin - 2 * \wordSize}}

\begin{stackDiagram}[yFactor=0.35,showAddress=true]{\frameMainBegin}{\frameDefinitelyEnd}
\callStack

\begin{funFrame}[name=main,color=GreenYellow]{\frameMainBegin}{\frameMainEnd}
\funReturnAddr{\frameMainRetaddr}
\frameLocal{1}{\frameMainLocalOne}
\end{funFrame}

\ifthenelse{\step<13}{
\trapFrame[name=ExnC,handler=\trapExnCHandler,next=\trapExnCNext,color=WildStrawberry]{\trapExnCBegin}
\coordAtRight{exnCtrap}{\exnHandlerAfterExnCTrap}
}{}


\ifthenelse{\step<10}{
\trapFrame[name=ExnB,handler=\trapExnBHandler,next=\trapExnBNext,color=OrangeRed]{\trapExnBBegin}
\coordAtRight{exnBtrap}{\exnHandlerAfterExnBTrap}
\draw [->,>=latex] (exnBtrap.east) to [out=0,in=0] (exnCtrap.east);

\begin{funFrame}[name=maybe,color=YellowGreen]{\frameMaybeBegin}{\frameMaybeEnd}
\funReturnAddr{\frameMaybeRetaddr}
\frameLocal{1}{\frameMaybeLocalOne}
\end{funFrame}

\begin{funFrame}[name=probably,color=Green]{\frameProbablyBegin}{\frameProbablyEnd}
\funReturnAddr{\frameProbablyRetaddr}
\frameLocal{1}{\frameProbablyLocalOne}
\end{funFrame}
}{}


\ifthenelse{\step<5}{
\trapFrame[name=ExnA,handler=\trapExnAHandler,next=\trapExnANext,color=Red]{\trapExnABegin}
\coordAtRight{exnAtrap}{\exnHandlerAfterExnATrap}
\draw [->,>=latex] (exnAtrap.east) to [out=0,in=0] (exnBtrap.east);

\begin{funFrame}[name=definitely,color=JungleGreen]{\frameDefinitelyBegin}{\frameDefinitelyEnd}
\funReturnAddr{\frameDefinitelyRetaddr}
\frameLocal{1}{\frameDefinitelyLocalOne}
\end{funFrame}

\begin{funFrame}[color=SeaGreen]{\frameCamlRaiseExnBegin}{\frameCamlRaiseExnEnd}
\funReturnAddr{\frameCamlRaiseExnRetaddr}
\end{funFrame}
}{}


\ifthenelse{\step>5 \and \step<10}{
\begin{funFrame}[name=reraise,color=JungleGreen]{\frameCamlReraiseExnBeginvB}{\frameCamlReraiseExnEndvB}
\funReturnAddr{\frameCamlReraiseExnRetaddrvB}
\end{funFrame}
}{}

\ifthenelse{\step>10 \and \step<13}{
\begin{funFrame}[name=reraise,color=JungleGreen]{\frameCamlReraiseExnBeginvC}{\frameCamlReraiseExnEndvC}
\funReturnAddr{\frameCamlReraiseExnRetaddrvC}
\end{funFrame}
}{}


\coordinate (bottom) at ($(stackBL) + (0,-0.25)$);


\ifthenelse{\step=1 \or \step=2}{
\domainStateBacktrace[pos=(bottom),yFactor=0.03]{\exnHandlerAfterExnATrap}{}{0}
\draw [->,>=latex] (exnHandlerRight.east) to [out=0,in=0] (exnAtrap.east);
}{}

\ifthenelse{\step>4 \and \step<7}{
\domainStateBacktrace[pos=(bottom),yFactor=0.03]{\exnHandlerAfterExnPopTrapvA}{\nextFrameDescrPCvB}{\stashBacktraceBacktracePosvA}
}{}

\ifthenelse{\step>9 \and \step<12}{
\domainStateBacktrace[pos=(bottom),yFactor=0.03]{\exnHandlerAfterExnPopTrapvB}{\nextFrameDescrPCvE}{\stashBacktraceBacktracePosvB}
}{}

\ifthenelse{\step=5}{
\draw [->,>=latex] (exnHandlerRight.east) to [out=0,in=0] (exnBtrap.east);
}{}

\ifthenelse{\step=10}{
\draw [->,>=latex] (exnHandlerRight.east) to [out=0,in=0] (exnCtrap.east);
}{}

\ifthenelse{\step=2}{
\showRegister{sp}{\stashBacktraceArgSPvA}
}{}

\ifthenelse{\step=3}{
\domainStateBacktrace[pos=(bottom),yFactor=0.03]{\exnHandlerAfterExnATrap}{\nextFrameDescrPCvA}{1}
\showRegister{sp}{\nextFrameDescrSPvA}
}{}

\ifthenelse{\step=4}{
\domainStateBacktrace[pos=(bottom),yFactor=0.03]{\exnHandlerAfterExnATrap}{\nextFrameDescrPCvB}{2}
\showRegister[opacity=0.25]{sp}{\nextFrameDescrSPvA}
\showRegister[offset=-0.8]{sp}{\nextFrameDescrSPvB}
}{}

% Drawing order matter for z-index trapsp/sp
\ifthenelse{\step>1 \and \step<5}{
\showRegister{trapsp}{\stashBacktraceArgTrapSPvA}
\showRegister{pc}{\frameCamlRaiseExnEnd}
}{}

\ifthenelse{\step>5 \and \step<10}{
\draw [->,>=latex] (exnHandlerRight.east) to [out=0,in=0] (exnBtrap.east);
}{}

\ifthenelse{\step=7}{
\domainStateBacktrace[pos=(bottom),yFactor=0.03]{\exnHandlerAfterExnPopTrapvA}{\nextFrameDescrPCvC}{3}
\showRegister[opacity=0.25]{sp}{\stashBacktraceArgSPvB}
\showRegister{sp}{\nextFrameDescrSPvC}
}{}

\ifthenelse{\step=8}{
\domainStateBacktrace[pos=(bottom),yFactor=0.03]{\exnHandlerAfterExnPopTrapvA}{\nextFrameDescrPCvD}{4}
\showRegister[opacity=0.25]{sp}{\nextFrameDescrSPvC}
\showRegister{sp}{\nextFrameDescrSPvD}
}{}

\ifthenelse{\step=9}{
\domainStateBacktrace[pos=(bottom),yFactor=0.03]{\exnHandlerAfterExnPopTrapvA}{\nextFrameDescrPCvE}{5}
\showRegister[opacity=0.25]{sp}{\nextFrameDescrSPvD}
\showRegister[offset=-0.8]{sp}{\nextFrameDescrSPvE}
}{}

% Drawing order matter for z-index trapsp/sp
\ifthenelse{\step>6 \and \step<10}{
\showRegister{pc}{\frameCamlReraiseExnEndvB}
\showRegister{trapsp}{\stashBacktraceArgTrapSPvB}
}{}


\ifthenelse{\step>10 \and \step<13}{
\draw [->,>=latex] (exnHandlerRight.east) to [out=0,in=0] (exnCtrap.east);
}{}

\ifthenelse{\step=12}{
\domainStateBacktrace[pos=(bottom),yFactor=0.03]{\exnHandlerAfterExnPopTrapvB}{\nextFrameDescrPCvF}{\stashBacktraceBacktracePosvC}
}{}

% Drawing order matter for z-index trapsp/sp
\ifthenelse{\step>10 \and \step<13}{
\showRegister{pc}{\frameCamlReraiseExnEndvC}
\showRegister{trapsp}{\stashBacktraceArgTrapSPvC}
}{}

\end{stackDiagram}
\end{tikzpicture}
}
%\onslide*<4>{\providecommand\step{4}\begin{tikzpicture}[font=\sffamily\tiny]
\input{../src/backtrace/gdb.out}

\newcommand\trapExnAEnd{\xinteval{\trapExnABegin - 2 * \wordSize}}
\newcommand\trapExnBEnd{\xinteval{\trapExnBBegin - 2 * \wordSize}}
\newcommand\trapExnCEnd{\xinteval{\trapExnCBegin - 2 * \wordSize}}

\begin{stackDiagram}[yFactor=0.35,showAddress=true]{\frameMainBegin}{\frameDefinitelyEnd}
\callStack

\begin{funFrame}[name=main,color=GreenYellow]{\frameMainBegin}{\frameMainEnd}
\funReturnAddr{\frameMainRetaddr}
\frameLocal{1}{\frameMainLocalOne}
\end{funFrame}

\ifthenelse{\step<13}{
\trapFrame[name=ExnC,handler=\trapExnCHandler,next=\trapExnCNext,color=WildStrawberry]{\trapExnCBegin}
\coordAtRight{exnCtrap}{\exnHandlerAfterExnCTrap}
}{}


\ifthenelse{\step<10}{
\trapFrame[name=ExnB,handler=\trapExnBHandler,next=\trapExnBNext,color=OrangeRed]{\trapExnBBegin}
\coordAtRight{exnBtrap}{\exnHandlerAfterExnBTrap}
\draw [->,>=latex] (exnBtrap.east) to [out=0,in=0] (exnCtrap.east);

\begin{funFrame}[name=maybe,color=YellowGreen]{\frameMaybeBegin}{\frameMaybeEnd}
\funReturnAddr{\frameMaybeRetaddr}
\frameLocal{1}{\frameMaybeLocalOne}
\end{funFrame}

\begin{funFrame}[name=probably,color=Green]{\frameProbablyBegin}{\frameProbablyEnd}
\funReturnAddr{\frameProbablyRetaddr}
\frameLocal{1}{\frameProbablyLocalOne}
\end{funFrame}
}{}


\ifthenelse{\step<5}{
\trapFrame[name=ExnA,handler=\trapExnAHandler,next=\trapExnANext,color=Red]{\trapExnABegin}
\coordAtRight{exnAtrap}{\exnHandlerAfterExnATrap}
\draw [->,>=latex] (exnAtrap.east) to [out=0,in=0] (exnBtrap.east);

\begin{funFrame}[name=definitely,color=JungleGreen]{\frameDefinitelyBegin}{\frameDefinitelyEnd}
\funReturnAddr{\frameDefinitelyRetaddr}
\frameLocal{1}{\frameDefinitelyLocalOne}
\end{funFrame}

\begin{funFrame}[color=SeaGreen]{\frameCamlRaiseExnBegin}{\frameCamlRaiseExnEnd}
\funReturnAddr{\frameCamlRaiseExnRetaddr}
\end{funFrame}
}{}


\ifthenelse{\step>5 \and \step<10}{
\begin{funFrame}[name=reraise,color=JungleGreen]{\frameCamlReraiseExnBeginvB}{\frameCamlReraiseExnEndvB}
\funReturnAddr{\frameCamlReraiseExnRetaddrvB}
\end{funFrame}
}{}

\ifthenelse{\step>10 \and \step<13}{
\begin{funFrame}[name=reraise,color=JungleGreen]{\frameCamlReraiseExnBeginvC}{\frameCamlReraiseExnEndvC}
\funReturnAddr{\frameCamlReraiseExnRetaddrvC}
\end{funFrame}
}{}


\coordinate (bottom) at ($(stackBL) + (0,-0.25)$);


\ifthenelse{\step=1 \or \step=2}{
\domainStateBacktrace[pos=(bottom),yFactor=0.03]{\exnHandlerAfterExnATrap}{}{0}
\draw [->,>=latex] (exnHandlerRight.east) to [out=0,in=0] (exnAtrap.east);
}{}

\ifthenelse{\step>4 \and \step<7}{
\domainStateBacktrace[pos=(bottom),yFactor=0.03]{\exnHandlerAfterExnPopTrapvA}{\nextFrameDescrPCvB}{\stashBacktraceBacktracePosvA}
}{}

\ifthenelse{\step>9 \and \step<12}{
\domainStateBacktrace[pos=(bottom),yFactor=0.03]{\exnHandlerAfterExnPopTrapvB}{\nextFrameDescrPCvE}{\stashBacktraceBacktracePosvB}
}{}

\ifthenelse{\step=5}{
\draw [->,>=latex] (exnHandlerRight.east) to [out=0,in=0] (exnBtrap.east);
}{}

\ifthenelse{\step=10}{
\draw [->,>=latex] (exnHandlerRight.east) to [out=0,in=0] (exnCtrap.east);
}{}

\ifthenelse{\step=2}{
\showRegister{sp}{\stashBacktraceArgSPvA}
}{}

\ifthenelse{\step=3}{
\domainStateBacktrace[pos=(bottom),yFactor=0.03]{\exnHandlerAfterExnATrap}{\nextFrameDescrPCvA}{1}
\showRegister{sp}{\nextFrameDescrSPvA}
}{}

\ifthenelse{\step=4}{
\domainStateBacktrace[pos=(bottom),yFactor=0.03]{\exnHandlerAfterExnATrap}{\nextFrameDescrPCvB}{2}
\showRegister[opacity=0.25]{sp}{\nextFrameDescrSPvA}
\showRegister[offset=-0.8]{sp}{\nextFrameDescrSPvB}
}{}

% Drawing order matter for z-index trapsp/sp
\ifthenelse{\step>1 \and \step<5}{
\showRegister{trapsp}{\stashBacktraceArgTrapSPvA}
\showRegister{pc}{\frameCamlRaiseExnEnd}
}{}

\ifthenelse{\step>5 \and \step<10}{
\draw [->,>=latex] (exnHandlerRight.east) to [out=0,in=0] (exnBtrap.east);
}{}

\ifthenelse{\step=7}{
\domainStateBacktrace[pos=(bottom),yFactor=0.03]{\exnHandlerAfterExnPopTrapvA}{\nextFrameDescrPCvC}{3}
\showRegister[opacity=0.25]{sp}{\stashBacktraceArgSPvB}
\showRegister{sp}{\nextFrameDescrSPvC}
}{}

\ifthenelse{\step=8}{
\domainStateBacktrace[pos=(bottom),yFactor=0.03]{\exnHandlerAfterExnPopTrapvA}{\nextFrameDescrPCvD}{4}
\showRegister[opacity=0.25]{sp}{\nextFrameDescrSPvC}
\showRegister{sp}{\nextFrameDescrSPvD}
}{}

\ifthenelse{\step=9}{
\domainStateBacktrace[pos=(bottom),yFactor=0.03]{\exnHandlerAfterExnPopTrapvA}{\nextFrameDescrPCvE}{5}
\showRegister[opacity=0.25]{sp}{\nextFrameDescrSPvD}
\showRegister[offset=-0.8]{sp}{\nextFrameDescrSPvE}
}{}

% Drawing order matter for z-index trapsp/sp
\ifthenelse{\step>6 \and \step<10}{
\showRegister{pc}{\frameCamlReraiseExnEndvB}
\showRegister{trapsp}{\stashBacktraceArgTrapSPvB}
}{}


\ifthenelse{\step>10 \and \step<13}{
\draw [->,>=latex] (exnHandlerRight.east) to [out=0,in=0] (exnCtrap.east);
}{}

\ifthenelse{\step=12}{
\domainStateBacktrace[pos=(bottom),yFactor=0.03]{\exnHandlerAfterExnPopTrapvB}{\nextFrameDescrPCvF}{\stashBacktraceBacktracePosvC}
}{}

% Drawing order matter for z-index trapsp/sp
\ifthenelse{\step>10 \and \step<13}{
\showRegister{pc}{\frameCamlReraiseExnEndvC}
\showRegister{trapsp}{\stashBacktraceArgTrapSPvC}
}{}

\end{stackDiagram}
\end{tikzpicture}
}
%\onslide*<5>{\providecommand\step{5}\begin{tikzpicture}[font=\sffamily\tiny]
\input{../src/backtrace/gdb.out}

\newcommand\trapExnAEnd{\xinteval{\trapExnABegin - 2 * \wordSize}}
\newcommand\trapExnBEnd{\xinteval{\trapExnBBegin - 2 * \wordSize}}
\newcommand\trapExnCEnd{\xinteval{\trapExnCBegin - 2 * \wordSize}}

\begin{stackDiagram}[yFactor=0.35,showAddress=true]{\frameMainBegin}{\frameDefinitelyEnd}
\callStack

\begin{funFrame}[name=main,color=GreenYellow]{\frameMainBegin}{\frameMainEnd}
\funReturnAddr{\frameMainRetaddr}
\frameLocal{1}{\frameMainLocalOne}
\end{funFrame}

\ifthenelse{\step<13}{
\trapFrame[name=ExnC,handler=\trapExnCHandler,next=\trapExnCNext,color=WildStrawberry]{\trapExnCBegin}
\coordAtRight{exnCtrap}{\exnHandlerAfterExnCTrap}
}{}


\ifthenelse{\step<10}{
\trapFrame[name=ExnB,handler=\trapExnBHandler,next=\trapExnBNext,color=OrangeRed]{\trapExnBBegin}
\coordAtRight{exnBtrap}{\exnHandlerAfterExnBTrap}
\draw [->,>=latex] (exnBtrap.east) to [out=0,in=0] (exnCtrap.east);

\begin{funFrame}[name=maybe,color=YellowGreen]{\frameMaybeBegin}{\frameMaybeEnd}
\funReturnAddr{\frameMaybeRetaddr}
\frameLocal{1}{\frameMaybeLocalOne}
\end{funFrame}

\begin{funFrame}[name=probably,color=Green]{\frameProbablyBegin}{\frameProbablyEnd}
\funReturnAddr{\frameProbablyRetaddr}
\frameLocal{1}{\frameProbablyLocalOne}
\end{funFrame}
}{}


\ifthenelse{\step<5}{
\trapFrame[name=ExnA,handler=\trapExnAHandler,next=\trapExnANext,color=Red]{\trapExnABegin}
\coordAtRight{exnAtrap}{\exnHandlerAfterExnATrap}
\draw [->,>=latex] (exnAtrap.east) to [out=0,in=0] (exnBtrap.east);

\begin{funFrame}[name=definitely,color=JungleGreen]{\frameDefinitelyBegin}{\frameDefinitelyEnd}
\funReturnAddr{\frameDefinitelyRetaddr}
\frameLocal{1}{\frameDefinitelyLocalOne}
\end{funFrame}

\begin{funFrame}[color=SeaGreen]{\frameCamlRaiseExnBegin}{\frameCamlRaiseExnEnd}
\funReturnAddr{\frameCamlRaiseExnRetaddr}
\end{funFrame}
}{}


\ifthenelse{\step>5 \and \step<10}{
\begin{funFrame}[name=reraise,color=JungleGreen]{\frameCamlReraiseExnBeginvB}{\frameCamlReraiseExnEndvB}
\funReturnAddr{\frameCamlReraiseExnRetaddrvB}
\end{funFrame}
}{}

\ifthenelse{\step>10 \and \step<13}{
\begin{funFrame}[name=reraise,color=JungleGreen]{\frameCamlReraiseExnBeginvC}{\frameCamlReraiseExnEndvC}
\funReturnAddr{\frameCamlReraiseExnRetaddrvC}
\end{funFrame}
}{}


\coordinate (bottom) at ($(stackBL) + (0,-0.25)$);


\ifthenelse{\step=1 \or \step=2}{
\domainStateBacktrace[pos=(bottom),yFactor=0.03]{\exnHandlerAfterExnATrap}{}{0}
\draw [->,>=latex] (exnHandlerRight.east) to [out=0,in=0] (exnAtrap.east);
}{}

\ifthenelse{\step>4 \and \step<7}{
\domainStateBacktrace[pos=(bottom),yFactor=0.03]{\exnHandlerAfterExnPopTrapvA}{\nextFrameDescrPCvB}{\stashBacktraceBacktracePosvA}
}{}

\ifthenelse{\step>9 \and \step<12}{
\domainStateBacktrace[pos=(bottom),yFactor=0.03]{\exnHandlerAfterExnPopTrapvB}{\nextFrameDescrPCvE}{\stashBacktraceBacktracePosvB}
}{}

\ifthenelse{\step=5}{
\draw [->,>=latex] (exnHandlerRight.east) to [out=0,in=0] (exnBtrap.east);
}{}

\ifthenelse{\step=10}{
\draw [->,>=latex] (exnHandlerRight.east) to [out=0,in=0] (exnCtrap.east);
}{}

\ifthenelse{\step=2}{
\showRegister{sp}{\stashBacktraceArgSPvA}
}{}

\ifthenelse{\step=3}{
\domainStateBacktrace[pos=(bottom),yFactor=0.03]{\exnHandlerAfterExnATrap}{\nextFrameDescrPCvA}{1}
\showRegister{sp}{\nextFrameDescrSPvA}
}{}

\ifthenelse{\step=4}{
\domainStateBacktrace[pos=(bottom),yFactor=0.03]{\exnHandlerAfterExnATrap}{\nextFrameDescrPCvB}{2}
\showRegister[opacity=0.25]{sp}{\nextFrameDescrSPvA}
\showRegister[offset=-0.8]{sp}{\nextFrameDescrSPvB}
}{}

% Drawing order matter for z-index trapsp/sp
\ifthenelse{\step>1 \and \step<5}{
\showRegister{trapsp}{\stashBacktraceArgTrapSPvA}
\showRegister{pc}{\frameCamlRaiseExnEnd}
}{}

\ifthenelse{\step>5 \and \step<10}{
\draw [->,>=latex] (exnHandlerRight.east) to [out=0,in=0] (exnBtrap.east);
}{}

\ifthenelse{\step=7}{
\domainStateBacktrace[pos=(bottom),yFactor=0.03]{\exnHandlerAfterExnPopTrapvA}{\nextFrameDescrPCvC}{3}
\showRegister[opacity=0.25]{sp}{\stashBacktraceArgSPvB}
\showRegister{sp}{\nextFrameDescrSPvC}
}{}

\ifthenelse{\step=8}{
\domainStateBacktrace[pos=(bottom),yFactor=0.03]{\exnHandlerAfterExnPopTrapvA}{\nextFrameDescrPCvD}{4}
\showRegister[opacity=0.25]{sp}{\nextFrameDescrSPvC}
\showRegister{sp}{\nextFrameDescrSPvD}
}{}

\ifthenelse{\step=9}{
\domainStateBacktrace[pos=(bottom),yFactor=0.03]{\exnHandlerAfterExnPopTrapvA}{\nextFrameDescrPCvE}{5}
\showRegister[opacity=0.25]{sp}{\nextFrameDescrSPvD}
\showRegister[offset=-0.8]{sp}{\nextFrameDescrSPvE}
}{}

% Drawing order matter for z-index trapsp/sp
\ifthenelse{\step>6 \and \step<10}{
\showRegister{pc}{\frameCamlReraiseExnEndvB}
\showRegister{trapsp}{\stashBacktraceArgTrapSPvB}
}{}


\ifthenelse{\step>10 \and \step<13}{
\draw [->,>=latex] (exnHandlerRight.east) to [out=0,in=0] (exnCtrap.east);
}{}

\ifthenelse{\step=12}{
\domainStateBacktrace[pos=(bottom),yFactor=0.03]{\exnHandlerAfterExnPopTrapvB}{\nextFrameDescrPCvF}{\stashBacktraceBacktracePosvC}
}{}

% Drawing order matter for z-index trapsp/sp
\ifthenelse{\step>10 \and \step<13}{
\showRegister{pc}{\frameCamlReraiseExnEndvC}
\showRegister{trapsp}{\stashBacktraceArgTrapSPvC}
}{}

\end{stackDiagram}
\end{tikzpicture}
}
%\onslide*<6>{\providecommand\step{6}\begin{tikzpicture}[font=\sffamily\tiny]
\input{../src/backtrace/gdb.out}

\newcommand\trapExnAEnd{\xinteval{\trapExnABegin - 2 * \wordSize}}
\newcommand\trapExnBEnd{\xinteval{\trapExnBBegin - 2 * \wordSize}}
\newcommand\trapExnCEnd{\xinteval{\trapExnCBegin - 2 * \wordSize}}

\begin{stackDiagram}[yFactor=0.35,showAddress=true]{\frameMainBegin}{\frameDefinitelyEnd}
\callStack

\begin{funFrame}[name=main,color=GreenYellow]{\frameMainBegin}{\frameMainEnd}
\funReturnAddr{\frameMainRetaddr}
\frameLocal{1}{\frameMainLocalOne}
\end{funFrame}

\ifthenelse{\step<13}{
\trapFrame[name=ExnC,handler=\trapExnCHandler,next=\trapExnCNext,color=WildStrawberry]{\trapExnCBegin}
\coordAtRight{exnCtrap}{\exnHandlerAfterExnCTrap}
}{}


\ifthenelse{\step<10}{
\trapFrame[name=ExnB,handler=\trapExnBHandler,next=\trapExnBNext,color=OrangeRed]{\trapExnBBegin}
\coordAtRight{exnBtrap}{\exnHandlerAfterExnBTrap}
\draw [->,>=latex] (exnBtrap.east) to [out=0,in=0] (exnCtrap.east);

\begin{funFrame}[name=maybe,color=YellowGreen]{\frameMaybeBegin}{\frameMaybeEnd}
\funReturnAddr{\frameMaybeRetaddr}
\frameLocal{1}{\frameMaybeLocalOne}
\end{funFrame}

\begin{funFrame}[name=probably,color=Green]{\frameProbablyBegin}{\frameProbablyEnd}
\funReturnAddr{\frameProbablyRetaddr}
\frameLocal{1}{\frameProbablyLocalOne}
\end{funFrame}
}{}


\ifthenelse{\step<5}{
\trapFrame[name=ExnA,handler=\trapExnAHandler,next=\trapExnANext,color=Red]{\trapExnABegin}
\coordAtRight{exnAtrap}{\exnHandlerAfterExnATrap}
\draw [->,>=latex] (exnAtrap.east) to [out=0,in=0] (exnBtrap.east);

\begin{funFrame}[name=definitely,color=JungleGreen]{\frameDefinitelyBegin}{\frameDefinitelyEnd}
\funReturnAddr{\frameDefinitelyRetaddr}
\frameLocal{1}{\frameDefinitelyLocalOne}
\end{funFrame}

\begin{funFrame}[color=SeaGreen]{\frameCamlRaiseExnBegin}{\frameCamlRaiseExnEnd}
\funReturnAddr{\frameCamlRaiseExnRetaddr}
\end{funFrame}
}{}


\ifthenelse{\step>5 \and \step<10}{
\begin{funFrame}[name=reraise,color=JungleGreen]{\frameCamlReraiseExnBeginvB}{\frameCamlReraiseExnEndvB}
\funReturnAddr{\frameCamlReraiseExnRetaddrvB}
\end{funFrame}
}{}

\ifthenelse{\step>10 \and \step<13}{
\begin{funFrame}[name=reraise,color=JungleGreen]{\frameCamlReraiseExnBeginvC}{\frameCamlReraiseExnEndvC}
\funReturnAddr{\frameCamlReraiseExnRetaddrvC}
\end{funFrame}
}{}


\coordinate (bottom) at ($(stackBL) + (0,-0.25)$);


\ifthenelse{\step=1 \or \step=2}{
\domainStateBacktrace[pos=(bottom),yFactor=0.03]{\exnHandlerAfterExnATrap}{}{0}
\draw [->,>=latex] (exnHandlerRight.east) to [out=0,in=0] (exnAtrap.east);
}{}

\ifthenelse{\step>4 \and \step<7}{
\domainStateBacktrace[pos=(bottom),yFactor=0.03]{\exnHandlerAfterExnPopTrapvA}{\nextFrameDescrPCvB}{\stashBacktraceBacktracePosvA}
}{}

\ifthenelse{\step>9 \and \step<12}{
\domainStateBacktrace[pos=(bottom),yFactor=0.03]{\exnHandlerAfterExnPopTrapvB}{\nextFrameDescrPCvE}{\stashBacktraceBacktracePosvB}
}{}

\ifthenelse{\step=5}{
\draw [->,>=latex] (exnHandlerRight.east) to [out=0,in=0] (exnBtrap.east);
}{}

\ifthenelse{\step=10}{
\draw [->,>=latex] (exnHandlerRight.east) to [out=0,in=0] (exnCtrap.east);
}{}

\ifthenelse{\step=2}{
\showRegister{sp}{\stashBacktraceArgSPvA}
}{}

\ifthenelse{\step=3}{
\domainStateBacktrace[pos=(bottom),yFactor=0.03]{\exnHandlerAfterExnATrap}{\nextFrameDescrPCvA}{1}
\showRegister{sp}{\nextFrameDescrSPvA}
}{}

\ifthenelse{\step=4}{
\domainStateBacktrace[pos=(bottom),yFactor=0.03]{\exnHandlerAfterExnATrap}{\nextFrameDescrPCvB}{2}
\showRegister[opacity=0.25]{sp}{\nextFrameDescrSPvA}
\showRegister[offset=-0.8]{sp}{\nextFrameDescrSPvB}
}{}

% Drawing order matter for z-index trapsp/sp
\ifthenelse{\step>1 \and \step<5}{
\showRegister{trapsp}{\stashBacktraceArgTrapSPvA}
\showRegister{pc}{\frameCamlRaiseExnEnd}
}{}

\ifthenelse{\step>5 \and \step<10}{
\draw [->,>=latex] (exnHandlerRight.east) to [out=0,in=0] (exnBtrap.east);
}{}

\ifthenelse{\step=7}{
\domainStateBacktrace[pos=(bottom),yFactor=0.03]{\exnHandlerAfterExnPopTrapvA}{\nextFrameDescrPCvC}{3}
\showRegister[opacity=0.25]{sp}{\stashBacktraceArgSPvB}
\showRegister{sp}{\nextFrameDescrSPvC}
}{}

\ifthenelse{\step=8}{
\domainStateBacktrace[pos=(bottom),yFactor=0.03]{\exnHandlerAfterExnPopTrapvA}{\nextFrameDescrPCvD}{4}
\showRegister[opacity=0.25]{sp}{\nextFrameDescrSPvC}
\showRegister{sp}{\nextFrameDescrSPvD}
}{}

\ifthenelse{\step=9}{
\domainStateBacktrace[pos=(bottom),yFactor=0.03]{\exnHandlerAfterExnPopTrapvA}{\nextFrameDescrPCvE}{5}
\showRegister[opacity=0.25]{sp}{\nextFrameDescrSPvD}
\showRegister[offset=-0.8]{sp}{\nextFrameDescrSPvE}
}{}

% Drawing order matter for z-index trapsp/sp
\ifthenelse{\step>6 \and \step<10}{
\showRegister{pc}{\frameCamlReraiseExnEndvB}
\showRegister{trapsp}{\stashBacktraceArgTrapSPvB}
}{}


\ifthenelse{\step>10 \and \step<13}{
\draw [->,>=latex] (exnHandlerRight.east) to [out=0,in=0] (exnCtrap.east);
}{}

\ifthenelse{\step=12}{
\domainStateBacktrace[pos=(bottom),yFactor=0.03]{\exnHandlerAfterExnPopTrapvB}{\nextFrameDescrPCvF}{\stashBacktraceBacktracePosvC}
}{}

% Drawing order matter for z-index trapsp/sp
\ifthenelse{\step>10 \and \step<13}{
\showRegister{pc}{\frameCamlReraiseExnEndvC}
\showRegister{trapsp}{\stashBacktraceArgTrapSPvC}
}{}

\end{stackDiagram}
\end{tikzpicture}
}
%\onslide*<7>{\providecommand\step{7}\begin{tikzpicture}[font=\sffamily\tiny]
\input{../src/backtrace/gdb.out}

\newcommand\trapExnAEnd{\xinteval{\trapExnABegin - 2 * \wordSize}}
\newcommand\trapExnBEnd{\xinteval{\trapExnBBegin - 2 * \wordSize}}
\newcommand\trapExnCEnd{\xinteval{\trapExnCBegin - 2 * \wordSize}}

\begin{stackDiagram}[yFactor=0.35,showAddress=true]{\frameMainBegin}{\frameDefinitelyEnd}
\callStack

\begin{funFrame}[name=main,color=GreenYellow]{\frameMainBegin}{\frameMainEnd}
\funReturnAddr{\frameMainRetaddr}
\frameLocal{1}{\frameMainLocalOne}
\end{funFrame}

\ifthenelse{\step<13}{
\trapFrame[name=ExnC,handler=\trapExnCHandler,next=\trapExnCNext,color=WildStrawberry]{\trapExnCBegin}
\coordAtRight{exnCtrap}{\exnHandlerAfterExnCTrap}
}{}


\ifthenelse{\step<10}{
\trapFrame[name=ExnB,handler=\trapExnBHandler,next=\trapExnBNext,color=OrangeRed]{\trapExnBBegin}
\coordAtRight{exnBtrap}{\exnHandlerAfterExnBTrap}
\draw [->,>=latex] (exnBtrap.east) to [out=0,in=0] (exnCtrap.east);

\begin{funFrame}[name=maybe,color=YellowGreen]{\frameMaybeBegin}{\frameMaybeEnd}
\funReturnAddr{\frameMaybeRetaddr}
\frameLocal{1}{\frameMaybeLocalOne}
\end{funFrame}

\begin{funFrame}[name=probably,color=Green]{\frameProbablyBegin}{\frameProbablyEnd}
\funReturnAddr{\frameProbablyRetaddr}
\frameLocal{1}{\frameProbablyLocalOne}
\end{funFrame}
}{}


\ifthenelse{\step<5}{
\trapFrame[name=ExnA,handler=\trapExnAHandler,next=\trapExnANext,color=Red]{\trapExnABegin}
\coordAtRight{exnAtrap}{\exnHandlerAfterExnATrap}
\draw [->,>=latex] (exnAtrap.east) to [out=0,in=0] (exnBtrap.east);

\begin{funFrame}[name=definitely,color=JungleGreen]{\frameDefinitelyBegin}{\frameDefinitelyEnd}
\funReturnAddr{\frameDefinitelyRetaddr}
\frameLocal{1}{\frameDefinitelyLocalOne}
\end{funFrame}

\begin{funFrame}[color=SeaGreen]{\frameCamlRaiseExnBegin}{\frameCamlRaiseExnEnd}
\funReturnAddr{\frameCamlRaiseExnRetaddr}
\end{funFrame}
}{}


\ifthenelse{\step>5 \and \step<10}{
\begin{funFrame}[name=reraise,color=JungleGreen]{\frameCamlReraiseExnBeginvB}{\frameCamlReraiseExnEndvB}
\funReturnAddr{\frameCamlReraiseExnRetaddrvB}
\end{funFrame}
}{}

\ifthenelse{\step>10 \and \step<13}{
\begin{funFrame}[name=reraise,color=JungleGreen]{\frameCamlReraiseExnBeginvC}{\frameCamlReraiseExnEndvC}
\funReturnAddr{\frameCamlReraiseExnRetaddrvC}
\end{funFrame}
}{}


\coordinate (bottom) at ($(stackBL) + (0,-0.25)$);


\ifthenelse{\step=1 \or \step=2}{
\domainStateBacktrace[pos=(bottom),yFactor=0.03]{\exnHandlerAfterExnATrap}{}{0}
\draw [->,>=latex] (exnHandlerRight.east) to [out=0,in=0] (exnAtrap.east);
}{}

\ifthenelse{\step>4 \and \step<7}{
\domainStateBacktrace[pos=(bottom),yFactor=0.03]{\exnHandlerAfterExnPopTrapvA}{\nextFrameDescrPCvB}{\stashBacktraceBacktracePosvA}
}{}

\ifthenelse{\step>9 \and \step<12}{
\domainStateBacktrace[pos=(bottom),yFactor=0.03]{\exnHandlerAfterExnPopTrapvB}{\nextFrameDescrPCvE}{\stashBacktraceBacktracePosvB}
}{}

\ifthenelse{\step=5}{
\draw [->,>=latex] (exnHandlerRight.east) to [out=0,in=0] (exnBtrap.east);
}{}

\ifthenelse{\step=10}{
\draw [->,>=latex] (exnHandlerRight.east) to [out=0,in=0] (exnCtrap.east);
}{}

\ifthenelse{\step=2}{
\showRegister{sp}{\stashBacktraceArgSPvA}
}{}

\ifthenelse{\step=3}{
\domainStateBacktrace[pos=(bottom),yFactor=0.03]{\exnHandlerAfterExnATrap}{\nextFrameDescrPCvA}{1}
\showRegister{sp}{\nextFrameDescrSPvA}
}{}

\ifthenelse{\step=4}{
\domainStateBacktrace[pos=(bottom),yFactor=0.03]{\exnHandlerAfterExnATrap}{\nextFrameDescrPCvB}{2}
\showRegister[opacity=0.25]{sp}{\nextFrameDescrSPvA}
\showRegister[offset=-0.8]{sp}{\nextFrameDescrSPvB}
}{}

% Drawing order matter for z-index trapsp/sp
\ifthenelse{\step>1 \and \step<5}{
\showRegister{trapsp}{\stashBacktraceArgTrapSPvA}
\showRegister{pc}{\frameCamlRaiseExnEnd}
}{}

\ifthenelse{\step>5 \and \step<10}{
\draw [->,>=latex] (exnHandlerRight.east) to [out=0,in=0] (exnBtrap.east);
}{}

\ifthenelse{\step=7}{
\domainStateBacktrace[pos=(bottom),yFactor=0.03]{\exnHandlerAfterExnPopTrapvA}{\nextFrameDescrPCvC}{3}
\showRegister[opacity=0.25]{sp}{\stashBacktraceArgSPvB}
\showRegister{sp}{\nextFrameDescrSPvC}
}{}

\ifthenelse{\step=8}{
\domainStateBacktrace[pos=(bottom),yFactor=0.03]{\exnHandlerAfterExnPopTrapvA}{\nextFrameDescrPCvD}{4}
\showRegister[opacity=0.25]{sp}{\nextFrameDescrSPvC}
\showRegister{sp}{\nextFrameDescrSPvD}
}{}

\ifthenelse{\step=9}{
\domainStateBacktrace[pos=(bottom),yFactor=0.03]{\exnHandlerAfterExnPopTrapvA}{\nextFrameDescrPCvE}{5}
\showRegister[opacity=0.25]{sp}{\nextFrameDescrSPvD}
\showRegister[offset=-0.8]{sp}{\nextFrameDescrSPvE}
}{}

% Drawing order matter for z-index trapsp/sp
\ifthenelse{\step>6 \and \step<10}{
\showRegister{pc}{\frameCamlReraiseExnEndvB}
\showRegister{trapsp}{\stashBacktraceArgTrapSPvB}
}{}


\ifthenelse{\step>10 \and \step<13}{
\draw [->,>=latex] (exnHandlerRight.east) to [out=0,in=0] (exnCtrap.east);
}{}

\ifthenelse{\step=12}{
\domainStateBacktrace[pos=(bottom),yFactor=0.03]{\exnHandlerAfterExnPopTrapvB}{\nextFrameDescrPCvF}{\stashBacktraceBacktracePosvC}
}{}

% Drawing order matter for z-index trapsp/sp
\ifthenelse{\step>10 \and \step<13}{
\showRegister{pc}{\frameCamlReraiseExnEndvC}
\showRegister{trapsp}{\stashBacktraceArgTrapSPvC}
}{}

\end{stackDiagram}
\end{tikzpicture}
}
%\onslide*<8>{\providecommand\step{8}\begin{tikzpicture}[font=\sffamily\tiny]
\input{../src/backtrace/gdb.out}

\newcommand\trapExnAEnd{\xinteval{\trapExnABegin - 2 * \wordSize}}
\newcommand\trapExnBEnd{\xinteval{\trapExnBBegin - 2 * \wordSize}}
\newcommand\trapExnCEnd{\xinteval{\trapExnCBegin - 2 * \wordSize}}

\begin{stackDiagram}[yFactor=0.35,showAddress=true]{\frameMainBegin}{\frameDefinitelyEnd}
\callStack

\begin{funFrame}[name=main,color=GreenYellow]{\frameMainBegin}{\frameMainEnd}
\funReturnAddr{\frameMainRetaddr}
\frameLocal{1}{\frameMainLocalOne}
\end{funFrame}

\ifthenelse{\step<13}{
\trapFrame[name=ExnC,handler=\trapExnCHandler,next=\trapExnCNext,color=WildStrawberry]{\trapExnCBegin}
\coordAtRight{exnCtrap}{\exnHandlerAfterExnCTrap}
}{}


\ifthenelse{\step<10}{
\trapFrame[name=ExnB,handler=\trapExnBHandler,next=\trapExnBNext,color=OrangeRed]{\trapExnBBegin}
\coordAtRight{exnBtrap}{\exnHandlerAfterExnBTrap}
\draw [->,>=latex] (exnBtrap.east) to [out=0,in=0] (exnCtrap.east);

\begin{funFrame}[name=maybe,color=YellowGreen]{\frameMaybeBegin}{\frameMaybeEnd}
\funReturnAddr{\frameMaybeRetaddr}
\frameLocal{1}{\frameMaybeLocalOne}
\end{funFrame}

\begin{funFrame}[name=probably,color=Green]{\frameProbablyBegin}{\frameProbablyEnd}
\funReturnAddr{\frameProbablyRetaddr}
\frameLocal{1}{\frameProbablyLocalOne}
\end{funFrame}
}{}


\ifthenelse{\step<5}{
\trapFrame[name=ExnA,handler=\trapExnAHandler,next=\trapExnANext,color=Red]{\trapExnABegin}
\coordAtRight{exnAtrap}{\exnHandlerAfterExnATrap}
\draw [->,>=latex] (exnAtrap.east) to [out=0,in=0] (exnBtrap.east);

\begin{funFrame}[name=definitely,color=JungleGreen]{\frameDefinitelyBegin}{\frameDefinitelyEnd}
\funReturnAddr{\frameDefinitelyRetaddr}
\frameLocal{1}{\frameDefinitelyLocalOne}
\end{funFrame}

\begin{funFrame}[color=SeaGreen]{\frameCamlRaiseExnBegin}{\frameCamlRaiseExnEnd}
\funReturnAddr{\frameCamlRaiseExnRetaddr}
\end{funFrame}
}{}


\ifthenelse{\step>5 \and \step<10}{
\begin{funFrame}[name=reraise,color=JungleGreen]{\frameCamlReraiseExnBeginvB}{\frameCamlReraiseExnEndvB}
\funReturnAddr{\frameCamlReraiseExnRetaddrvB}
\end{funFrame}
}{}

\ifthenelse{\step>10 \and \step<13}{
\begin{funFrame}[name=reraise,color=JungleGreen]{\frameCamlReraiseExnBeginvC}{\frameCamlReraiseExnEndvC}
\funReturnAddr{\frameCamlReraiseExnRetaddrvC}
\end{funFrame}
}{}


\coordinate (bottom) at ($(stackBL) + (0,-0.25)$);


\ifthenelse{\step=1 \or \step=2}{
\domainStateBacktrace[pos=(bottom),yFactor=0.03]{\exnHandlerAfterExnATrap}{}{0}
\draw [->,>=latex] (exnHandlerRight.east) to [out=0,in=0] (exnAtrap.east);
}{}

\ifthenelse{\step>4 \and \step<7}{
\domainStateBacktrace[pos=(bottom),yFactor=0.03]{\exnHandlerAfterExnPopTrapvA}{\nextFrameDescrPCvB}{\stashBacktraceBacktracePosvA}
}{}

\ifthenelse{\step>9 \and \step<12}{
\domainStateBacktrace[pos=(bottom),yFactor=0.03]{\exnHandlerAfterExnPopTrapvB}{\nextFrameDescrPCvE}{\stashBacktraceBacktracePosvB}
}{}

\ifthenelse{\step=5}{
\draw [->,>=latex] (exnHandlerRight.east) to [out=0,in=0] (exnBtrap.east);
}{}

\ifthenelse{\step=10}{
\draw [->,>=latex] (exnHandlerRight.east) to [out=0,in=0] (exnCtrap.east);
}{}

\ifthenelse{\step=2}{
\showRegister{sp}{\stashBacktraceArgSPvA}
}{}

\ifthenelse{\step=3}{
\domainStateBacktrace[pos=(bottom),yFactor=0.03]{\exnHandlerAfterExnATrap}{\nextFrameDescrPCvA}{1}
\showRegister{sp}{\nextFrameDescrSPvA}
}{}

\ifthenelse{\step=4}{
\domainStateBacktrace[pos=(bottom),yFactor=0.03]{\exnHandlerAfterExnATrap}{\nextFrameDescrPCvB}{2}
\showRegister[opacity=0.25]{sp}{\nextFrameDescrSPvA}
\showRegister[offset=-0.8]{sp}{\nextFrameDescrSPvB}
}{}

% Drawing order matter for z-index trapsp/sp
\ifthenelse{\step>1 \and \step<5}{
\showRegister{trapsp}{\stashBacktraceArgTrapSPvA}
\showRegister{pc}{\frameCamlRaiseExnEnd}
}{}

\ifthenelse{\step>5 \and \step<10}{
\draw [->,>=latex] (exnHandlerRight.east) to [out=0,in=0] (exnBtrap.east);
}{}

\ifthenelse{\step=7}{
\domainStateBacktrace[pos=(bottom),yFactor=0.03]{\exnHandlerAfterExnPopTrapvA}{\nextFrameDescrPCvC}{3}
\showRegister[opacity=0.25]{sp}{\stashBacktraceArgSPvB}
\showRegister{sp}{\nextFrameDescrSPvC}
}{}

\ifthenelse{\step=8}{
\domainStateBacktrace[pos=(bottom),yFactor=0.03]{\exnHandlerAfterExnPopTrapvA}{\nextFrameDescrPCvD}{4}
\showRegister[opacity=0.25]{sp}{\nextFrameDescrSPvC}
\showRegister{sp}{\nextFrameDescrSPvD}
}{}

\ifthenelse{\step=9}{
\domainStateBacktrace[pos=(bottom),yFactor=0.03]{\exnHandlerAfterExnPopTrapvA}{\nextFrameDescrPCvE}{5}
\showRegister[opacity=0.25]{sp}{\nextFrameDescrSPvD}
\showRegister[offset=-0.8]{sp}{\nextFrameDescrSPvE}
}{}

% Drawing order matter for z-index trapsp/sp
\ifthenelse{\step>6 \and \step<10}{
\showRegister{pc}{\frameCamlReraiseExnEndvB}
\showRegister{trapsp}{\stashBacktraceArgTrapSPvB}
}{}


\ifthenelse{\step>10 \and \step<13}{
\draw [->,>=latex] (exnHandlerRight.east) to [out=0,in=0] (exnCtrap.east);
}{}

\ifthenelse{\step=12}{
\domainStateBacktrace[pos=(bottom),yFactor=0.03]{\exnHandlerAfterExnPopTrapvB}{\nextFrameDescrPCvF}{\stashBacktraceBacktracePosvC}
}{}

% Drawing order matter for z-index trapsp/sp
\ifthenelse{\step>10 \and \step<13}{
\showRegister{pc}{\frameCamlReraiseExnEndvC}
\showRegister{trapsp}{\stashBacktraceArgTrapSPvC}
}{}

\end{stackDiagram}
\end{tikzpicture}
}
%\onslide*<9>{\providecommand\step{9}\begin{tikzpicture}[font=\sffamily\tiny]
\input{../src/backtrace/gdb.out}

\newcommand\trapExnAEnd{\xinteval{\trapExnABegin - 2 * \wordSize}}
\newcommand\trapExnBEnd{\xinteval{\trapExnBBegin - 2 * \wordSize}}
\newcommand\trapExnCEnd{\xinteval{\trapExnCBegin - 2 * \wordSize}}

\begin{stackDiagram}[yFactor=0.35,showAddress=true]{\frameMainBegin}{\frameDefinitelyEnd}
\callStack

\begin{funFrame}[name=main,color=GreenYellow]{\frameMainBegin}{\frameMainEnd}
\funReturnAddr{\frameMainRetaddr}
\frameLocal{1}{\frameMainLocalOne}
\end{funFrame}

\ifthenelse{\step<13}{
\trapFrame[name=ExnC,handler=\trapExnCHandler,next=\trapExnCNext,color=WildStrawberry]{\trapExnCBegin}
\coordAtRight{exnCtrap}{\exnHandlerAfterExnCTrap}
}{}


\ifthenelse{\step<10}{
\trapFrame[name=ExnB,handler=\trapExnBHandler,next=\trapExnBNext,color=OrangeRed]{\trapExnBBegin}
\coordAtRight{exnBtrap}{\exnHandlerAfterExnBTrap}
\draw [->,>=latex] (exnBtrap.east) to [out=0,in=0] (exnCtrap.east);

\begin{funFrame}[name=maybe,color=YellowGreen]{\frameMaybeBegin}{\frameMaybeEnd}
\funReturnAddr{\frameMaybeRetaddr}
\frameLocal{1}{\frameMaybeLocalOne}
\end{funFrame}

\begin{funFrame}[name=probably,color=Green]{\frameProbablyBegin}{\frameProbablyEnd}
\funReturnAddr{\frameProbablyRetaddr}
\frameLocal{1}{\frameProbablyLocalOne}
\end{funFrame}
}{}


\ifthenelse{\step<5}{
\trapFrame[name=ExnA,handler=\trapExnAHandler,next=\trapExnANext,color=Red]{\trapExnABegin}
\coordAtRight{exnAtrap}{\exnHandlerAfterExnATrap}
\draw [->,>=latex] (exnAtrap.east) to [out=0,in=0] (exnBtrap.east);

\begin{funFrame}[name=definitely,color=JungleGreen]{\frameDefinitelyBegin}{\frameDefinitelyEnd}
\funReturnAddr{\frameDefinitelyRetaddr}
\frameLocal{1}{\frameDefinitelyLocalOne}
\end{funFrame}

\begin{funFrame}[color=SeaGreen]{\frameCamlRaiseExnBegin}{\frameCamlRaiseExnEnd}
\funReturnAddr{\frameCamlRaiseExnRetaddr}
\end{funFrame}
}{}


\ifthenelse{\step>5 \and \step<10}{
\begin{funFrame}[name=reraise,color=JungleGreen]{\frameCamlReraiseExnBeginvB}{\frameCamlReraiseExnEndvB}
\funReturnAddr{\frameCamlReraiseExnRetaddrvB}
\end{funFrame}
}{}

\ifthenelse{\step>10 \and \step<13}{
\begin{funFrame}[name=reraise,color=JungleGreen]{\frameCamlReraiseExnBeginvC}{\frameCamlReraiseExnEndvC}
\funReturnAddr{\frameCamlReraiseExnRetaddrvC}
\end{funFrame}
}{}


\coordinate (bottom) at ($(stackBL) + (0,-0.25)$);


\ifthenelse{\step=1 \or \step=2}{
\domainStateBacktrace[pos=(bottom),yFactor=0.03]{\exnHandlerAfterExnATrap}{}{0}
\draw [->,>=latex] (exnHandlerRight.east) to [out=0,in=0] (exnAtrap.east);
}{}

\ifthenelse{\step>4 \and \step<7}{
\domainStateBacktrace[pos=(bottom),yFactor=0.03]{\exnHandlerAfterExnPopTrapvA}{\nextFrameDescrPCvB}{\stashBacktraceBacktracePosvA}
}{}

\ifthenelse{\step>9 \and \step<12}{
\domainStateBacktrace[pos=(bottom),yFactor=0.03]{\exnHandlerAfterExnPopTrapvB}{\nextFrameDescrPCvE}{\stashBacktraceBacktracePosvB}
}{}

\ifthenelse{\step=5}{
\draw [->,>=latex] (exnHandlerRight.east) to [out=0,in=0] (exnBtrap.east);
}{}

\ifthenelse{\step=10}{
\draw [->,>=latex] (exnHandlerRight.east) to [out=0,in=0] (exnCtrap.east);
}{}

\ifthenelse{\step=2}{
\showRegister{sp}{\stashBacktraceArgSPvA}
}{}

\ifthenelse{\step=3}{
\domainStateBacktrace[pos=(bottom),yFactor=0.03]{\exnHandlerAfterExnATrap}{\nextFrameDescrPCvA}{1}
\showRegister{sp}{\nextFrameDescrSPvA}
}{}

\ifthenelse{\step=4}{
\domainStateBacktrace[pos=(bottom),yFactor=0.03]{\exnHandlerAfterExnATrap}{\nextFrameDescrPCvB}{2}
\showRegister[opacity=0.25]{sp}{\nextFrameDescrSPvA}
\showRegister[offset=-0.8]{sp}{\nextFrameDescrSPvB}
}{}

% Drawing order matter for z-index trapsp/sp
\ifthenelse{\step>1 \and \step<5}{
\showRegister{trapsp}{\stashBacktraceArgTrapSPvA}
\showRegister{pc}{\frameCamlRaiseExnEnd}
}{}

\ifthenelse{\step>5 \and \step<10}{
\draw [->,>=latex] (exnHandlerRight.east) to [out=0,in=0] (exnBtrap.east);
}{}

\ifthenelse{\step=7}{
\domainStateBacktrace[pos=(bottom),yFactor=0.03]{\exnHandlerAfterExnPopTrapvA}{\nextFrameDescrPCvC}{3}
\showRegister[opacity=0.25]{sp}{\stashBacktraceArgSPvB}
\showRegister{sp}{\nextFrameDescrSPvC}
}{}

\ifthenelse{\step=8}{
\domainStateBacktrace[pos=(bottom),yFactor=0.03]{\exnHandlerAfterExnPopTrapvA}{\nextFrameDescrPCvD}{4}
\showRegister[opacity=0.25]{sp}{\nextFrameDescrSPvC}
\showRegister{sp}{\nextFrameDescrSPvD}
}{}

\ifthenelse{\step=9}{
\domainStateBacktrace[pos=(bottom),yFactor=0.03]{\exnHandlerAfterExnPopTrapvA}{\nextFrameDescrPCvE}{5}
\showRegister[opacity=0.25]{sp}{\nextFrameDescrSPvD}
\showRegister[offset=-0.8]{sp}{\nextFrameDescrSPvE}
}{}

% Drawing order matter for z-index trapsp/sp
\ifthenelse{\step>6 \and \step<10}{
\showRegister{pc}{\frameCamlReraiseExnEndvB}
\showRegister{trapsp}{\stashBacktraceArgTrapSPvB}
}{}


\ifthenelse{\step>10 \and \step<13}{
\draw [->,>=latex] (exnHandlerRight.east) to [out=0,in=0] (exnCtrap.east);
}{}

\ifthenelse{\step=12}{
\domainStateBacktrace[pos=(bottom),yFactor=0.03]{\exnHandlerAfterExnPopTrapvB}{\nextFrameDescrPCvF}{\stashBacktraceBacktracePosvC}
}{}

% Drawing order matter for z-index trapsp/sp
\ifthenelse{\step>10 \and \step<13}{
\showRegister{pc}{\frameCamlReraiseExnEndvC}
\showRegister{trapsp}{\stashBacktraceArgTrapSPvC}
}{}

\end{stackDiagram}
\end{tikzpicture}
}
%\onslide*<10>{\providecommand\step{10}\begin{tikzpicture}[font=\sffamily\tiny]
\input{../src/backtrace/gdb.out}

\newcommand\trapExnAEnd{\xinteval{\trapExnABegin - 2 * \wordSize}}
\newcommand\trapExnBEnd{\xinteval{\trapExnBBegin - 2 * \wordSize}}
\newcommand\trapExnCEnd{\xinteval{\trapExnCBegin - 2 * \wordSize}}

\begin{stackDiagram}[yFactor=0.35,showAddress=true]{\frameMainBegin}{\frameDefinitelyEnd}
\callStack

\begin{funFrame}[name=main,color=GreenYellow]{\frameMainBegin}{\frameMainEnd}
\funReturnAddr{\frameMainRetaddr}
\frameLocal{1}{\frameMainLocalOne}
\end{funFrame}

\ifthenelse{\step<13}{
\trapFrame[name=ExnC,handler=\trapExnCHandler,next=\trapExnCNext,color=WildStrawberry]{\trapExnCBegin}
\coordAtRight{exnCtrap}{\exnHandlerAfterExnCTrap}
}{}


\ifthenelse{\step<10}{
\trapFrame[name=ExnB,handler=\trapExnBHandler,next=\trapExnBNext,color=OrangeRed]{\trapExnBBegin}
\coordAtRight{exnBtrap}{\exnHandlerAfterExnBTrap}
\draw [->,>=latex] (exnBtrap.east) to [out=0,in=0] (exnCtrap.east);

\begin{funFrame}[name=maybe,color=YellowGreen]{\frameMaybeBegin}{\frameMaybeEnd}
\funReturnAddr{\frameMaybeRetaddr}
\frameLocal{1}{\frameMaybeLocalOne}
\end{funFrame}

\begin{funFrame}[name=probably,color=Green]{\frameProbablyBegin}{\frameProbablyEnd}
\funReturnAddr{\frameProbablyRetaddr}
\frameLocal{1}{\frameProbablyLocalOne}
\end{funFrame}
}{}


\ifthenelse{\step<5}{
\trapFrame[name=ExnA,handler=\trapExnAHandler,next=\trapExnANext,color=Red]{\trapExnABegin}
\coordAtRight{exnAtrap}{\exnHandlerAfterExnATrap}
\draw [->,>=latex] (exnAtrap.east) to [out=0,in=0] (exnBtrap.east);

\begin{funFrame}[name=definitely,color=JungleGreen]{\frameDefinitelyBegin}{\frameDefinitelyEnd}
\funReturnAddr{\frameDefinitelyRetaddr}
\frameLocal{1}{\frameDefinitelyLocalOne}
\end{funFrame}

\begin{funFrame}[color=SeaGreen]{\frameCamlRaiseExnBegin}{\frameCamlRaiseExnEnd}
\funReturnAddr{\frameCamlRaiseExnRetaddr}
\end{funFrame}
}{}


\ifthenelse{\step>5 \and \step<10}{
\begin{funFrame}[name=reraise,color=JungleGreen]{\frameCamlReraiseExnBeginvB}{\frameCamlReraiseExnEndvB}
\funReturnAddr{\frameCamlReraiseExnRetaddrvB}
\end{funFrame}
}{}

\ifthenelse{\step>10 \and \step<13}{
\begin{funFrame}[name=reraise,color=JungleGreen]{\frameCamlReraiseExnBeginvC}{\frameCamlReraiseExnEndvC}
\funReturnAddr{\frameCamlReraiseExnRetaddrvC}
\end{funFrame}
}{}


\coordinate (bottom) at ($(stackBL) + (0,-0.25)$);


\ifthenelse{\step=1 \or \step=2}{
\domainStateBacktrace[pos=(bottom),yFactor=0.03]{\exnHandlerAfterExnATrap}{}{0}
\draw [->,>=latex] (exnHandlerRight.east) to [out=0,in=0] (exnAtrap.east);
}{}

\ifthenelse{\step>4 \and \step<7}{
\domainStateBacktrace[pos=(bottom),yFactor=0.03]{\exnHandlerAfterExnPopTrapvA}{\nextFrameDescrPCvB}{\stashBacktraceBacktracePosvA}
}{}

\ifthenelse{\step>9 \and \step<12}{
\domainStateBacktrace[pos=(bottom),yFactor=0.03]{\exnHandlerAfterExnPopTrapvB}{\nextFrameDescrPCvE}{\stashBacktraceBacktracePosvB}
}{}

\ifthenelse{\step=5}{
\draw [->,>=latex] (exnHandlerRight.east) to [out=0,in=0] (exnBtrap.east);
}{}

\ifthenelse{\step=10}{
\draw [->,>=latex] (exnHandlerRight.east) to [out=0,in=0] (exnCtrap.east);
}{}

\ifthenelse{\step=2}{
\showRegister{sp}{\stashBacktraceArgSPvA}
}{}

\ifthenelse{\step=3}{
\domainStateBacktrace[pos=(bottom),yFactor=0.03]{\exnHandlerAfterExnATrap}{\nextFrameDescrPCvA}{1}
\showRegister{sp}{\nextFrameDescrSPvA}
}{}

\ifthenelse{\step=4}{
\domainStateBacktrace[pos=(bottom),yFactor=0.03]{\exnHandlerAfterExnATrap}{\nextFrameDescrPCvB}{2}
\showRegister[opacity=0.25]{sp}{\nextFrameDescrSPvA}
\showRegister[offset=-0.8]{sp}{\nextFrameDescrSPvB}
}{}

% Drawing order matter for z-index trapsp/sp
\ifthenelse{\step>1 \and \step<5}{
\showRegister{trapsp}{\stashBacktraceArgTrapSPvA}
\showRegister{pc}{\frameCamlRaiseExnEnd}
}{}

\ifthenelse{\step>5 \and \step<10}{
\draw [->,>=latex] (exnHandlerRight.east) to [out=0,in=0] (exnBtrap.east);
}{}

\ifthenelse{\step=7}{
\domainStateBacktrace[pos=(bottom),yFactor=0.03]{\exnHandlerAfterExnPopTrapvA}{\nextFrameDescrPCvC}{3}
\showRegister[opacity=0.25]{sp}{\stashBacktraceArgSPvB}
\showRegister{sp}{\nextFrameDescrSPvC}
}{}

\ifthenelse{\step=8}{
\domainStateBacktrace[pos=(bottom),yFactor=0.03]{\exnHandlerAfterExnPopTrapvA}{\nextFrameDescrPCvD}{4}
\showRegister[opacity=0.25]{sp}{\nextFrameDescrSPvC}
\showRegister{sp}{\nextFrameDescrSPvD}
}{}

\ifthenelse{\step=9}{
\domainStateBacktrace[pos=(bottom),yFactor=0.03]{\exnHandlerAfterExnPopTrapvA}{\nextFrameDescrPCvE}{5}
\showRegister[opacity=0.25]{sp}{\nextFrameDescrSPvD}
\showRegister[offset=-0.8]{sp}{\nextFrameDescrSPvE}
}{}

% Drawing order matter for z-index trapsp/sp
\ifthenelse{\step>6 \and \step<10}{
\showRegister{pc}{\frameCamlReraiseExnEndvB}
\showRegister{trapsp}{\stashBacktraceArgTrapSPvB}
}{}


\ifthenelse{\step>10 \and \step<13}{
\draw [->,>=latex] (exnHandlerRight.east) to [out=0,in=0] (exnCtrap.east);
}{}

\ifthenelse{\step=12}{
\domainStateBacktrace[pos=(bottom),yFactor=0.03]{\exnHandlerAfterExnPopTrapvB}{\nextFrameDescrPCvF}{\stashBacktraceBacktracePosvC}
}{}

% Drawing order matter for z-index trapsp/sp
\ifthenelse{\step>10 \and \step<13}{
\showRegister{pc}{\frameCamlReraiseExnEndvC}
\showRegister{trapsp}{\stashBacktraceArgTrapSPvC}
}{}

\end{stackDiagram}
\end{tikzpicture}
}
%\onslide*<11>{\providecommand\step{11}\begin{tikzpicture}[font=\sffamily\tiny]
\input{../src/backtrace/gdb.out}

\newcommand\trapExnAEnd{\xinteval{\trapExnABegin - 2 * \wordSize}}
\newcommand\trapExnBEnd{\xinteval{\trapExnBBegin - 2 * \wordSize}}
\newcommand\trapExnCEnd{\xinteval{\trapExnCBegin - 2 * \wordSize}}

\begin{stackDiagram}[yFactor=0.35,showAddress=true]{\frameMainBegin}{\frameDefinitelyEnd}
\callStack

\begin{funFrame}[name=main,color=GreenYellow]{\frameMainBegin}{\frameMainEnd}
\funReturnAddr{\frameMainRetaddr}
\frameLocal{1}{\frameMainLocalOne}
\end{funFrame}

\ifthenelse{\step<13}{
\trapFrame[name=ExnC,handler=\trapExnCHandler,next=\trapExnCNext,color=WildStrawberry]{\trapExnCBegin}
\coordAtRight{exnCtrap}{\exnHandlerAfterExnCTrap}
}{}


\ifthenelse{\step<10}{
\trapFrame[name=ExnB,handler=\trapExnBHandler,next=\trapExnBNext,color=OrangeRed]{\trapExnBBegin}
\coordAtRight{exnBtrap}{\exnHandlerAfterExnBTrap}
\draw [->,>=latex] (exnBtrap.east) to [out=0,in=0] (exnCtrap.east);

\begin{funFrame}[name=maybe,color=YellowGreen]{\frameMaybeBegin}{\frameMaybeEnd}
\funReturnAddr{\frameMaybeRetaddr}
\frameLocal{1}{\frameMaybeLocalOne}
\end{funFrame}

\begin{funFrame}[name=probably,color=Green]{\frameProbablyBegin}{\frameProbablyEnd}
\funReturnAddr{\frameProbablyRetaddr}
\frameLocal{1}{\frameProbablyLocalOne}
\end{funFrame}
}{}


\ifthenelse{\step<5}{
\trapFrame[name=ExnA,handler=\trapExnAHandler,next=\trapExnANext,color=Red]{\trapExnABegin}
\coordAtRight{exnAtrap}{\exnHandlerAfterExnATrap}
\draw [->,>=latex] (exnAtrap.east) to [out=0,in=0] (exnBtrap.east);

\begin{funFrame}[name=definitely,color=JungleGreen]{\frameDefinitelyBegin}{\frameDefinitelyEnd}
\funReturnAddr{\frameDefinitelyRetaddr}
\frameLocal{1}{\frameDefinitelyLocalOne}
\end{funFrame}

\begin{funFrame}[color=SeaGreen]{\frameCamlRaiseExnBegin}{\frameCamlRaiseExnEnd}
\funReturnAddr{\frameCamlRaiseExnRetaddr}
\end{funFrame}
}{}


\ifthenelse{\step>5 \and \step<10}{
\begin{funFrame}[name=reraise,color=JungleGreen]{\frameCamlReraiseExnBeginvB}{\frameCamlReraiseExnEndvB}
\funReturnAddr{\frameCamlReraiseExnRetaddrvB}
\end{funFrame}
}{}

\ifthenelse{\step>10 \and \step<13}{
\begin{funFrame}[name=reraise,color=JungleGreen]{\frameCamlReraiseExnBeginvC}{\frameCamlReraiseExnEndvC}
\funReturnAddr{\frameCamlReraiseExnRetaddrvC}
\end{funFrame}
}{}


\coordinate (bottom) at ($(stackBL) + (0,-0.25)$);


\ifthenelse{\step=1 \or \step=2}{
\domainStateBacktrace[pos=(bottom),yFactor=0.03]{\exnHandlerAfterExnATrap}{}{0}
\draw [->,>=latex] (exnHandlerRight.east) to [out=0,in=0] (exnAtrap.east);
}{}

\ifthenelse{\step>4 \and \step<7}{
\domainStateBacktrace[pos=(bottom),yFactor=0.03]{\exnHandlerAfterExnPopTrapvA}{\nextFrameDescrPCvB}{\stashBacktraceBacktracePosvA}
}{}

\ifthenelse{\step>9 \and \step<12}{
\domainStateBacktrace[pos=(bottom),yFactor=0.03]{\exnHandlerAfterExnPopTrapvB}{\nextFrameDescrPCvE}{\stashBacktraceBacktracePosvB}
}{}

\ifthenelse{\step=5}{
\draw [->,>=latex] (exnHandlerRight.east) to [out=0,in=0] (exnBtrap.east);
}{}

\ifthenelse{\step=10}{
\draw [->,>=latex] (exnHandlerRight.east) to [out=0,in=0] (exnCtrap.east);
}{}

\ifthenelse{\step=2}{
\showRegister{sp}{\stashBacktraceArgSPvA}
}{}

\ifthenelse{\step=3}{
\domainStateBacktrace[pos=(bottom),yFactor=0.03]{\exnHandlerAfterExnATrap}{\nextFrameDescrPCvA}{1}
\showRegister{sp}{\nextFrameDescrSPvA}
}{}

\ifthenelse{\step=4}{
\domainStateBacktrace[pos=(bottom),yFactor=0.03]{\exnHandlerAfterExnATrap}{\nextFrameDescrPCvB}{2}
\showRegister[opacity=0.25]{sp}{\nextFrameDescrSPvA}
\showRegister[offset=-0.8]{sp}{\nextFrameDescrSPvB}
}{}

% Drawing order matter for z-index trapsp/sp
\ifthenelse{\step>1 \and \step<5}{
\showRegister{trapsp}{\stashBacktraceArgTrapSPvA}
\showRegister{pc}{\frameCamlRaiseExnEnd}
}{}

\ifthenelse{\step>5 \and \step<10}{
\draw [->,>=latex] (exnHandlerRight.east) to [out=0,in=0] (exnBtrap.east);
}{}

\ifthenelse{\step=7}{
\domainStateBacktrace[pos=(bottom),yFactor=0.03]{\exnHandlerAfterExnPopTrapvA}{\nextFrameDescrPCvC}{3}
\showRegister[opacity=0.25]{sp}{\stashBacktraceArgSPvB}
\showRegister{sp}{\nextFrameDescrSPvC}
}{}

\ifthenelse{\step=8}{
\domainStateBacktrace[pos=(bottom),yFactor=0.03]{\exnHandlerAfterExnPopTrapvA}{\nextFrameDescrPCvD}{4}
\showRegister[opacity=0.25]{sp}{\nextFrameDescrSPvC}
\showRegister{sp}{\nextFrameDescrSPvD}
}{}

\ifthenelse{\step=9}{
\domainStateBacktrace[pos=(bottom),yFactor=0.03]{\exnHandlerAfterExnPopTrapvA}{\nextFrameDescrPCvE}{5}
\showRegister[opacity=0.25]{sp}{\nextFrameDescrSPvD}
\showRegister[offset=-0.8]{sp}{\nextFrameDescrSPvE}
}{}

% Drawing order matter for z-index trapsp/sp
\ifthenelse{\step>6 \and \step<10}{
\showRegister{pc}{\frameCamlReraiseExnEndvB}
\showRegister{trapsp}{\stashBacktraceArgTrapSPvB}
}{}


\ifthenelse{\step>10 \and \step<13}{
\draw [->,>=latex] (exnHandlerRight.east) to [out=0,in=0] (exnCtrap.east);
}{}

\ifthenelse{\step=12}{
\domainStateBacktrace[pos=(bottom),yFactor=0.03]{\exnHandlerAfterExnPopTrapvB}{\nextFrameDescrPCvF}{\stashBacktraceBacktracePosvC}
}{}

% Drawing order matter for z-index trapsp/sp
\ifthenelse{\step>10 \and \step<13}{
\showRegister{pc}{\frameCamlReraiseExnEndvC}
\showRegister{trapsp}{\stashBacktraceArgTrapSPvC}
}{}

\end{stackDiagram}
\end{tikzpicture}
}
%\onslide*<12>{\providecommand\step{12}\begin{tikzpicture}[font=\sffamily\tiny]
\input{../src/backtrace/gdb.out}

\newcommand\trapExnAEnd{\xinteval{\trapExnABegin - 2 * \wordSize}}
\newcommand\trapExnBEnd{\xinteval{\trapExnBBegin - 2 * \wordSize}}
\newcommand\trapExnCEnd{\xinteval{\trapExnCBegin - 2 * \wordSize}}

\begin{stackDiagram}[yFactor=0.35,showAddress=true]{\frameMainBegin}{\frameDefinitelyEnd}
\callStack

\begin{funFrame}[name=main,color=GreenYellow]{\frameMainBegin}{\frameMainEnd}
\funReturnAddr{\frameMainRetaddr}
\frameLocal{1}{\frameMainLocalOne}
\end{funFrame}

\ifthenelse{\step<13}{
\trapFrame[name=ExnC,handler=\trapExnCHandler,next=\trapExnCNext,color=WildStrawberry]{\trapExnCBegin}
\coordAtRight{exnCtrap}{\exnHandlerAfterExnCTrap}
}{}


\ifthenelse{\step<10}{
\trapFrame[name=ExnB,handler=\trapExnBHandler,next=\trapExnBNext,color=OrangeRed]{\trapExnBBegin}
\coordAtRight{exnBtrap}{\exnHandlerAfterExnBTrap}
\draw [->,>=latex] (exnBtrap.east) to [out=0,in=0] (exnCtrap.east);

\begin{funFrame}[name=maybe,color=YellowGreen]{\frameMaybeBegin}{\frameMaybeEnd}
\funReturnAddr{\frameMaybeRetaddr}
\frameLocal{1}{\frameMaybeLocalOne}
\end{funFrame}

\begin{funFrame}[name=probably,color=Green]{\frameProbablyBegin}{\frameProbablyEnd}
\funReturnAddr{\frameProbablyRetaddr}
\frameLocal{1}{\frameProbablyLocalOne}
\end{funFrame}
}{}


\ifthenelse{\step<5}{
\trapFrame[name=ExnA,handler=\trapExnAHandler,next=\trapExnANext,color=Red]{\trapExnABegin}
\coordAtRight{exnAtrap}{\exnHandlerAfterExnATrap}
\draw [->,>=latex] (exnAtrap.east) to [out=0,in=0] (exnBtrap.east);

\begin{funFrame}[name=definitely,color=JungleGreen]{\frameDefinitelyBegin}{\frameDefinitelyEnd}
\funReturnAddr{\frameDefinitelyRetaddr}
\frameLocal{1}{\frameDefinitelyLocalOne}
\end{funFrame}

\begin{funFrame}[color=SeaGreen]{\frameCamlRaiseExnBegin}{\frameCamlRaiseExnEnd}
\funReturnAddr{\frameCamlRaiseExnRetaddr}
\end{funFrame}
}{}


\ifthenelse{\step>5 \and \step<10}{
\begin{funFrame}[name=reraise,color=JungleGreen]{\frameCamlReraiseExnBeginvB}{\frameCamlReraiseExnEndvB}
\funReturnAddr{\frameCamlReraiseExnRetaddrvB}
\end{funFrame}
}{}

\ifthenelse{\step>10 \and \step<13}{
\begin{funFrame}[name=reraise,color=JungleGreen]{\frameCamlReraiseExnBeginvC}{\frameCamlReraiseExnEndvC}
\funReturnAddr{\frameCamlReraiseExnRetaddrvC}
\end{funFrame}
}{}


\coordinate (bottom) at ($(stackBL) + (0,-0.25)$);


\ifthenelse{\step=1 \or \step=2}{
\domainStateBacktrace[pos=(bottom),yFactor=0.03]{\exnHandlerAfterExnATrap}{}{0}
\draw [->,>=latex] (exnHandlerRight.east) to [out=0,in=0] (exnAtrap.east);
}{}

\ifthenelse{\step>4 \and \step<7}{
\domainStateBacktrace[pos=(bottom),yFactor=0.03]{\exnHandlerAfterExnPopTrapvA}{\nextFrameDescrPCvB}{\stashBacktraceBacktracePosvA}
}{}

\ifthenelse{\step>9 \and \step<12}{
\domainStateBacktrace[pos=(bottom),yFactor=0.03]{\exnHandlerAfterExnPopTrapvB}{\nextFrameDescrPCvE}{\stashBacktraceBacktracePosvB}
}{}

\ifthenelse{\step=5}{
\draw [->,>=latex] (exnHandlerRight.east) to [out=0,in=0] (exnBtrap.east);
}{}

\ifthenelse{\step=10}{
\draw [->,>=latex] (exnHandlerRight.east) to [out=0,in=0] (exnCtrap.east);
}{}

\ifthenelse{\step=2}{
\showRegister{sp}{\stashBacktraceArgSPvA}
}{}

\ifthenelse{\step=3}{
\domainStateBacktrace[pos=(bottom),yFactor=0.03]{\exnHandlerAfterExnATrap}{\nextFrameDescrPCvA}{1}
\showRegister{sp}{\nextFrameDescrSPvA}
}{}

\ifthenelse{\step=4}{
\domainStateBacktrace[pos=(bottom),yFactor=0.03]{\exnHandlerAfterExnATrap}{\nextFrameDescrPCvB}{2}
\showRegister[opacity=0.25]{sp}{\nextFrameDescrSPvA}
\showRegister[offset=-0.8]{sp}{\nextFrameDescrSPvB}
}{}

% Drawing order matter for z-index trapsp/sp
\ifthenelse{\step>1 \and \step<5}{
\showRegister{trapsp}{\stashBacktraceArgTrapSPvA}
\showRegister{pc}{\frameCamlRaiseExnEnd}
}{}

\ifthenelse{\step>5 \and \step<10}{
\draw [->,>=latex] (exnHandlerRight.east) to [out=0,in=0] (exnBtrap.east);
}{}

\ifthenelse{\step=7}{
\domainStateBacktrace[pos=(bottom),yFactor=0.03]{\exnHandlerAfterExnPopTrapvA}{\nextFrameDescrPCvC}{3}
\showRegister[opacity=0.25]{sp}{\stashBacktraceArgSPvB}
\showRegister{sp}{\nextFrameDescrSPvC}
}{}

\ifthenelse{\step=8}{
\domainStateBacktrace[pos=(bottom),yFactor=0.03]{\exnHandlerAfterExnPopTrapvA}{\nextFrameDescrPCvD}{4}
\showRegister[opacity=0.25]{sp}{\nextFrameDescrSPvC}
\showRegister{sp}{\nextFrameDescrSPvD}
}{}

\ifthenelse{\step=9}{
\domainStateBacktrace[pos=(bottom),yFactor=0.03]{\exnHandlerAfterExnPopTrapvA}{\nextFrameDescrPCvE}{5}
\showRegister[opacity=0.25]{sp}{\nextFrameDescrSPvD}
\showRegister[offset=-0.8]{sp}{\nextFrameDescrSPvE}
}{}

% Drawing order matter for z-index trapsp/sp
\ifthenelse{\step>6 \and \step<10}{
\showRegister{pc}{\frameCamlReraiseExnEndvB}
\showRegister{trapsp}{\stashBacktraceArgTrapSPvB}
}{}


\ifthenelse{\step>10 \and \step<13}{
\draw [->,>=latex] (exnHandlerRight.east) to [out=0,in=0] (exnCtrap.east);
}{}

\ifthenelse{\step=12}{
\domainStateBacktrace[pos=(bottom),yFactor=0.03]{\exnHandlerAfterExnPopTrapvB}{\nextFrameDescrPCvF}{\stashBacktraceBacktracePosvC}
}{}

% Drawing order matter for z-index trapsp/sp
\ifthenelse{\step>10 \and \step<13}{
\showRegister{pc}{\frameCamlReraiseExnEndvC}
\showRegister{trapsp}{\stashBacktraceArgTrapSPvC}
}{}

\end{stackDiagram}
\end{tikzpicture}
}
%\onslide*<13>{\providecommand\step{13}\begin{tikzpicture}[font=\sffamily\tiny]
\input{../src/backtrace/gdb.out}

\newcommand\trapExnAEnd{\xinteval{\trapExnABegin - 2 * \wordSize}}
\newcommand\trapExnBEnd{\xinteval{\trapExnBBegin - 2 * \wordSize}}
\newcommand\trapExnCEnd{\xinteval{\trapExnCBegin - 2 * \wordSize}}

\begin{stackDiagram}[yFactor=0.35,showAddress=true]{\frameMainBegin}{\frameDefinitelyEnd}
\callStack

\begin{funFrame}[name=main,color=GreenYellow]{\frameMainBegin}{\frameMainEnd}
\funReturnAddr{\frameMainRetaddr}
\frameLocal{1}{\frameMainLocalOne}
\end{funFrame}

\ifthenelse{\step<13}{
\trapFrame[name=ExnC,handler=\trapExnCHandler,next=\trapExnCNext,color=WildStrawberry]{\trapExnCBegin}
\coordAtRight{exnCtrap}{\exnHandlerAfterExnCTrap}
}{}


\ifthenelse{\step<10}{
\trapFrame[name=ExnB,handler=\trapExnBHandler,next=\trapExnBNext,color=OrangeRed]{\trapExnBBegin}
\coordAtRight{exnBtrap}{\exnHandlerAfterExnBTrap}
\draw [->,>=latex] (exnBtrap.east) to [out=0,in=0] (exnCtrap.east);

\begin{funFrame}[name=maybe,color=YellowGreen]{\frameMaybeBegin}{\frameMaybeEnd}
\funReturnAddr{\frameMaybeRetaddr}
\frameLocal{1}{\frameMaybeLocalOne}
\end{funFrame}

\begin{funFrame}[name=probably,color=Green]{\frameProbablyBegin}{\frameProbablyEnd}
\funReturnAddr{\frameProbablyRetaddr}
\frameLocal{1}{\frameProbablyLocalOne}
\end{funFrame}
}{}


\ifthenelse{\step<5}{
\trapFrame[name=ExnA,handler=\trapExnAHandler,next=\trapExnANext,color=Red]{\trapExnABegin}
\coordAtRight{exnAtrap}{\exnHandlerAfterExnATrap}
\draw [->,>=latex] (exnAtrap.east) to [out=0,in=0] (exnBtrap.east);

\begin{funFrame}[name=definitely,color=JungleGreen]{\frameDefinitelyBegin}{\frameDefinitelyEnd}
\funReturnAddr{\frameDefinitelyRetaddr}
\frameLocal{1}{\frameDefinitelyLocalOne}
\end{funFrame}

\begin{funFrame}[color=SeaGreen]{\frameCamlRaiseExnBegin}{\frameCamlRaiseExnEnd}
\funReturnAddr{\frameCamlRaiseExnRetaddr}
\end{funFrame}
}{}


\ifthenelse{\step>5 \and \step<10}{
\begin{funFrame}[name=reraise,color=JungleGreen]{\frameCamlReraiseExnBeginvB}{\frameCamlReraiseExnEndvB}
\funReturnAddr{\frameCamlReraiseExnRetaddrvB}
\end{funFrame}
}{}

\ifthenelse{\step>10 \and \step<13}{
\begin{funFrame}[name=reraise,color=JungleGreen]{\frameCamlReraiseExnBeginvC}{\frameCamlReraiseExnEndvC}
\funReturnAddr{\frameCamlReraiseExnRetaddrvC}
\end{funFrame}
}{}


\coordinate (bottom) at ($(stackBL) + (0,-0.25)$);


\ifthenelse{\step=1 \or \step=2}{
\domainStateBacktrace[pos=(bottom),yFactor=0.03]{\exnHandlerAfterExnATrap}{}{0}
\draw [->,>=latex] (exnHandlerRight.east) to [out=0,in=0] (exnAtrap.east);
}{}

\ifthenelse{\step>4 \and \step<7}{
\domainStateBacktrace[pos=(bottom),yFactor=0.03]{\exnHandlerAfterExnPopTrapvA}{\nextFrameDescrPCvB}{\stashBacktraceBacktracePosvA}
}{}

\ifthenelse{\step>9 \and \step<12}{
\domainStateBacktrace[pos=(bottom),yFactor=0.03]{\exnHandlerAfterExnPopTrapvB}{\nextFrameDescrPCvE}{\stashBacktraceBacktracePosvB}
}{}

\ifthenelse{\step=5}{
\draw [->,>=latex] (exnHandlerRight.east) to [out=0,in=0] (exnBtrap.east);
}{}

\ifthenelse{\step=10}{
\draw [->,>=latex] (exnHandlerRight.east) to [out=0,in=0] (exnCtrap.east);
}{}

\ifthenelse{\step=2}{
\showRegister{sp}{\stashBacktraceArgSPvA}
}{}

\ifthenelse{\step=3}{
\domainStateBacktrace[pos=(bottom),yFactor=0.03]{\exnHandlerAfterExnATrap}{\nextFrameDescrPCvA}{1}
\showRegister{sp}{\nextFrameDescrSPvA}
}{}

\ifthenelse{\step=4}{
\domainStateBacktrace[pos=(bottom),yFactor=0.03]{\exnHandlerAfterExnATrap}{\nextFrameDescrPCvB}{2}
\showRegister[opacity=0.25]{sp}{\nextFrameDescrSPvA}
\showRegister[offset=-0.8]{sp}{\nextFrameDescrSPvB}
}{}

% Drawing order matter for z-index trapsp/sp
\ifthenelse{\step>1 \and \step<5}{
\showRegister{trapsp}{\stashBacktraceArgTrapSPvA}
\showRegister{pc}{\frameCamlRaiseExnEnd}
}{}

\ifthenelse{\step>5 \and \step<10}{
\draw [->,>=latex] (exnHandlerRight.east) to [out=0,in=0] (exnBtrap.east);
}{}

\ifthenelse{\step=7}{
\domainStateBacktrace[pos=(bottom),yFactor=0.03]{\exnHandlerAfterExnPopTrapvA}{\nextFrameDescrPCvC}{3}
\showRegister[opacity=0.25]{sp}{\stashBacktraceArgSPvB}
\showRegister{sp}{\nextFrameDescrSPvC}
}{}

\ifthenelse{\step=8}{
\domainStateBacktrace[pos=(bottom),yFactor=0.03]{\exnHandlerAfterExnPopTrapvA}{\nextFrameDescrPCvD}{4}
\showRegister[opacity=0.25]{sp}{\nextFrameDescrSPvC}
\showRegister{sp}{\nextFrameDescrSPvD}
}{}

\ifthenelse{\step=9}{
\domainStateBacktrace[pos=(bottom),yFactor=0.03]{\exnHandlerAfterExnPopTrapvA}{\nextFrameDescrPCvE}{5}
\showRegister[opacity=0.25]{sp}{\nextFrameDescrSPvD}
\showRegister[offset=-0.8]{sp}{\nextFrameDescrSPvE}
}{}

% Drawing order matter for z-index trapsp/sp
\ifthenelse{\step>6 \and \step<10}{
\showRegister{pc}{\frameCamlReraiseExnEndvB}
\showRegister{trapsp}{\stashBacktraceArgTrapSPvB}
}{}


\ifthenelse{\step>10 \and \step<13}{
\draw [->,>=latex] (exnHandlerRight.east) to [out=0,in=0] (exnCtrap.east);
}{}

\ifthenelse{\step=12}{
\domainStateBacktrace[pos=(bottom),yFactor=0.03]{\exnHandlerAfterExnPopTrapvB}{\nextFrameDescrPCvF}{\stashBacktraceBacktracePosvC}
}{}

% Drawing order matter for z-index trapsp/sp
\ifthenelse{\step>10 \and \step<13}{
\showRegister{pc}{\frameCamlReraiseExnEndvC}
\showRegister{trapsp}{\stashBacktraceArgTrapSPvC}
}{}

\end{stackDiagram}
\end{tikzpicture}
}


\subsection{raise\_notrace}
