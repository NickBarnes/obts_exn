%
% Exceptions are values
%
\subsection{Exceptions are values}
\frameSubsection{pictures/Potentials.png}{- There is no spoon.}

% https://dev.realworldocaml.org/error-handling.html#exceptions
% https://dev.realworldocaml.org/runtime-memory-layout.html
% https://v2.ocaml.org/manual/extensiblevariants.html
\begin{frame}{Exceptions are ordinary values}
\begin{itemize}
\item Exceptions are ordinary values\footnotemark
\item No longer special: an extensible variant type (\text{``}not fully defined in one place\text{''})
\item \mintinline{ocaml}{exception Exc of int} is equivalent to \mintinline{ocaml}{type exn += Exc of int}
\item Like variants, if the exception has no parameter, memory allocation isn't needed
\item Exceptions with parameter (\mintinline{ocaml}{exception <name> of <type>}) need heap allocation
\end{itemize}
\footnotetext[1]{https://dev.realworldocaml.org/error-handling.html\#exceptions}
\footnotetext[2]{https://v2.ocaml.org/manual/extensiblevariants.html}
\end{frame}


%
% Takeaway #1
%
\subsection*{Takeaway \#1}
\frameSubsectionTakeaway{}
\begin{frame}<1>[label=takeaway]
\frametitle{Takeaway}
\begin{enumerate}
\item<1-> No allocation to create an exception value for exception without parameter
\item<2-> No allocation to install an exception handler
\item<3-> Jumping to the last installed exception handler is done in constant time $O(1)$
\only<3>{
\begin{itemize}
\item No matter how many stack frames between the raising point and the exception handler
\item Unless... see \#4
\end{itemize}
}
\item<4-> Catching an exception is linear complexity $O(n)$ in terms of number of installed exception \emph{handlers}
\only<4>{
\begin{itemize}
\item Algorithm: trap linked-list traversal
\begin{itemize}
\item Discard stack frames up to the last installed handler
\item Execute the handler's pattern matching
\item If no match, re-raise the exception
\end{itemize}
\item Catching an exception early, in terms of traversed handlers, is cheaper
\end{itemize}
}
\item<5-> Compiling with debug information (\mintinline{sh}{-g}) and activating backtrace collection (\mintinline{sh}{OCAMLRUNPARAM=b} or \mintinline{ocaml}{Printexc.record_backtrace b}) causes complexity of jumping to an handler to be $O(n)$ in terms of number of stack frames
\only<5>{
\begin{itemize}
\item Use \funcname{raise\_notrace} to inhibit the backtrace collection
\end{itemize}
\bigskip
\listocaml[firstline=205,lastline=210]{../ocaml/otherlibs/dynlink/dynlink\_compilerlibs/misc.ml}{otherlibs/dynlink/dynlink\_compilerlibs/misc.ml:205}
}
\end{enumerate}
\end{frame}



%
% Exception handlers
%
\subsection{Exception handlers}
\frameSubsection{pictures/Agents.png}{} %{- Deploy the sentinels. Immediately.}

% https://v2.ocaml.org/manual/expr.html#pattern-matching
\begin{frame}[fragile]{Syntax}{BNF\footnotemark}
\begin{grammar}
<expr> ::= <value-path>
  \alt \dots
	\alt `match' <expr> `with' <pattern-matching>
	\alt `try' <expr> `with' <pattern-matching>
  \alt \dots

<pattern-matching> ::= [ `|' ] <pattern> [`when' <expr>] `->' <expr> ( `|' <pattern> [`when' <expr>] `->' <expr> )

<pattern> ::= <value-name>
	\alt \dots
	\alt `exception' <pattern>
	\alt \dots
\end{grammar}
\footnotetext[1]{https://v2.ocaml.org/manual/expr.html\#pattern-matching}
\end{frame}

\begin{frame}{Example of exception handler}{Introduction}
\begin{columns}[c]
\begin{column}{0.6\textwidth}
\begin{itemize}
\item To catch exceptions, an exception handler is needed
\item Let's see the machinery involved in installing (and removing!) one
\end{itemize}
\end{column}
\begin{column}{0.4\textwidth}
\listocaml{../src/trap/trap.ml}{trap/trap.ml}
\end{column}
\end{columns}
\end{frame}

\begin{frame}{Example of exception handler}{Source code}
\begin{columns}[c]
\begin{column}{0.6\textwidth}
\begin{itemize}
\item A custom exception type (\typename{ExnA})
\item \funcname{main} is called by \funcname{camlTrap\_\_entry} with \funcarg{42} as argument
\item \funcname{main} installs an exception handler for \typename{ExnA}
\item \funcname{main} returns its argument incremented by one
\item \funcname{maybe\_raise} raises \typename{ExnA} if the \localname{Should.raise} returns \funcarg{true}
\item Otherwise \funcname{maybe\_raise} returns its argument incremented by one
\end{itemize}
\end{column}
\begin{column}{0.4\textwidth}
\listocaml{../src/trap/trap.ml}{trap/trap.ml}
\end{column}
\end{columns}
\end{frame}

\begin{frame}{Example of exception handler}{CMM, linear and assembly}
\begin{columns}[c]
\begin{column}{0.6\textwidth}
\begin{itemize}
% FIXME Get a better definition
\item CMM is the lowest structured representation of code
\item Linear is below CMM and common to all native codes
\item The assembly emitted from linear is architecture-specific (amd64 target for x86-64 CPU)
\end{itemize}
\end{column}
\begin{column}{0.4\textwidth}
\listocaml[firstline=8,lastline=11]{../src/trap/trap.ml}{trap/trap.ml}
\end{column}
\end{columns}
\end{frame}

\begin{frame}{Example of exception handler}{Exception handler's code}
\begin{columns}[c]
\begin{column}{0.5\textwidth}
\begin{itemize}
\item<1-> In case of exception in the body of the \mintinline{ocaml}{try ... with}, execution flow must be redirected to the handler code (where the pattern matching happens)
\item<1-> To this end, the \mintinline{ocaml}{try ... with} installs a trap on the stack
\begin{itemize}
\item<1-> Installs the trap before executing the body (at the \mintinline{ocaml}{try})
\item<1-> Removes the trap after succesfull execution of the body (at the \mintinline{ocaml}{with})
\end{itemize}
\item<2-> CMM doesn't let us witness those operations
\item<3-> But linear does!
\item<4-> Although we can't see the trap implementation (since it's architecture-specific)
\item<5-> We have to go deeper: let's look at the disassembly of \funcname{main}
\end{itemize}
\smallskip
\listocaml[firstline=8,lastline=11,highlightlines={9-10}]{../src/trap/trap.ml}{trap/trap.ml}
\end{column}
\begin{column}{0.5\textwidth}
\only<1-2>{\listcmm[highlightlines={2-7}]{../src/trap/camlTrap\_\_main\_273.cmm}{main CMM}}%
\only<3>{\listlinear[highlightlines={5-20}]{../src/trap/camlTrap\_\_main\_273.linear}{main linear}}%
\only<4>{\listocaml[firstline=255,lastline=268]{../ocaml/asmcomp/linearize.ml}{asmcomp/linearize.ml:255}}%
\only<5->{\listobjdump[firstline=2,lastline=36]{../src/trap/camlTrap\_\_main\_273.objdump}{main disassembly}}%
\end{column}
\end{columns}
\end{frame}

\begin{frame}{Example of exception handler}{Stack overflow handling}
\begin{columns}[c]
\begin{column}{0.5\textwidth}
\onslide*<1>{
\begin{itemize}
\item Let's ignore the stack overflow handling
\item OCaml stack size is limited (always on a fiber now!)
\item Functions that need stack space must check that creating the stack frame wouldn't cause a stack-overflow
\item If the frame would overflow, the stack (fiber) has to be reallocated
\end{itemize}
\bigskip
\listocaml[firstline=8,lastline=11]{../src/trap/trap.ml}{trap/trap.ml}
}
\onslide*<2>{
\listocaml[firstline=906,lastline=915]{../ocaml/asmcomp/amd64/emit.mlp}{asmcomp/amd64/emit.mlp:906}
\listocaml[firstline=921,lastline=931]{../ocaml/asmcomp/amd64/emit.mlp}{asmcomp/amd64/emit.mlp:921}
}
\end{column}
\begin{column}{0.5\textwidth}
\listobjdump[firstline=2,lastline=36,highlightlines={2-4,33-36}]{../src/trap/camlTrap\_\_main\_273.objdump}{main disassembly}
\end{column}
\end{columns}
\end{frame}


%
% Installing a trap
%
\subsection{Installing a trap}
\frameSubsection{pictures/Mouse.png}{- It's a trap, get out! - Oh, no!}


\begin{frame}{Example of exception handler}{Frame setup}
\begin{columns}
\begin{column}{0.5\textwidth}
\onslide*<1-2>{
\begin{itemize}
\item<1-> \funcname{entry} called \funcname{main} with \mintinline{gas}{call camlTrap__main_273}: the return address into \funcname{entry} is pushed on the stack, and execution jumps to \funcname{main}
\item<2-> Some stack space is reserved for storing \funcname{main}'s argument
\end{itemize}
\bigskip
}%
\onslide*<3>{
\listobjdump[firstline=5,lastline=29,highlightlines={5-6}]{../src/trap/camlTrap\_\_main\_273.objdump}{main disassembly}
\medskip
}%
\listocaml[firstline=8,lastline=11,highlightlines=8]{../src/trap/trap.ml}{trap/trap.ml}
\end{column}
\begin{column}{0.5\textwidth}
\centering
\onslide*<1>{\providecommand\step{1}\begin{tikzpicture}[font=\sffamily\tiny]
\input{../src/trap/gdb.out}

\newcommand\trapEnd{\xinteval{\trapBegin - 2 * \wordSize}}
\newcommand\regRspStepOne{\xinteval{\frameMainBegin - 1 * \wordSize}}
\newcommand\regRspStepTwo{\xinteval{\frameMainBegin - 2 * \wordSize}}
\newcommand\regRspStepThree{\xinteval{\frameMainBegin - 3 * \wordSize}}
\newcommand\regRspStepFour{\xinteval{\frameMainBegin - 4 * \wordSize}}


\begin{stackDiagram}[yFactor=0.4,showAddress=true]{\frameEntryBegin}{\frameMaybeEnd}
\callStack

\begin{funFrame}[name=entry,retaddr=\frameEntryRetaddr,color=GreenYellow]{\frameEntryBegin}{\frameEntryEnd}
\end{funFrame}

\ifthenelse{\step=1}
{\begin{funFrame}[color=YellowGreen]{\frameMainBegin}{\regRspStepOne}}
{\begin{funFrame}[name=main,color=YellowGreen]{\frameMainBegin}{\frameMainEnd}}
\funReturnAddr{\frameMainRetaddr}
\ifthenelse{\step=1}
{}{\frameLocal{1}{\frameMainLocalOne}}
\end{funFrame}


\ifthenelse{\step=3 \or \step=4}
{\frameLocalAt{\regRspStepThree}{\trapHandler}}{}
\ifthenelse{\step=4}
{\frameLocalAt[hex=true]{\regRspStepFour}{\trapNext}}{}
\ifthenelse{\step>4}
{\trapFrame[name=trap,handler=\trapHandler,next=\trapNext,color=WildStrawberry]{\trapBegin}}{}

\ifthenelse{\step=6}{
\begin{funFrame}[name=maybe,color=Green]{\frameMaybeBegin}{\frameMaybeEnd}
\funReturnAddr{\frameMaybeRetaddr}
\frameLocal{1}{\frameMaybeLocalOne}
\end{funFrame}
}{}


\ifthenelse{\step=1}{
\showRegister[color=black!25]{RSP}{\frameMainBegin}
\showRegister{RSP}{\regRspStepOne}
}{}
\ifthenelse{\step=2}{
\showRegister[color=black!25]{RSP}{\regRspStepOne}
\showRegister{RSP}{\regRspStepTwo}
}{}
\ifthenelse{\step=3}{
\showRegister[color=black!25]{RSP}{\regRspStepTwo}
\showRegister{RSP}{\regRspStepThree}
}{}
\ifthenelse{\step>3 \and \step<6}{
\showRegister[color=black!25]{RSP}{\regRspStepThree}
\showRegister{RSP}{\regRspStepFour}
}{}
\ifthenelse{\step=6}{
\showRegister[color=black!25]{RSP}{\regRspStepFour}
\showRegister{RSP}{\frameMaybeEnd}
}{}


\coordinate (bottom) at ($(stackBL) + (0,-0.5)$);

\ifthenelse{\step=4}
{\domainState[pos=(bottom)]{\exnHandlerBeforeTrap}}{}

\ifthenelse{\step>4}{
\domainState[pos=(bottom)]{\exnHandlerAfterTrap}
\coordAtRight{trap}{\exnHandlerAfterTrap}
\draw [->,>=latex] (exnHandlerRight.east) to [out=0,in=0] (trap.east);
}
{}

\end{stackDiagram}
\end{tikzpicture}
}%
\onslide*<2->{\providecommand\step{2}\begin{tikzpicture}[font=\sffamily\tiny]
\input{../src/trap/gdb.out}

\newcommand\trapEnd{\xinteval{\trapBegin - 2 * \wordSize}}
\newcommand\regRspStepOne{\xinteval{\frameMainBegin - 1 * \wordSize}}
\newcommand\regRspStepTwo{\xinteval{\frameMainBegin - 2 * \wordSize}}
\newcommand\regRspStepThree{\xinteval{\frameMainBegin - 3 * \wordSize}}
\newcommand\regRspStepFour{\xinteval{\frameMainBegin - 4 * \wordSize}}


\begin{stackDiagram}[yFactor=0.4,showAddress=true]{\frameEntryBegin}{\frameMaybeEnd}
\callStack

\begin{funFrame}[name=entry,retaddr=\frameEntryRetaddr,color=GreenYellow]{\frameEntryBegin}{\frameEntryEnd}
\end{funFrame}

\ifthenelse{\step=1}
{\begin{funFrame}[color=YellowGreen]{\frameMainBegin}{\regRspStepOne}}
{\begin{funFrame}[name=main,color=YellowGreen]{\frameMainBegin}{\frameMainEnd}}
\funReturnAddr{\frameMainRetaddr}
\ifthenelse{\step=1}
{}{\frameLocal{1}{\frameMainLocalOne}}
\end{funFrame}


\ifthenelse{\step=3 \or \step=4}
{\frameLocalAt{\regRspStepThree}{\trapHandler}}{}
\ifthenelse{\step=4}
{\frameLocalAt[hex=true]{\regRspStepFour}{\trapNext}}{}
\ifthenelse{\step>4}
{\trapFrame[name=trap,handler=\trapHandler,next=\trapNext,color=WildStrawberry]{\trapBegin}}{}

\ifthenelse{\step=6}{
\begin{funFrame}[name=maybe,color=Green]{\frameMaybeBegin}{\frameMaybeEnd}
\funReturnAddr{\frameMaybeRetaddr}
\frameLocal{1}{\frameMaybeLocalOne}
\end{funFrame}
}{}


\ifthenelse{\step=1}{
\showRegister[color=black!25]{RSP}{\frameMainBegin}
\showRegister{RSP}{\regRspStepOne}
}{}
\ifthenelse{\step=2}{
\showRegister[color=black!25]{RSP}{\regRspStepOne}
\showRegister{RSP}{\regRspStepTwo}
}{}
\ifthenelse{\step=3}{
\showRegister[color=black!25]{RSP}{\regRspStepTwo}
\showRegister{RSP}{\regRspStepThree}
}{}
\ifthenelse{\step>3 \and \step<6}{
\showRegister[color=black!25]{RSP}{\regRspStepThree}
\showRegister{RSP}{\regRspStepFour}
}{}
\ifthenelse{\step=6}{
\showRegister[color=black!25]{RSP}{\regRspStepFour}
\showRegister{RSP}{\frameMaybeEnd}
}{}


\coordinate (bottom) at ($(stackBL) + (0,-0.5)$);

\ifthenelse{\step=4}
{\domainState[pos=(bottom)]{\exnHandlerBeforeTrap}}{}

\ifthenelse{\step>4}{
\domainState[pos=(bottom)]{\exnHandlerAfterTrap}
\coordAtRight{trap}{\exnHandlerAfterTrap}
\draw [->,>=latex] (exnHandlerRight.east) to [out=0,in=0] (trap.east);
}
{}

\end{stackDiagram}
\end{tikzpicture}
}%
\end{column}
\end{columns}
\end{frame}

\begin{frame}{Example of exception handler}{Entering the try...with}
\begin{columns}
\begin{column}{0.5\textwidth}
\begin{itemize}
\item Now reaching the expression protected by the exception handler
\begin{itemize}
\item Before executing the call to \funcname{maybe\_raise}, the \mintinline{ocaml}{try ... with} install a trap on the stack: \funcname{push trap}
\item If \funcname{maybe\_raise} completes without error, the trap is removed: \funcname{pop trap}
\end{itemize}
\end{itemize}
\bigskip
\listocaml[firstline=8,lastline=11,highlightlines=9]{../src/trap/trap.ml}{trap/trap.ml}
\end{column}
\begin{column}{0.5\textwidth}
\only<1>{\listcmm[highlightlines=2]{../src/trap/camlTrap\_\_main\_273.cmm}{main CMM}}%
\only<2>{\listlinear[highlightlines=5]{../src/trap/camlTrap\_\_main\_273.linear}{main linear}}%
\only<3>{\listocaml[firstline=837,lastline=850]{../ocaml/asmcomp/amd64/emit.mlp}{asmcomp/amd64/emit.mlp:837}}%
\end{column}
\end{columns}
\end{frame}

\begin{frame}{Constructing the trap}{Handler's address}
\begin{columns}
\begin{column}{0.5\textwidth}
\onslide*<1-2>{
\begin{itemize}
\item<1-> The trap is a two-words object constructed on the stack right after the \funcname{main} frame
\item<1-> It redirects the execution flow to the handler in case of an exception
\item<2-> The address of the handler's code is pushed onto the stack
\item<2-> The handler (pattern matching) code is part of the \funcname{main} function code
\end{itemize}
}
\onslide*<3>{\listocaml[firstline=837,lastline=850,highlightlines={838-846}]{../ocaml/asmcomp/amd64/emit.mlp}{asmcomp/amd64/emit.mlp:837}}
\onslide*<4>{\mintobjdump[firstline=5,lastline=29,highlightlines={7-8,16-27}]{../src/trap/camlTrap\_\_main\_273.objdump}}
\medskip
\listocaml[firstline=8,lastline=11,highlightlines=9]{../src/trap/trap.ml}{trap/trap.ml}
\end{column}
\begin{column}{0.5\textwidth}
\centering
\providecommand\step{3}\begin{tikzpicture}[font=\sffamily\tiny]
\input{../src/trap/gdb.out}

\newcommand\trapEnd{\xinteval{\trapBegin - 2 * \wordSize}}
\newcommand\regRspStepOne{\xinteval{\frameMainBegin - 1 * \wordSize}}
\newcommand\regRspStepTwo{\xinteval{\frameMainBegin - 2 * \wordSize}}
\newcommand\regRspStepThree{\xinteval{\frameMainBegin - 3 * \wordSize}}
\newcommand\regRspStepFour{\xinteval{\frameMainBegin - 4 * \wordSize}}


\begin{stackDiagram}[yFactor=0.4,showAddress=true]{\frameEntryBegin}{\frameMaybeEnd}
\callStack

\begin{funFrame}[name=entry,retaddr=\frameEntryRetaddr,color=GreenYellow]{\frameEntryBegin}{\frameEntryEnd}
\end{funFrame}

\ifthenelse{\step=1}
{\begin{funFrame}[color=YellowGreen]{\frameMainBegin}{\regRspStepOne}}
{\begin{funFrame}[name=main,color=YellowGreen]{\frameMainBegin}{\frameMainEnd}}
\funReturnAddr{\frameMainRetaddr}
\ifthenelse{\step=1}
{}{\frameLocal{1}{\frameMainLocalOne}}
\end{funFrame}


\ifthenelse{\step=3 \or \step=4}
{\frameLocalAt{\regRspStepThree}{\trapHandler}}{}
\ifthenelse{\step=4}
{\frameLocalAt[hex=true]{\regRspStepFour}{\trapNext}}{}
\ifthenelse{\step>4}
{\trapFrame[name=trap,handler=\trapHandler,next=\trapNext,color=WildStrawberry]{\trapBegin}}{}

\ifthenelse{\step=6}{
\begin{funFrame}[name=maybe,color=Green]{\frameMaybeBegin}{\frameMaybeEnd}
\funReturnAddr{\frameMaybeRetaddr}
\frameLocal{1}{\frameMaybeLocalOne}
\end{funFrame}
}{}


\ifthenelse{\step=1}{
\showRegister[color=black!25]{RSP}{\frameMainBegin}
\showRegister{RSP}{\regRspStepOne}
}{}
\ifthenelse{\step=2}{
\showRegister[color=black!25]{RSP}{\regRspStepOne}
\showRegister{RSP}{\regRspStepTwo}
}{}
\ifthenelse{\step=3}{
\showRegister[color=black!25]{RSP}{\regRspStepTwo}
\showRegister{RSP}{\regRspStepThree}
}{}
\ifthenelse{\step>3 \and \step<6}{
\showRegister[color=black!25]{RSP}{\regRspStepThree}
\showRegister{RSP}{\regRspStepFour}
}{}
\ifthenelse{\step=6}{
\showRegister[color=black!25]{RSP}{\regRspStepFour}
\showRegister{RSP}{\frameMaybeEnd}
}{}


\coordinate (bottom) at ($(stackBL) + (0,-0.5)$);

\ifthenelse{\step=4}
{\domainState[pos=(bottom)]{\exnHandlerBeforeTrap}}{}

\ifthenelse{\step>4}{
\domainState[pos=(bottom)]{\exnHandlerAfterTrap}
\coordAtRight{trap}{\exnHandlerAfterTrap}
\draw [->,>=latex] (exnHandlerRight.east) to [out=0,in=0] (trap.east);
}
{}

\end{stackDiagram}
\end{tikzpicture}

\end{column}
\end{columns}
\end{frame}

% TODO notes:
% - trap_{sp_off,barrier_off,barrier_block} -> bytecode
% - backtrace_{pos,active,buffer} -> part II
\begin{frame}{Introducing caml\_domain\_state}{First mention of the runtime!}
\begin{columns}
\begin{column}{0.5\textwidth}
\begin{listing}
\onslide*<1,3>{
\begin{itemize}
\item<1-> \typename{caml\_domain\_state} is a C structure part of the runtime
\item<1-> It holds information useful to the runtime to manage a domain (each domain has its \typename{caml\_domain\_state})
\item<3-> \localname{exn\_handler} is a pointer to the \emph{last} installed \typename{trap}
\end{itemize}
}
\onslide*<2>{\mintc[lastline=32]{src/ocaml/domain\_state.h}}
\end{listing}
\end{column}
\begin{column}{0.5\textwidth}
\begin{listing}
\onslide*<2>{\mintc[firstline=32]{src/ocaml/domain\_state.h}}
\onslide*<3->{\listc[highlightlines={7}]{src/ocaml/domain\_state\_abbr.h}{runtime/caml/domain\_state.h}}
\end{listing}
\end{column}
\end{columns}
\end{frame}

\begin{frame}{Constructing the trap}{Saving previous trap address}
\begin{columns}
\begin{column}{0.5\textwidth}
\onslide*<1>{
\begin{itemize}
% TODO grey color
\item The address of the handler's code is pushed onto the stack
% TODO /grey color
\item The value of \localname{exn\_handler} from \typename{caml\_domain\_state} is pushed onto the stack
\item \localname{r14} register points to the current domain's \typename{caml\_domain\_state}
\end{itemize}
}
\onslide*<2>{\listocaml[firstline=837,lastline=850,highlightlines={847-848}]{../ocaml/asmcomp/amd64/emit.mlp}{asmcomp/amd64/emit.mlp:838}}
\onslide*<3>{\mintobjdump[firstline=5,lastline=29,highlightlines=9]{../src/trap/camlTrap\_\_main\_273.objdump}}
\medskip
\listocaml[firstline=8,lastline=11,highlightlines=9]{../src/trap/trap.ml}{trap/trap.ml}
\end{column}
\begin{column}{0.5\textwidth}
\centering
\providecommand\step{4}\begin{tikzpicture}[font=\sffamily\tiny]
\input{../src/trap/gdb.out}

\newcommand\trapEnd{\xinteval{\trapBegin - 2 * \wordSize}}
\newcommand\regRspStepOne{\xinteval{\frameMainBegin - 1 * \wordSize}}
\newcommand\regRspStepTwo{\xinteval{\frameMainBegin - 2 * \wordSize}}
\newcommand\regRspStepThree{\xinteval{\frameMainBegin - 3 * \wordSize}}
\newcommand\regRspStepFour{\xinteval{\frameMainBegin - 4 * \wordSize}}


\begin{stackDiagram}[yFactor=0.4,showAddress=true]{\frameEntryBegin}{\frameMaybeEnd}
\callStack

\begin{funFrame}[name=entry,retaddr=\frameEntryRetaddr,color=GreenYellow]{\frameEntryBegin}{\frameEntryEnd}
\end{funFrame}

\ifthenelse{\step=1}
{\begin{funFrame}[color=YellowGreen]{\frameMainBegin}{\regRspStepOne}}
{\begin{funFrame}[name=main,color=YellowGreen]{\frameMainBegin}{\frameMainEnd}}
\funReturnAddr{\frameMainRetaddr}
\ifthenelse{\step=1}
{}{\frameLocal{1}{\frameMainLocalOne}}
\end{funFrame}


\ifthenelse{\step=3 \or \step=4}
{\frameLocalAt{\regRspStepThree}{\trapHandler}}{}
\ifthenelse{\step=4}
{\frameLocalAt[hex=true]{\regRspStepFour}{\trapNext}}{}
\ifthenelse{\step>4}
{\trapFrame[name=trap,handler=\trapHandler,next=\trapNext,color=WildStrawberry]{\trapBegin}}{}

\ifthenelse{\step=6}{
\begin{funFrame}[name=maybe,color=Green]{\frameMaybeBegin}{\frameMaybeEnd}
\funReturnAddr{\frameMaybeRetaddr}
\frameLocal{1}{\frameMaybeLocalOne}
\end{funFrame}
}{}


\ifthenelse{\step=1}{
\showRegister[color=black!25]{RSP}{\frameMainBegin}
\showRegister{RSP}{\regRspStepOne}
}{}
\ifthenelse{\step=2}{
\showRegister[color=black!25]{RSP}{\regRspStepOne}
\showRegister{RSP}{\regRspStepTwo}
}{}
\ifthenelse{\step=3}{
\showRegister[color=black!25]{RSP}{\regRspStepTwo}
\showRegister{RSP}{\regRspStepThree}
}{}
\ifthenelse{\step>3 \and \step<6}{
\showRegister[color=black!25]{RSP}{\regRspStepThree}
\showRegister{RSP}{\regRspStepFour}
}{}
\ifthenelse{\step=6}{
\showRegister[color=black!25]{RSP}{\regRspStepFour}
\showRegister{RSP}{\frameMaybeEnd}
}{}


\coordinate (bottom) at ($(stackBL) + (0,-0.5)$);

\ifthenelse{\step=4}
{\domainState[pos=(bottom)]{\exnHandlerBeforeTrap}}{}

\ifthenelse{\step>4}{
\domainState[pos=(bottom)]{\exnHandlerAfterTrap}
\coordAtRight{trap}{\exnHandlerAfterTrap}
\draw [->,>=latex] (exnHandlerRight.east) to [out=0,in=0] (trap.east);
}
{}

\end{stackDiagram}
\end{tikzpicture}

\end{column}
\end{columns}
\end{frame}

\begin{frame}{Constructing the trap}{Updating exn\_handler}
\begin{columns}
\begin{column}{0.5\textwidth}
\onslide*<1>{
\begin{itemize}
% TODO grey color
\item The address of the handler's code is pushed onto the stack
\item The value of \localname{exn\_handler} from \typename{caml\_domain\_state} is pushed onto the stack
% TODO /grey color
\item \localname{domain\_state->exn\_handler} has to point to the last installed trap
\item Its value is updated to the just-constructed trap's address
\end{itemize}
}
\onslide*<2>{\listocaml[firstline=837,lastline=850,highlightlines=849]{../ocaml/asmcomp/amd64/emit.mlp}{asmcomp/amd64/emit.mlp:849}}
\onslide*<3>{\mintobjdump[firstline=5,lastline=29,highlightlines=10]{../src/trap/camlTrap\_\_main\_273.objdump}}
\medskip
\listocaml[firstline=8,lastline=11,highlightlines=10]{../src/trap/trap.ml}{trap/trap.ml}
\end{column}
\begin{column}{0.5\textwidth}
\centering
\providecommand\step{5}\begin{tikzpicture}[font=\sffamily\tiny]
\input{../src/trap/gdb.out}

\newcommand\trapEnd{\xinteval{\trapBegin - 2 * \wordSize}}
\newcommand\regRspStepOne{\xinteval{\frameMainBegin - 1 * \wordSize}}
\newcommand\regRspStepTwo{\xinteval{\frameMainBegin - 2 * \wordSize}}
\newcommand\regRspStepThree{\xinteval{\frameMainBegin - 3 * \wordSize}}
\newcommand\regRspStepFour{\xinteval{\frameMainBegin - 4 * \wordSize}}


\begin{stackDiagram}[yFactor=0.4,showAddress=true]{\frameEntryBegin}{\frameMaybeEnd}
\callStack

\begin{funFrame}[name=entry,retaddr=\frameEntryRetaddr,color=GreenYellow]{\frameEntryBegin}{\frameEntryEnd}
\end{funFrame}

\ifthenelse{\step=1}
{\begin{funFrame}[color=YellowGreen]{\frameMainBegin}{\regRspStepOne}}
{\begin{funFrame}[name=main,color=YellowGreen]{\frameMainBegin}{\frameMainEnd}}
\funReturnAddr{\frameMainRetaddr}
\ifthenelse{\step=1}
{}{\frameLocal{1}{\frameMainLocalOne}}
\end{funFrame}


\ifthenelse{\step=3 \or \step=4}
{\frameLocalAt{\regRspStepThree}{\trapHandler}}{}
\ifthenelse{\step=4}
{\frameLocalAt[hex=true]{\regRspStepFour}{\trapNext}}{}
\ifthenelse{\step>4}
{\trapFrame[name=trap,handler=\trapHandler,next=\trapNext,color=WildStrawberry]{\trapBegin}}{}

\ifthenelse{\step=6}{
\begin{funFrame}[name=maybe,color=Green]{\frameMaybeBegin}{\frameMaybeEnd}
\funReturnAddr{\frameMaybeRetaddr}
\frameLocal{1}{\frameMaybeLocalOne}
\end{funFrame}
}{}


\ifthenelse{\step=1}{
\showRegister[color=black!25]{RSP}{\frameMainBegin}
\showRegister{RSP}{\regRspStepOne}
}{}
\ifthenelse{\step=2}{
\showRegister[color=black!25]{RSP}{\regRspStepOne}
\showRegister{RSP}{\regRspStepTwo}
}{}
\ifthenelse{\step=3}{
\showRegister[color=black!25]{RSP}{\regRspStepTwo}
\showRegister{RSP}{\regRspStepThree}
}{}
\ifthenelse{\step>3 \and \step<6}{
\showRegister[color=black!25]{RSP}{\regRspStepThree}
\showRegister{RSP}{\regRspStepFour}
}{}
\ifthenelse{\step=6}{
\showRegister[color=black!25]{RSP}{\regRspStepFour}
\showRegister{RSP}{\frameMaybeEnd}
}{}


\coordinate (bottom) at ($(stackBL) + (0,-0.5)$);

\ifthenelse{\step=4}
{\domainState[pos=(bottom)]{\exnHandlerBeforeTrap}}{}

\ifthenelse{\step>4}{
\domainState[pos=(bottom)]{\exnHandlerAfterTrap}
\coordAtRight{trap}{\exnHandlerAfterTrap}
\draw [->,>=latex] (exnHandlerRight.east) to [out=0,in=0] (trap.east);
}
{}

\end{stackDiagram}
\end{tikzpicture}

\end{column}
\end{columns}
\end{frame}


\begin{frame}{OCaml exception trap}{More details to come later!}
\begin{itemize}
\item On amd64, an exception trap could be represented with this C structure
\end{itemize}
\bigskip
\centering{\mintc[firstline=1,lastline=6]{src/ocaml/trap\_partial.c}} % FIXME center
\end{frame}


%
% Takeaway #2
%
\subsection*{Takeaway \#2}
\frameSubsectionTakeaway{}
\againframe<2>{takeaway}


\begin{frame}{Executing the body}{Protected by the try...with}
\begin{columns}
\begin{column}{0.5\textwidth}
\onslide*<1>{
\begin{itemize}
\item Now executing the expression protected by the \mintinline{ocaml}{try ... with}: a call to \funcname{maybe\_raise}
\begin{itemize}
\item Which may complete without exception: the trap has to be removed, and the handler code is not executed
\item Which may raise \typename{ExnA}: execution flow is redirected to the handler code thanks to the trap
\end{itemize}
\end{itemize}
}
\only<2>{\mintcmm[highlightlines=2]{../src/trap/camlTrap\_\_main\_273.cmm}}%
\only<3>{\mintlinear[firstline=2,highlightlines=7]{../src/trap/camlTrap\_\_main\_273.linear}}%
\only<4>{\mintobjdump[firstline=5,lastline=29,highlightlines=11]{../src/trap/camlTrap\_\_main\_273.objdump}}%
\listocaml[firstline=3,lastline=11,highlightlines={3-6,9}]{../src/trap/trap.ml}{trap/trap.ml}
\end{column}
\begin{column}{0.5\textwidth}
\centering
\providecommand\step{6}\begin{tikzpicture}[font=\sffamily\tiny]
\input{../src/trap/gdb.out}

\newcommand\trapEnd{\xinteval{\trapBegin - 2 * \wordSize}}
\newcommand\regRspStepOne{\xinteval{\frameMainBegin - 1 * \wordSize}}
\newcommand\regRspStepTwo{\xinteval{\frameMainBegin - 2 * \wordSize}}
\newcommand\regRspStepThree{\xinteval{\frameMainBegin - 3 * \wordSize}}
\newcommand\regRspStepFour{\xinteval{\frameMainBegin - 4 * \wordSize}}


\begin{stackDiagram}[yFactor=0.4,showAddress=true]{\frameEntryBegin}{\frameMaybeEnd}
\callStack

\begin{funFrame}[name=entry,retaddr=\frameEntryRetaddr,color=GreenYellow]{\frameEntryBegin}{\frameEntryEnd}
\end{funFrame}

\ifthenelse{\step=1}
{\begin{funFrame}[color=YellowGreen]{\frameMainBegin}{\regRspStepOne}}
{\begin{funFrame}[name=main,color=YellowGreen]{\frameMainBegin}{\frameMainEnd}}
\funReturnAddr{\frameMainRetaddr}
\ifthenelse{\step=1}
{}{\frameLocal{1}{\frameMainLocalOne}}
\end{funFrame}


\ifthenelse{\step=3 \or \step=4}
{\frameLocalAt{\regRspStepThree}{\trapHandler}}{}
\ifthenelse{\step=4}
{\frameLocalAt[hex=true]{\regRspStepFour}{\trapNext}}{}
\ifthenelse{\step>4}
{\trapFrame[name=trap,handler=\trapHandler,next=\trapNext,color=WildStrawberry]{\trapBegin}}{}

\ifthenelse{\step=6}{
\begin{funFrame}[name=maybe,color=Green]{\frameMaybeBegin}{\frameMaybeEnd}
\funReturnAddr{\frameMaybeRetaddr}
\frameLocal{1}{\frameMaybeLocalOne}
\end{funFrame}
}{}


\ifthenelse{\step=1}{
\showRegister[color=black!25]{RSP}{\frameMainBegin}
\showRegister{RSP}{\regRspStepOne}
}{}
\ifthenelse{\step=2}{
\showRegister[color=black!25]{RSP}{\regRspStepOne}
\showRegister{RSP}{\regRspStepTwo}
}{}
\ifthenelse{\step=3}{
\showRegister[color=black!25]{RSP}{\regRspStepTwo}
\showRegister{RSP}{\regRspStepThree}
}{}
\ifthenelse{\step>3 \and \step<6}{
\showRegister[color=black!25]{RSP}{\regRspStepThree}
\showRegister{RSP}{\regRspStepFour}
}{}
\ifthenelse{\step=6}{
\showRegister[color=black!25]{RSP}{\regRspStepFour}
\showRegister{RSP}{\frameMaybeEnd}
}{}


\coordinate (bottom) at ($(stackBL) + (0,-0.5)$);

\ifthenelse{\step=4}
{\domainState[pos=(bottom)]{\exnHandlerBeforeTrap}}{}

\ifthenelse{\step>4}{
\domainState[pos=(bottom)]{\exnHandlerAfterTrap}
\coordAtRight{trap}{\exnHandlerAfterTrap}
\draw [->,>=latex] (exnHandlerRight.east) to [out=0,in=0] (trap.east);
}
{}

\end{stackDiagram}
\end{tikzpicture}

\end{column}
\end{columns}
\end{frame}


%
% Removing a trap
%
\subsection{Removing a trap}
\frameSubsection{pictures/bugged.png}{- I think you're bugged.}


\begin{frame}{maybe\_raise terminates normally}{}
\begin{columns}
\begin{column}{0.5\textwidth}
\onslide*<1,3>{
\begin{itemize}
\item<1-> \funcname{maybe\_raise} terminates without exception
\item<1-> Computes its return value and sets it into \regname{RAX}
\item<3-> Increments \regname{RSP} to remove its own stack frame
\end{itemize}
}
\onslide*<2>{
\mintobjdump[firstline=5,lastline=24,highlightlines={21-22}]{../src/noraise/camlNoraise\_\_maybe\_raise\_269.objdump}
}
\onslide*<4>{
\mintobjdump[firstline=5,lastline=24,highlightlines={23-24}]{../src/noraise/camlNoraise\_\_maybe\_raise\_269.objdump}
}
\medskip
\only<1>{\listocaml[firstline=3,lastline=11,highlightlines={6,9}]{../src/noraise/noraise.ml}{noraise/noraise.ml}}%
\onslide*<2->{\listocaml[firstline=3,lastline=6,highlightlines=6]{../src/noraise/noraise.ml}{noraise/noraise.ml}}%
\end{column}
\begin{column}{0.5\textwidth}
\centering
\onslide*<1-2>{\providecommand\step{1}\begin{tikzpicture}[font=\sffamily\tiny]
\input{../src/noraise/gdb.out}

\newcommand\trapEnd{\xinteval{\trapBegin - 2 * \wordSize}}
\newcommand\trapHandlerAddr{\xinteval{\trapBegin - 1 * \wordSize}}


\begin{stackDiagram}[yFactor=0.4,showAddress=true]{\frameEntryBegin}{\frameMaybeEnd}
\callStack

\begin{funFrame}[name=entry,retaddr=\frameEntryRetaddr,color=GreenYellow]{\frameEntryBegin}{\frameEntryEnd}
\end{funFrame}

\begin{funFrame}[name=main,color=YellowGreen]{\frameMainBegin}{\frameMainEnd}
\funReturnAddr{\frameMainRetaddr}
\frameLocal{1}{\frameMainLocalOne}
\end{funFrame}


\ifthenelse{\step=4}{
\trapFrame[name=trap,handler=\trapHandler,next=\trapNext,color=WildStrawberry,opacity=0.25]{\trapBegin}
}{
\trapFrame[name=trap,handler=\trapHandler,next=\trapNext,color=WildStrawberry]{\trapBegin}
}


\ifthenelse{\step=1}
{\begin{funFrame}[name=maybe,color=Green]{\frameMaybeBegin}{\frameMaybeEnd}}
{\begin{funFrame}[name=maybe,color=Green,opacity=0.25]{\frameMaybeBegin}{\frameMaybeEnd}}
\funReturnAddr{\frameMaybeRetaddr}
\frameLocal{1}{\frameMaybeLocalOne}
\end{funFrame}


\ifthenelse{\step=1}{
\showRegister{RSP}{\frameMaybeEnd}
}{}
\ifthenelse{\step=2}{
\showRegister{RSP}{\trapEnd}
\showRegister[color=black!25]{RSP}{\frameMaybeEnd}
}{}
\ifthenelse{\step=3}{
\showRegister{RSP}{\trapHandlerAddr}
\showRegister[color=black!25]{RSP}{\trapEnd}
}{}
\ifthenelse{\step=4}{
\showRegister{RSP}{\frameMainEnd}
\showRegister[color=black!25]{RSP}{\trapHandlerAddr}
}{}


\coordinate (bottom) at ($(stackBL) + (0,-0.5)$);

\ifthenelse{\step<3}{
\domainState[pos=(bottom)]{\exnHandlerAfterTrap}
\coordAtRight{trap}{\exnHandlerAfterTrap}
\draw [->,>=latex] (exnHandlerRight.east) to [out=0,in=0] (trap.east);
}{
\domainState[pos=(bottom)]{\exnHandlerBeforeTrap}
}

\end{stackDiagram}
\end{tikzpicture}
}
\onslide*<3->{\providecommand\step{2}\begin{tikzpicture}[font=\sffamily\tiny]
\input{../src/noraise/gdb.out}

\newcommand\trapEnd{\xinteval{\trapBegin - 2 * \wordSize}}
\newcommand\trapHandlerAddr{\xinteval{\trapBegin - 1 * \wordSize}}


\begin{stackDiagram}[yFactor=0.4,showAddress=true]{\frameEntryBegin}{\frameMaybeEnd}
\callStack

\begin{funFrame}[name=entry,retaddr=\frameEntryRetaddr,color=GreenYellow]{\frameEntryBegin}{\frameEntryEnd}
\end{funFrame}

\begin{funFrame}[name=main,color=YellowGreen]{\frameMainBegin}{\frameMainEnd}
\funReturnAddr{\frameMainRetaddr}
\frameLocal{1}{\frameMainLocalOne}
\end{funFrame}


\ifthenelse{\step=4}{
\trapFrame[name=trap,handler=\trapHandler,next=\trapNext,color=WildStrawberry,opacity=0.25]{\trapBegin}
}{
\trapFrame[name=trap,handler=\trapHandler,next=\trapNext,color=WildStrawberry]{\trapBegin}
}


\ifthenelse{\step=1}
{\begin{funFrame}[name=maybe,color=Green]{\frameMaybeBegin}{\frameMaybeEnd}}
{\begin{funFrame}[name=maybe,color=Green,opacity=0.25]{\frameMaybeBegin}{\frameMaybeEnd}}
\funReturnAddr{\frameMaybeRetaddr}
\frameLocal{1}{\frameMaybeLocalOne}
\end{funFrame}


\ifthenelse{\step=1}{
\showRegister{RSP}{\frameMaybeEnd}
}{}
\ifthenelse{\step=2}{
\showRegister{RSP}{\trapEnd}
\showRegister[color=black!25]{RSP}{\frameMaybeEnd}
}{}
\ifthenelse{\step=3}{
\showRegister{RSP}{\trapHandlerAddr}
\showRegister[color=black!25]{RSP}{\trapEnd}
}{}
\ifthenelse{\step=4}{
\showRegister{RSP}{\frameMainEnd}
\showRegister[color=black!25]{RSP}{\trapHandlerAddr}
}{}


\coordinate (bottom) at ($(stackBL) + (0,-0.5)$);

\ifthenelse{\step<3}{
\domainState[pos=(bottom)]{\exnHandlerAfterTrap}
\coordAtRight{trap}{\exnHandlerAfterTrap}
\draw [->,>=latex] (exnHandlerRight.east) to [out=0,in=0] (trap.east);
}{
\domainState[pos=(bottom)]{\exnHandlerBeforeTrap}
}

\end{stackDiagram}
\end{tikzpicture}
}
\end{column}
\end{columns}
\end{frame}

\begin{frame}{main removes the lingering trap}{Restoring domain\_state->exn\_handler}
\begin{columns}
\begin{column}{0.5\textwidth}
\only<1>{
\begin{itemize}
\item \localname{domain\_state->exn\_handler} is restored to its previous value from the trap
\end{itemize}
}%
\only<2>{
\listocaml[firstline=851,lastline=856,highlightlines={852-853}]{../ocaml/asmcomp/amd64/emit.mlp}{asmcomp/amd64/emit.mlp:851}
\mintlinear[firstline=2,lastline=9,highlightlines=8]{../src/noraise/camlNoraise\_\_main\_273.linear}
}%
\only<3>{\mintobjdump[firstline=5,lastline=32,highlightlines=12]{../src/noraise/camlNoraise\_\_main\_273.objdump}}%
\medskip
\listocaml[firstline=8,lastline=11,highlightlines=9]{../src/noraise/noraise.ml}{noraise/noraise.ml}
\end{column}
\begin{column}{0.5\textwidth}
\centering
\providecommand\step{3}\begin{tikzpicture}[font=\sffamily\tiny]
\input{../src/noraise/gdb.out}

\newcommand\trapEnd{\xinteval{\trapBegin - 2 * \wordSize}}
\newcommand\trapHandlerAddr{\xinteval{\trapBegin - 1 * \wordSize}}


\begin{stackDiagram}[yFactor=0.4,showAddress=true]{\frameEntryBegin}{\frameMaybeEnd}
\callStack

\begin{funFrame}[name=entry,retaddr=\frameEntryRetaddr,color=GreenYellow]{\frameEntryBegin}{\frameEntryEnd}
\end{funFrame}

\begin{funFrame}[name=main,color=YellowGreen]{\frameMainBegin}{\frameMainEnd}
\funReturnAddr{\frameMainRetaddr}
\frameLocal{1}{\frameMainLocalOne}
\end{funFrame}


\ifthenelse{\step=4}{
\trapFrame[name=trap,handler=\trapHandler,next=\trapNext,color=WildStrawberry,opacity=0.25]{\trapBegin}
}{
\trapFrame[name=trap,handler=\trapHandler,next=\trapNext,color=WildStrawberry]{\trapBegin}
}


\ifthenelse{\step=1}
{\begin{funFrame}[name=maybe,color=Green]{\frameMaybeBegin}{\frameMaybeEnd}}
{\begin{funFrame}[name=maybe,color=Green,opacity=0.25]{\frameMaybeBegin}{\frameMaybeEnd}}
\funReturnAddr{\frameMaybeRetaddr}
\frameLocal{1}{\frameMaybeLocalOne}
\end{funFrame}


\ifthenelse{\step=1}{
\showRegister{RSP}{\frameMaybeEnd}
}{}
\ifthenelse{\step=2}{
\showRegister{RSP}{\trapEnd}
\showRegister[color=black!25]{RSP}{\frameMaybeEnd}
}{}
\ifthenelse{\step=3}{
\showRegister{RSP}{\trapHandlerAddr}
\showRegister[color=black!25]{RSP}{\trapEnd}
}{}
\ifthenelse{\step=4}{
\showRegister{RSP}{\frameMainEnd}
\showRegister[color=black!25]{RSP}{\trapHandlerAddr}
}{}


\coordinate (bottom) at ($(stackBL) + (0,-0.5)$);

\ifthenelse{\step<3}{
\domainState[pos=(bottom)]{\exnHandlerAfterTrap}
\coordAtRight{trap}{\exnHandlerAfterTrap}
\draw [->,>=latex] (exnHandlerRight.east) to [out=0,in=0] (trap.east);
}{
\domainState[pos=(bottom)]{\exnHandlerBeforeTrap}
}

\end{stackDiagram}
\end{tikzpicture}

\end{column}
\end{columns}
\end{frame}

\begin{frame}{main removes the lingering trap}{Discarding handler address}
\begin{columns}
\begin{column}{0.5\textwidth}
\only<1>{
\begin{itemize}
% TODO grey
\item \localname{domain\_state->exn\_handler} is restored to its previous value from the trap
% TODO /grey
\item Discards the exception handler code address
\end{itemize}
}%
\only<2>{
\listocaml[firstline=851,lastline=856,highlightlines={854-855}]{../ocaml/asmcomp/amd64/emit.mlp}{asmcomp/amd64/emit.mlp:851}
\mintlinear[firstline=2,lastline=9,highlightlines=8]{../src/noraise/camlNoraise\_\_main\_273.linear}
}%
\only<3>{\mintobjdump[firstline=5,lastline=32,highlightlines=13]{../src/noraise/camlNoraise\_\_main\_273.objdump}}%
\medskip
\listocaml[firstline=8,lastline=11,highlightlines=9]{../src/noraise/noraise.ml}{noraise/noraise.ml}
\end{column}
\begin{column}{0.5\textwidth}
\centering
\providecommand\step{4}\begin{tikzpicture}[font=\sffamily\tiny]
\input{../src/noraise/gdb.out}

\newcommand\trapEnd{\xinteval{\trapBegin - 2 * \wordSize}}
\newcommand\trapHandlerAddr{\xinteval{\trapBegin - 1 * \wordSize}}


\begin{stackDiagram}[yFactor=0.4,showAddress=true]{\frameEntryBegin}{\frameMaybeEnd}
\callStack

\begin{funFrame}[name=entry,retaddr=\frameEntryRetaddr,color=GreenYellow]{\frameEntryBegin}{\frameEntryEnd}
\end{funFrame}

\begin{funFrame}[name=main,color=YellowGreen]{\frameMainBegin}{\frameMainEnd}
\funReturnAddr{\frameMainRetaddr}
\frameLocal{1}{\frameMainLocalOne}
\end{funFrame}


\ifthenelse{\step=4}{
\trapFrame[name=trap,handler=\trapHandler,next=\trapNext,color=WildStrawberry,opacity=0.25]{\trapBegin}
}{
\trapFrame[name=trap,handler=\trapHandler,next=\trapNext,color=WildStrawberry]{\trapBegin}
}


\ifthenelse{\step=1}
{\begin{funFrame}[name=maybe,color=Green]{\frameMaybeBegin}{\frameMaybeEnd}}
{\begin{funFrame}[name=maybe,color=Green,opacity=0.25]{\frameMaybeBegin}{\frameMaybeEnd}}
\funReturnAddr{\frameMaybeRetaddr}
\frameLocal{1}{\frameMaybeLocalOne}
\end{funFrame}


\ifthenelse{\step=1}{
\showRegister{RSP}{\frameMaybeEnd}
}{}
\ifthenelse{\step=2}{
\showRegister{RSP}{\trapEnd}
\showRegister[color=black!25]{RSP}{\frameMaybeEnd}
}{}
\ifthenelse{\step=3}{
\showRegister{RSP}{\trapHandlerAddr}
\showRegister[color=black!25]{RSP}{\trapEnd}
}{}
\ifthenelse{\step=4}{
\showRegister{RSP}{\frameMainEnd}
\showRegister[color=black!25]{RSP}{\trapHandlerAddr}
}{}


\coordinate (bottom) at ($(stackBL) + (0,-0.5)$);

\ifthenelse{\step<3}{
\domainState[pos=(bottom)]{\exnHandlerAfterTrap}
\coordAtRight{trap}{\exnHandlerAfterTrap}
\draw [->,>=latex] (exnHandlerRight.east) to [out=0,in=0] (trap.east);
}{
\domainState[pos=(bottom)]{\exnHandlerBeforeTrap}
}

\end{stackDiagram}
\end{tikzpicture}

\end{column}
\end{columns}
\end{frame}

\begin{frame}{main continues}{As if the trap was never here}
\begin{columns}
\begin{column}{0.5\textwidth}
\only<1>{
\begin{itemize}
\item \funcname{main} jumps over the exception handler code and continues its execution
\end{itemize}
\medskip
\mintlinear[firstline=2,lastline=9,highlightlines=9]{../src/noraise/camlNoraise\_\_main\_273.linear}
}%
\only<2>{\mintobjdump[firstline=5,lastline=32,highlightlines={14,29-32}]{../src/noraise/camlNoraise\_\_main\_273.objdump}}%
\medskip
\listocaml[firstline=8,lastline=11,highlightlines=11]{../src/noraise/noraise.ml}{noraise/noraise.ml}
\end{column}
\begin{column}{0.5\textwidth}
\centering
\providecommand\step{4}\begin{tikzpicture}[font=\sffamily\tiny]
\input{../src/noraise/gdb.out}

\newcommand\trapEnd{\xinteval{\trapBegin - 2 * \wordSize}}
\newcommand\trapHandlerAddr{\xinteval{\trapBegin - 1 * \wordSize}}


\begin{stackDiagram}[yFactor=0.4,showAddress=true]{\frameEntryBegin}{\frameMaybeEnd}
\callStack

\begin{funFrame}[name=entry,retaddr=\frameEntryRetaddr,color=GreenYellow]{\frameEntryBegin}{\frameEntryEnd}
\end{funFrame}

\begin{funFrame}[name=main,color=YellowGreen]{\frameMainBegin}{\frameMainEnd}
\funReturnAddr{\frameMainRetaddr}
\frameLocal{1}{\frameMainLocalOne}
\end{funFrame}


\ifthenelse{\step=4}{
\trapFrame[name=trap,handler=\trapHandler,next=\trapNext,color=WildStrawberry,opacity=0.25]{\trapBegin}
}{
\trapFrame[name=trap,handler=\trapHandler,next=\trapNext,color=WildStrawberry]{\trapBegin}
}


\ifthenelse{\step=1}
{\begin{funFrame}[name=maybe,color=Green]{\frameMaybeBegin}{\frameMaybeEnd}}
{\begin{funFrame}[name=maybe,color=Green,opacity=0.25]{\frameMaybeBegin}{\frameMaybeEnd}}
\funReturnAddr{\frameMaybeRetaddr}
\frameLocal{1}{\frameMaybeLocalOne}
\end{funFrame}


\ifthenelse{\step=1}{
\showRegister{RSP}{\frameMaybeEnd}
}{}
\ifthenelse{\step=2}{
\showRegister{RSP}{\trapEnd}
\showRegister[color=black!25]{RSP}{\frameMaybeEnd}
}{}
\ifthenelse{\step=3}{
\showRegister{RSP}{\trapHandlerAddr}
\showRegister[color=black!25]{RSP}{\trapEnd}
}{}
\ifthenelse{\step=4}{
\showRegister{RSP}{\frameMainEnd}
\showRegister[color=black!25]{RSP}{\trapHandlerAddr}
}{}


\coordinate (bottom) at ($(stackBL) + (0,-0.5)$);

\ifthenelse{\step<3}{
\domainState[pos=(bottom)]{\exnHandlerAfterTrap}
\coordAtRight{trap}{\exnHandlerAfterTrap}
\draw [->,>=latex] (exnHandlerRight.east) to [out=0,in=0] (trap.east);
}{
\domainState[pos=(bottom)]{\exnHandlerBeforeTrap}
}

\end{stackDiagram}
\end{tikzpicture}

\end{column}
\end{columns}
\end{frame}


%
% Raise
%
\subsection{Raise}
\frameSubsection{pictures/Neo\_takeoff.png}{}

\begin{frame}{maybe\_raise raises an exception}{Loading the exception}
\begin{columns}
\begin{column}{0.5\textwidth}
\only<1>{
\begin{itemize}
\item \funcname{maybe\_raise} somehow loads the exception value into \regname{RAX}
\end{itemize}
}%
\only<2>{
\mintcmm[highlightlines=7]{../src/raise/camlRaise\_\_maybe\_raise\_269.cmm}
}%
\only<3>{
\mintlinear[highlightlines={12-13}]{../src/raise/camlRaise\_\_maybe\_raise\_269.linear}
}%
\only<4>{
\mintobjdump[firstline=5,lastline=24,highlightlines={14-15}]{../src/raise/camlRaise\_\_maybe\_raise\_269.objdump}
}%
\medskip
\listocaml[firstline=3,lastline=6,highlightlines=5]{../src/raise/raise.ml}{raise/raise.ml}
\end{column}
\begin{column}{0.5\textwidth}
\centering
\providecommand\step{1}\begin{tikzpicture}[font=\sffamily\tiny]
\input{../src/nested/gdb.out}

\begin{stackDiagram}[yFactor=0.35,showAddress=true]{\frameMainBegin}{\frameDefinitelyEnd}
\callStack

\begin{funFrame}[name=main,color=GreenYellow]{\frameMainBegin}{\frameMainEnd}
\funReturnAddr{\frameMainRetaddr}
\frameLocal{1}{\frameMainLocalOne}
\end{funFrame}


\ifthenelse{\step<7}
{\trapFrame[name=ExnC,handler=\trapExnCHandler,next=\trapExnCNext,color=WildStrawberry]{\trapExnCBegin}}
{\trapFrame[name=ExnC,handler=\trapExnCHandler,next=\trapExnCNext,color=WildStrawberry,opacity=0.25]{\trapExnCBegin}}
\coordAtRight{exnCtrap}{\exnHandlerAfterExnCTrap}


\ifthenelse{\step<6}
{\trapFrame[name=ExnB,handler=\trapExnBHandler,next=\trapExnBNext,color=OrangeRed]{\trapExnBBegin}}
{\trapFrame[name=ExnB,handler=\trapExnBHandler,next=\trapExnBNext,color=OrangeRed,opacity=0.25]{\trapExnBBegin}}
\coordAtRight{exnBtrap}{\exnHandlerAfterExnBTrap}
\ifthenelse{\step<4}
{\draw [->,>=latex] (exnBtrap.east) to [out=0,in=0] (exnCtrap.east);}{}


\ifthenelse{\step<4}
{\begin{funFrame}[name=maybe,color=YellowGreen]{\frameMaybeBegin}{\frameMaybeEnd}}
{\begin{funFrame}[name=maybe,color=YellowGreen,opacity=0.25]{\frameMaybeBegin}{\frameMaybeEnd}}
\funReturnAddr{\frameMaybeRetaddr}
\frameLocal{1}{\frameMaybeLocalOne}
\end{funFrame}


\ifthenelse{\step<4}
{\begin{funFrame}[name=probably,color=Green]{\frameProbablyBegin}{\frameProbablyEnd}}
{\begin{funFrame}[name=probably,color=Green,opacity=0.25]{\frameProbablyBegin}{\frameProbablyEnd}}
\funReturnAddr{\frameProbablyRetaddr}
\frameLocal{1}{\frameProbablyLocalOne}
\end{funFrame}


\ifthenelse{\step<3}{
\trapFrame[name=ExnA,handler=\trapExnAHandler,next=\trapExnANext,color=Red]{\trapExnABegin}
\coordAtRight{exnAtrap}{\exnHandlerAfterExnATrap}
}{
\trapFrame[name=ExnA,handler=\trapExnAHandler,next=\trapExnANext,color=Red,opacity=0.25]{\trapExnABegin}
}
\ifthenelse{\step=1}
{\draw [->,>=latex] (exnAtrap.east) to [out=0,in=0] (exnBtrap.east);}{}


\begin{funFrame}[name=definitely,color=JungleGreen,opacity=0.25]{\frameDefinitelyBegin}{\frameDefinitelyEnd}
\funReturnAddr{\frameDefinitelyRetaddr}
\frameLocal{1}{\frameDefinitelyLocalOne}
\end{funFrame}


\coordinate (bottom) at ($(stackBL) + (0,-0.5)$);
\ifthenelse{\step=1}
{\domainState[pos=(bottom)]{\exnHandlerAfterExnATrap}}{}
\ifthenelse{\step>1 \and \step<5}
{\domainState[pos=(bottom)]{\exnHandlerAfterExnBTrap}}{}
\ifthenelse{\step>4}
{\domainState[pos=(bottom)]{\exnHandlerAfterExnCTrap}}{}

\ifthenelse{\step=1}
{\draw [->,>=latex] (exnHandlerRight.east) to [out=0,in=0] (exnAtrap.east);}{}
\ifthenelse{\step>1 \and \step<5}
{\draw [->,>=latex] (exnHandlerRight.east) to [out=0,in=0] (exnBtrap.east);}{}
\ifthenelse{\step>4 \and \step<7}
{\draw [->,>=latex] (exnHandlerRight.east) to [out=0,in=0] (exnCtrap.east);}{}


\ifthenelse{\step=1}{
\showRegister{RSP}{\rspAfterExnAMove}
\showRegister[color=black!25]{RSP}{\frameDefinitelyEnd}
}{}
\ifthenelse{\step=2}{
\showRegister{RSP}{\rspAfterExnAFirstPop}
\showRegister[color=black!25]{RSP}{\rspAfterExnAMove}
}{}
\ifthenelse{\step=3}{
\showRegister{RSP}{\rspAfterExnASecondPop}
\showRegister[color=black!25]{RSP}{\rspAfterExnAFirstPop}
}{}
\ifthenelse{\step=4}{
\showRegister{RSP}{\rspAfterExnBMove}
\showRegister[color=black!25]{RSP}{\rspAfterExnASecondPop}
}{}
\ifthenelse{\step=5}{
\showRegister{RSP}{\rspAfterExnBFirstPop}
\showRegister[color=black!25]{RSP}{\rspAfterExnBMove}
}{}
\ifthenelse{\step=6}{
\showRegister{RSP}{\rspAfterExnBSecondPop}
\showRegister[color=black!25]{RSP}{\rspAfterExnBFirstPop}
}{}
\ifthenelse{\step=7}{
\showRegister{RSP}{\rspAfterExnCSecondPop}
}{}


\end{stackDiagram}
\end{tikzpicture}

\end{column}
\end{columns}
\end{frame}


\begin{frame}{maybe\_raise raises an exception}{Jumping to the trap}
\begin{columns}
\begin{column}{0.5\textwidth}
\only<1,5>{
\begin{itemize}
% TODO gray
\item \funcname{maybe\_raise} somehow loads the exception value into \regname{RAX}
% TODO /gray
\item \funcname{maybe\_raise} calls \typename{raise} with \typename{ExnA}
\only<5>{
\begin{itemize}
\item Sets \regname{RSP} to the value of \localname{domain\_state->exn\_handler}
\item \regname{RSP} now points to the last installed trap
\item This effectively discards \funcname{maybe} stack frame!
\end{itemize}
}
\end{itemize}
}%
\only<2>{
\mintcmm[highlightlines=7]{../src/raise/camlRaise\_\_maybe\_raise\_269.cmm}
}%
\only<3>{
\mintlinear[firstline=2,highlightlines=14]{../src/raise/camlRaise\_\_maybe\_raise\_269.linear}
}%
\only<4>{
  \listocaml[firstline=857,lastline=870,highlightlines={865-869}]{../ocaml/asmcomp/amd64/emit.mlp}{asmcomp/amd64/emit.mlp:857}
}%
\only<6>{
\listocaml[firstline=857,lastline=870,highlightlines=866]{../ocaml/asmcomp/amd64/emit.mlp}{asmcomp/amd64/emit.mlp:857}
}%
\only<7>{
\mintobjdump[firstline=5,lastline=24,highlightlines=16]{../src/raise/camlRaise\_\_maybe\_raise\_269.objdump}
}%
\bigskip
\listocaml[firstline=3,lastline=6,highlightlines=5]{../src/raise/raise.ml}{raise/raise.ml}
\end{column}
\begin{column}{0.5\textwidth}
\centering
\providecommand\step{2}\begin{tikzpicture}[font=\sffamily\tiny]
\input{../src/nested/gdb.out}

\begin{stackDiagram}[yFactor=0.35,showAddress=true]{\frameMainBegin}{\frameDefinitelyEnd}
\callStack

\begin{funFrame}[name=main,color=GreenYellow]{\frameMainBegin}{\frameMainEnd}
\funReturnAddr{\frameMainRetaddr}
\frameLocal{1}{\frameMainLocalOne}
\end{funFrame}


\ifthenelse{\step<7}
{\trapFrame[name=ExnC,handler=\trapExnCHandler,next=\trapExnCNext,color=WildStrawberry]{\trapExnCBegin}}
{\trapFrame[name=ExnC,handler=\trapExnCHandler,next=\trapExnCNext,color=WildStrawberry,opacity=0.25]{\trapExnCBegin}}
\coordAtRight{exnCtrap}{\exnHandlerAfterExnCTrap}


\ifthenelse{\step<6}
{\trapFrame[name=ExnB,handler=\trapExnBHandler,next=\trapExnBNext,color=OrangeRed]{\trapExnBBegin}}
{\trapFrame[name=ExnB,handler=\trapExnBHandler,next=\trapExnBNext,color=OrangeRed,opacity=0.25]{\trapExnBBegin}}
\coordAtRight{exnBtrap}{\exnHandlerAfterExnBTrap}
\ifthenelse{\step<4}
{\draw [->,>=latex] (exnBtrap.east) to [out=0,in=0] (exnCtrap.east);}{}


\ifthenelse{\step<4}
{\begin{funFrame}[name=maybe,color=YellowGreen]{\frameMaybeBegin}{\frameMaybeEnd}}
{\begin{funFrame}[name=maybe,color=YellowGreen,opacity=0.25]{\frameMaybeBegin}{\frameMaybeEnd}}
\funReturnAddr{\frameMaybeRetaddr}
\frameLocal{1}{\frameMaybeLocalOne}
\end{funFrame}


\ifthenelse{\step<4}
{\begin{funFrame}[name=probably,color=Green]{\frameProbablyBegin}{\frameProbablyEnd}}
{\begin{funFrame}[name=probably,color=Green,opacity=0.25]{\frameProbablyBegin}{\frameProbablyEnd}}
\funReturnAddr{\frameProbablyRetaddr}
\frameLocal{1}{\frameProbablyLocalOne}
\end{funFrame}


\ifthenelse{\step<3}{
\trapFrame[name=ExnA,handler=\trapExnAHandler,next=\trapExnANext,color=Red]{\trapExnABegin}
\coordAtRight{exnAtrap}{\exnHandlerAfterExnATrap}
}{
\trapFrame[name=ExnA,handler=\trapExnAHandler,next=\trapExnANext,color=Red,opacity=0.25]{\trapExnABegin}
}
\ifthenelse{\step=1}
{\draw [->,>=latex] (exnAtrap.east) to [out=0,in=0] (exnBtrap.east);}{}


\begin{funFrame}[name=definitely,color=JungleGreen,opacity=0.25]{\frameDefinitelyBegin}{\frameDefinitelyEnd}
\funReturnAddr{\frameDefinitelyRetaddr}
\frameLocal{1}{\frameDefinitelyLocalOne}
\end{funFrame}


\coordinate (bottom) at ($(stackBL) + (0,-0.5)$);
\ifthenelse{\step=1}
{\domainState[pos=(bottom)]{\exnHandlerAfterExnATrap}}{}
\ifthenelse{\step>1 \and \step<5}
{\domainState[pos=(bottom)]{\exnHandlerAfterExnBTrap}}{}
\ifthenelse{\step>4}
{\domainState[pos=(bottom)]{\exnHandlerAfterExnCTrap}}{}

\ifthenelse{\step=1}
{\draw [->,>=latex] (exnHandlerRight.east) to [out=0,in=0] (exnAtrap.east);}{}
\ifthenelse{\step>1 \and \step<5}
{\draw [->,>=latex] (exnHandlerRight.east) to [out=0,in=0] (exnBtrap.east);}{}
\ifthenelse{\step>4 \and \step<7}
{\draw [->,>=latex] (exnHandlerRight.east) to [out=0,in=0] (exnCtrap.east);}{}


\ifthenelse{\step=1}{
\showRegister{RSP}{\rspAfterExnAMove}
\showRegister[color=black!25]{RSP}{\frameDefinitelyEnd}
}{}
\ifthenelse{\step=2}{
\showRegister{RSP}{\rspAfterExnAFirstPop}
\showRegister[color=black!25]{RSP}{\rspAfterExnAMove}
}{}
\ifthenelse{\step=3}{
\showRegister{RSP}{\rspAfterExnASecondPop}
\showRegister[color=black!25]{RSP}{\rspAfterExnAFirstPop}
}{}
\ifthenelse{\step=4}{
\showRegister{RSP}{\rspAfterExnBMove}
\showRegister[color=black!25]{RSP}{\rspAfterExnASecondPop}
}{}
\ifthenelse{\step=5}{
\showRegister{RSP}{\rspAfterExnBFirstPop}
\showRegister[color=black!25]{RSP}{\rspAfterExnBMove}
}{}
\ifthenelse{\step=6}{
\showRegister{RSP}{\rspAfterExnBSecondPop}
\showRegister[color=black!25]{RSP}{\rspAfterExnBFirstPop}
}{}
\ifthenelse{\step=7}{
\showRegister{RSP}{\rspAfterExnCSecondPop}
}{}


\end{stackDiagram}
\end{tikzpicture}

\end{column}
\end{columns}
\end{frame}


\begin{frame}{maybe\_raise raises an exception}{Restoring exn\_handler}
\begin{columns}
\begin{column}{0.5\textwidth}
\only<1>{
\begin{itemize}
% TODO gray
\item \funcname{maybe\_raise} somehow loads the exception value into \regname{RAX}
\item \funcname{maybe\_raise} calls \typename{raise} with \typename{ExnA}
\begin{itemize}
\item Sets \regname{RSP} to the value of \localname{domain\_state->exn\_handler}
% TODO /gray
\item Restores \localname{domain\_state->exn\_handler} from the trap
\item \localname{domain\_state->exn\_handler} now holds the value from before the trap was installed
\end{itemize}
\end{itemize}
}%
\only<2>{
\listocaml[firstline=857,lastline=870,highlightlines=867]{../ocaml/asmcomp/amd64/emit.mlp}{asmcomp/amd64/emit.mlp:857}
}%
\only<3>{
\mintobjdump[firstline=5,lastline=24,highlightlines=17]{../src/raise/camlRaise\_\_maybe\_raise\_269.objdump}
}%
\bigskip
\listocaml[firstline=3,lastline=6,highlightlines=5]{../src/raise/raise.ml}{raise/raise.ml}
\end{column}
\begin{column}{0.5\textwidth}
\centering
\providecommand\step{3}\begin{tikzpicture}[font=\sffamily\tiny]
\input{../src/nested/gdb.out}

\begin{stackDiagram}[yFactor=0.35,showAddress=true]{\frameMainBegin}{\frameDefinitelyEnd}
\callStack

\begin{funFrame}[name=main,color=GreenYellow]{\frameMainBegin}{\frameMainEnd}
\funReturnAddr{\frameMainRetaddr}
\frameLocal{1}{\frameMainLocalOne}
\end{funFrame}


\ifthenelse{\step<7}
{\trapFrame[name=ExnC,handler=\trapExnCHandler,next=\trapExnCNext,color=WildStrawberry]{\trapExnCBegin}}
{\trapFrame[name=ExnC,handler=\trapExnCHandler,next=\trapExnCNext,color=WildStrawberry,opacity=0.25]{\trapExnCBegin}}
\coordAtRight{exnCtrap}{\exnHandlerAfterExnCTrap}


\ifthenelse{\step<6}
{\trapFrame[name=ExnB,handler=\trapExnBHandler,next=\trapExnBNext,color=OrangeRed]{\trapExnBBegin}}
{\trapFrame[name=ExnB,handler=\trapExnBHandler,next=\trapExnBNext,color=OrangeRed,opacity=0.25]{\trapExnBBegin}}
\coordAtRight{exnBtrap}{\exnHandlerAfterExnBTrap}
\ifthenelse{\step<4}
{\draw [->,>=latex] (exnBtrap.east) to [out=0,in=0] (exnCtrap.east);}{}


\ifthenelse{\step<4}
{\begin{funFrame}[name=maybe,color=YellowGreen]{\frameMaybeBegin}{\frameMaybeEnd}}
{\begin{funFrame}[name=maybe,color=YellowGreen,opacity=0.25]{\frameMaybeBegin}{\frameMaybeEnd}}
\funReturnAddr{\frameMaybeRetaddr}
\frameLocal{1}{\frameMaybeLocalOne}
\end{funFrame}


\ifthenelse{\step<4}
{\begin{funFrame}[name=probably,color=Green]{\frameProbablyBegin}{\frameProbablyEnd}}
{\begin{funFrame}[name=probably,color=Green,opacity=0.25]{\frameProbablyBegin}{\frameProbablyEnd}}
\funReturnAddr{\frameProbablyRetaddr}
\frameLocal{1}{\frameProbablyLocalOne}
\end{funFrame}


\ifthenelse{\step<3}{
\trapFrame[name=ExnA,handler=\trapExnAHandler,next=\trapExnANext,color=Red]{\trapExnABegin}
\coordAtRight{exnAtrap}{\exnHandlerAfterExnATrap}
}{
\trapFrame[name=ExnA,handler=\trapExnAHandler,next=\trapExnANext,color=Red,opacity=0.25]{\trapExnABegin}
}
\ifthenelse{\step=1}
{\draw [->,>=latex] (exnAtrap.east) to [out=0,in=0] (exnBtrap.east);}{}


\begin{funFrame}[name=definitely,color=JungleGreen,opacity=0.25]{\frameDefinitelyBegin}{\frameDefinitelyEnd}
\funReturnAddr{\frameDefinitelyRetaddr}
\frameLocal{1}{\frameDefinitelyLocalOne}
\end{funFrame}


\coordinate (bottom) at ($(stackBL) + (0,-0.5)$);
\ifthenelse{\step=1}
{\domainState[pos=(bottom)]{\exnHandlerAfterExnATrap}}{}
\ifthenelse{\step>1 \and \step<5}
{\domainState[pos=(bottom)]{\exnHandlerAfterExnBTrap}}{}
\ifthenelse{\step>4}
{\domainState[pos=(bottom)]{\exnHandlerAfterExnCTrap}}{}

\ifthenelse{\step=1}
{\draw [->,>=latex] (exnHandlerRight.east) to [out=0,in=0] (exnAtrap.east);}{}
\ifthenelse{\step>1 \and \step<5}
{\draw [->,>=latex] (exnHandlerRight.east) to [out=0,in=0] (exnBtrap.east);}{}
\ifthenelse{\step>4 \and \step<7}
{\draw [->,>=latex] (exnHandlerRight.east) to [out=0,in=0] (exnCtrap.east);}{}


\ifthenelse{\step=1}{
\showRegister{RSP}{\rspAfterExnAMove}
\showRegister[color=black!25]{RSP}{\frameDefinitelyEnd}
}{}
\ifthenelse{\step=2}{
\showRegister{RSP}{\rspAfterExnAFirstPop}
\showRegister[color=black!25]{RSP}{\rspAfterExnAMove}
}{}
\ifthenelse{\step=3}{
\showRegister{RSP}{\rspAfterExnASecondPop}
\showRegister[color=black!25]{RSP}{\rspAfterExnAFirstPop}
}{}
\ifthenelse{\step=4}{
\showRegister{RSP}{\rspAfterExnBMove}
\showRegister[color=black!25]{RSP}{\rspAfterExnASecondPop}
}{}
\ifthenelse{\step=5}{
\showRegister{RSP}{\rspAfterExnBFirstPop}
\showRegister[color=black!25]{RSP}{\rspAfterExnBMove}
}{}
\ifthenelse{\step=6}{
\showRegister{RSP}{\rspAfterExnBSecondPop}
\showRegister[color=black!25]{RSP}{\rspAfterExnBFirstPop}
}{}
\ifthenelse{\step=7}{
\showRegister{RSP}{\rspAfterExnCSecondPop}
}{}


\end{stackDiagram}
\end{tikzpicture}

\end{column}
\end{columns}
\end{frame}


\begin{frame}{maybe\_raise raises an exception}{Jumping to the handler code}
\begin{columns}
\begin{column}{0.5\textwidth}
\only<1>{
\begin{itemize}
% TODO gray
\item \funcname{maybe\_raise} somehow loads the exception value into \regname{RAX}
\item \funcname{maybe\_raise} calls \typename{raise} with \typename{ExnA}
\begin{itemize}
\item Sets \regname{RSP} to the value of \localname{domain\_state->exn\_handler}
\item Restores \localname{domain\_state->exn\_handler} from the trap
% TODO /gray
\item The execution flow jumps to the handler code inside \funcname{main} function
\item The trap is now completely removed from the stack!
\end{itemize}
\end{itemize}
}%
\only<2>{
\listocaml[firstline=857,lastline=870,highlightlines={868-869}]{../ocaml/asmcomp/amd64/emit.mlp}{asmcomp/amd64/emit.mlp:857}
}%
\only<3>{
\mintobjdump[firstline=5,lastline=24,highlightlines={18-19}]{../src/raise/camlRaise\_\_maybe\_raise\_269.objdump}
}%
\bigskip
\listocaml[firstline=3,lastline=11,highlightlines=5]{../src/raise/raise.ml}{raise/raise.ml}
\end{column}
\begin{column}{0.5\textwidth}
\centering
\providecommand\step{4}\begin{tikzpicture}[font=\sffamily\tiny]
\input{../src/nested/gdb.out}

\begin{stackDiagram}[yFactor=0.35,showAddress=true]{\frameMainBegin}{\frameDefinitelyEnd}
\callStack

\begin{funFrame}[name=main,color=GreenYellow]{\frameMainBegin}{\frameMainEnd}
\funReturnAddr{\frameMainRetaddr}
\frameLocal{1}{\frameMainLocalOne}
\end{funFrame}


\ifthenelse{\step<7}
{\trapFrame[name=ExnC,handler=\trapExnCHandler,next=\trapExnCNext,color=WildStrawberry]{\trapExnCBegin}}
{\trapFrame[name=ExnC,handler=\trapExnCHandler,next=\trapExnCNext,color=WildStrawberry,opacity=0.25]{\trapExnCBegin}}
\coordAtRight{exnCtrap}{\exnHandlerAfterExnCTrap}


\ifthenelse{\step<6}
{\trapFrame[name=ExnB,handler=\trapExnBHandler,next=\trapExnBNext,color=OrangeRed]{\trapExnBBegin}}
{\trapFrame[name=ExnB,handler=\trapExnBHandler,next=\trapExnBNext,color=OrangeRed,opacity=0.25]{\trapExnBBegin}}
\coordAtRight{exnBtrap}{\exnHandlerAfterExnBTrap}
\ifthenelse{\step<4}
{\draw [->,>=latex] (exnBtrap.east) to [out=0,in=0] (exnCtrap.east);}{}


\ifthenelse{\step<4}
{\begin{funFrame}[name=maybe,color=YellowGreen]{\frameMaybeBegin}{\frameMaybeEnd}}
{\begin{funFrame}[name=maybe,color=YellowGreen,opacity=0.25]{\frameMaybeBegin}{\frameMaybeEnd}}
\funReturnAddr{\frameMaybeRetaddr}
\frameLocal{1}{\frameMaybeLocalOne}
\end{funFrame}


\ifthenelse{\step<4}
{\begin{funFrame}[name=probably,color=Green]{\frameProbablyBegin}{\frameProbablyEnd}}
{\begin{funFrame}[name=probably,color=Green,opacity=0.25]{\frameProbablyBegin}{\frameProbablyEnd}}
\funReturnAddr{\frameProbablyRetaddr}
\frameLocal{1}{\frameProbablyLocalOne}
\end{funFrame}


\ifthenelse{\step<3}{
\trapFrame[name=ExnA,handler=\trapExnAHandler,next=\trapExnANext,color=Red]{\trapExnABegin}
\coordAtRight{exnAtrap}{\exnHandlerAfterExnATrap}
}{
\trapFrame[name=ExnA,handler=\trapExnAHandler,next=\trapExnANext,color=Red,opacity=0.25]{\trapExnABegin}
}
\ifthenelse{\step=1}
{\draw [->,>=latex] (exnAtrap.east) to [out=0,in=0] (exnBtrap.east);}{}


\begin{funFrame}[name=definitely,color=JungleGreen,opacity=0.25]{\frameDefinitelyBegin}{\frameDefinitelyEnd}
\funReturnAddr{\frameDefinitelyRetaddr}
\frameLocal{1}{\frameDefinitelyLocalOne}
\end{funFrame}


\coordinate (bottom) at ($(stackBL) + (0,-0.5)$);
\ifthenelse{\step=1}
{\domainState[pos=(bottom)]{\exnHandlerAfterExnATrap}}{}
\ifthenelse{\step>1 \and \step<5}
{\domainState[pos=(bottom)]{\exnHandlerAfterExnBTrap}}{}
\ifthenelse{\step>4}
{\domainState[pos=(bottom)]{\exnHandlerAfterExnCTrap}}{}

\ifthenelse{\step=1}
{\draw [->,>=latex] (exnHandlerRight.east) to [out=0,in=0] (exnAtrap.east);}{}
\ifthenelse{\step>1 \and \step<5}
{\draw [->,>=latex] (exnHandlerRight.east) to [out=0,in=0] (exnBtrap.east);}{}
\ifthenelse{\step>4 \and \step<7}
{\draw [->,>=latex] (exnHandlerRight.east) to [out=0,in=0] (exnCtrap.east);}{}


\ifthenelse{\step=1}{
\showRegister{RSP}{\rspAfterExnAMove}
\showRegister[color=black!25]{RSP}{\frameDefinitelyEnd}
}{}
\ifthenelse{\step=2}{
\showRegister{RSP}{\rspAfterExnAFirstPop}
\showRegister[color=black!25]{RSP}{\rspAfterExnAMove}
}{}
\ifthenelse{\step=3}{
\showRegister{RSP}{\rspAfterExnASecondPop}
\showRegister[color=black!25]{RSP}{\rspAfterExnAFirstPop}
}{}
\ifthenelse{\step=4}{
\showRegister{RSP}{\rspAfterExnBMove}
\showRegister[color=black!25]{RSP}{\rspAfterExnASecondPop}
}{}
\ifthenelse{\step=5}{
\showRegister{RSP}{\rspAfterExnBFirstPop}
\showRegister[color=black!25]{RSP}{\rspAfterExnBMove}
}{}
\ifthenelse{\step=6}{
\showRegister{RSP}{\rspAfterExnBSecondPop}
\showRegister[color=black!25]{RSP}{\rspAfterExnBFirstPop}
}{}
\ifthenelse{\step=7}{
\showRegister{RSP}{\rspAfterExnCSecondPop}
}{}


\end{stackDiagram}
\end{tikzpicture}

\end{column}
\end{columns}
\end{frame}


\begin{frame}{maybe\_raise raises an exception}{Executing the handler code}
\begin{columns}
\begin{column}{0.5\textwidth}
\only<1>{
\begin{itemize}
% TODO gray
\item \funcname{maybe\_raise} somehow loads the exception value into \regname{RAX}
\item \funcname{maybe\_raise} calls \typename{raise} with \typename{ExnA}
\begin{itemize}
\item Sets \regname{RSP} to the value of \localname{domain\_state->exn\_handler}
\item Restores \localname{domain\_state->exn\_handler} from the trap
\item The execution flow jumps to the handler code inside \funcname{main} function
\item The trap is now completely removed from the stack!
\end{itemize}
% TODO /gray
\item The handler pattern matches against the caught exception
\end{itemize}
}%
\only<2>{
\mintcmm[highlightlines={3-7}]{../src/raise/camlRaise\_\_main\_273.cmm}
}%
\only<3>{
\mintlinear[firstline=2,lastline=32,highlightlines={11-20}]{../src/raise/camlRaise\_\_main\_273.linear}
}%
\only<4>{
\mintobjdump[firstline=5,lastline=32,highlightlines={16-27}]{../src/raise/camlRaise\_\_main\_273.objdump}
}%
\bigskip
\listocaml[firstline=8,lastline=11,highlightlines=10]{../src/raise/raise.ml}{raise/raise.ml}
\end{column}
\begin{column}{0.5\textwidth}
\centering
\providecommand\step{4}\begin{tikzpicture}[font=\sffamily\tiny]
\input{../src/nested/gdb.out}

\begin{stackDiagram}[yFactor=0.35,showAddress=true]{\frameMainBegin}{\frameDefinitelyEnd}
\callStack

\begin{funFrame}[name=main,color=GreenYellow]{\frameMainBegin}{\frameMainEnd}
\funReturnAddr{\frameMainRetaddr}
\frameLocal{1}{\frameMainLocalOne}
\end{funFrame}


\ifthenelse{\step<7}
{\trapFrame[name=ExnC,handler=\trapExnCHandler,next=\trapExnCNext,color=WildStrawberry]{\trapExnCBegin}}
{\trapFrame[name=ExnC,handler=\trapExnCHandler,next=\trapExnCNext,color=WildStrawberry,opacity=0.25]{\trapExnCBegin}}
\coordAtRight{exnCtrap}{\exnHandlerAfterExnCTrap}


\ifthenelse{\step<6}
{\trapFrame[name=ExnB,handler=\trapExnBHandler,next=\trapExnBNext,color=OrangeRed]{\trapExnBBegin}}
{\trapFrame[name=ExnB,handler=\trapExnBHandler,next=\trapExnBNext,color=OrangeRed,opacity=0.25]{\trapExnBBegin}}
\coordAtRight{exnBtrap}{\exnHandlerAfterExnBTrap}
\ifthenelse{\step<4}
{\draw [->,>=latex] (exnBtrap.east) to [out=0,in=0] (exnCtrap.east);}{}


\ifthenelse{\step<4}
{\begin{funFrame}[name=maybe,color=YellowGreen]{\frameMaybeBegin}{\frameMaybeEnd}}
{\begin{funFrame}[name=maybe,color=YellowGreen,opacity=0.25]{\frameMaybeBegin}{\frameMaybeEnd}}
\funReturnAddr{\frameMaybeRetaddr}
\frameLocal{1}{\frameMaybeLocalOne}
\end{funFrame}


\ifthenelse{\step<4}
{\begin{funFrame}[name=probably,color=Green]{\frameProbablyBegin}{\frameProbablyEnd}}
{\begin{funFrame}[name=probably,color=Green,opacity=0.25]{\frameProbablyBegin}{\frameProbablyEnd}}
\funReturnAddr{\frameProbablyRetaddr}
\frameLocal{1}{\frameProbablyLocalOne}
\end{funFrame}


\ifthenelse{\step<3}{
\trapFrame[name=ExnA,handler=\trapExnAHandler,next=\trapExnANext,color=Red]{\trapExnABegin}
\coordAtRight{exnAtrap}{\exnHandlerAfterExnATrap}
}{
\trapFrame[name=ExnA,handler=\trapExnAHandler,next=\trapExnANext,color=Red,opacity=0.25]{\trapExnABegin}
}
\ifthenelse{\step=1}
{\draw [->,>=latex] (exnAtrap.east) to [out=0,in=0] (exnBtrap.east);}{}


\begin{funFrame}[name=definitely,color=JungleGreen,opacity=0.25]{\frameDefinitelyBegin}{\frameDefinitelyEnd}
\funReturnAddr{\frameDefinitelyRetaddr}
\frameLocal{1}{\frameDefinitelyLocalOne}
\end{funFrame}


\coordinate (bottom) at ($(stackBL) + (0,-0.5)$);
\ifthenelse{\step=1}
{\domainState[pos=(bottom)]{\exnHandlerAfterExnATrap}}{}
\ifthenelse{\step>1 \and \step<5}
{\domainState[pos=(bottom)]{\exnHandlerAfterExnBTrap}}{}
\ifthenelse{\step>4}
{\domainState[pos=(bottom)]{\exnHandlerAfterExnCTrap}}{}

\ifthenelse{\step=1}
{\draw [->,>=latex] (exnHandlerRight.east) to [out=0,in=0] (exnAtrap.east);}{}
\ifthenelse{\step>1 \and \step<5}
{\draw [->,>=latex] (exnHandlerRight.east) to [out=0,in=0] (exnBtrap.east);}{}
\ifthenelse{\step>4 \and \step<7}
{\draw [->,>=latex] (exnHandlerRight.east) to [out=0,in=0] (exnCtrap.east);}{}


\ifthenelse{\step=1}{
\showRegister{RSP}{\rspAfterExnAMove}
\showRegister[color=black!25]{RSP}{\frameDefinitelyEnd}
}{}
\ifthenelse{\step=2}{
\showRegister{RSP}{\rspAfterExnAFirstPop}
\showRegister[color=black!25]{RSP}{\rspAfterExnAMove}
}{}
\ifthenelse{\step=3}{
\showRegister{RSP}{\rspAfterExnASecondPop}
\showRegister[color=black!25]{RSP}{\rspAfterExnAFirstPop}
}{}
\ifthenelse{\step=4}{
\showRegister{RSP}{\rspAfterExnBMove}
\showRegister[color=black!25]{RSP}{\rspAfterExnASecondPop}
}{}
\ifthenelse{\step=5}{
\showRegister{RSP}{\rspAfterExnBFirstPop}
\showRegister[color=black!25]{RSP}{\rspAfterExnBMove}
}{}
\ifthenelse{\step=6}{
\showRegister{RSP}{\rspAfterExnBSecondPop}
\showRegister[color=black!25]{RSP}{\rspAfterExnBFirstPop}
}{}
\ifthenelse{\step=7}{
\showRegister{RSP}{\rspAfterExnCSecondPop}
}{}


\end{stackDiagram}
\end{tikzpicture}

\end{column}
\end{columns}
\end{frame}


\begin{frame}{Aside}{Comparison with C++ exceptions}
\begin{columns}
\begin{column}{0.5\textwidth}
\begin{itemize}
\item RAII: Resource Acquisition Is Initialization
\begin{itemize}
\item Bind the life cycle of a resource that must be acquired to the lifetime of an object
\item Not only about allocated memory: also file descriptors, mutexes\dots
\item Guarante exception safety
\end{itemize}
\item Raising an exception causes stack unwinding
\item Objects of an unwinded function are destructed (in reverse order of construction)
\item Destruction of a RAII object releases the associated ressource
\end{itemize}
\only<2>{
\bigskip
\inputminted[fontsize=\footnotesize]{text}{src/cpp/out.txt}
}%
\end{column}
\begin{column}{0.5\textwidth}
\mintcpp[firstline=5]{src/cpp/raii.cpp}
\end{column}
\end{columns}
\footnotetext[1]{LousyString is incorrect by the \href{https://en.cppreference.com/w/cpp/language/rule\_of\_three}{rule of three/five/zero}}
\end{frame}
% https://en.cppreference.com/w/cpp/language/raii


\begin{frame}{Finalization}{Reproducing frame unwinding behaviour}
\text{``}Finalization can be performed by trapping all exceptions, performing the finalization, then re-raising the exception.\text{''} \footnotemark[1]
\bigskip
\centering
\mintocaml{src/ocaml/finalization.ml}
\footnotetext[1]{\url{https://v2.ocaml.org/manual/coreexamples.html\#s\%3Aexceptions}}
\end{frame}


%
% Takeaway #3
%
\subsection*{Takeaway \#3}
\frameSubsectionTakeaway{}
\againframe<3>{takeaway}


%
% Questions?
%
\subsection*{Questions?}
\frameSubsection{pictures/show\_me.png}{- I know kung fu. - Show me.}

\begin{frame}{Exception handler and tail (recursive) calls}{Which one(s) is (are) tail-recursive? Why?}
\begin{columns}
\begin{column}{0.45\textwidth}
\listocaml[firstline=7,lastline=13]{../src/tailrec/ex1/example.ml}{tailrec/ex1}
\medskip
\listocaml[firstline=7,lastline=14]{../src/tailrec/ex2/example.ml}{tailrec/ex2}
\end{column}
\begin{column}{0.55\textwidth}
\centering
\only<2>{\listlinear[firstline=2,highlightlines={10-19}]{../src/tailrec/ex1/camlExample\_\_f\_272.linear}{tailrec/ex1}}%
\only<3>{\listobjdump[firstline=5,lastline=32,highlightlines={14-23}]{../src/tailrec/ex1/camlExample\_\_f\_272.objdump}{tailrec/ex1}}%
\only<4>{\providecommand\step{1}\begin{tikzpicture}[font=\sffamily\tiny]
\input{../src/tailrec/ex1/gdb.out}

\begin{stackDiagram}[yFactor=0.35,showAddress=true]{\frameEntryBegin}{\frameFEndvD}
\callStack

\begin{funFrame}[name=entry,color=GreenYellow]{\frameEntryBegin}{\frameEntryEnd}
\funReturnAddr{\frameEntryRetaddr}
\end{funFrame}


\begin{funFrame}[name=f,color=YellowGreen]{\frameFBeginvA}{\frameFEndvA}
\funReturnAddr{\frameFRetaddrvA}
\end{funFrame}

\trapFrame[name=Trap,handler=\trapHandlervA,next=\trapNextvA,color=WildStrawberry]{\trapBeginvA}
\coordAtRight{trapA}{\trapEndvA}


\ifthenelse{\step>1}{
\begin{funFrame}[name=f,color=Green]{\frameFBeginvB}{\frameFEndvB}
\funReturnAddr{\frameFRetaddrvB}
\end{funFrame}

\trapFrame[name=Trap,handler=\trapHandlervB,next=\trapNextvB,color=OrangeRed]{\trapBeginvB}
\coordAtRight{trapB}{\trapEndvB}
% No spoilers!
%\draw [->,>=latex] (trapB.east) to [out=0,in=0] (trapA.east);
}{}


\ifthenelse{\step>2}{
\begin{funFrame}[name=f,color=JungleGreen]{\frameFBeginvC}{\frameFEndvC}
\funReturnAddr{\frameFRetaddrvC}
\end{funFrame}

\trapFrame[name=Trap,handler=\trapHandlervC,next=\trapNextvC,color=Red]{\trapBeginvC}
\coordAtRight{trapC}{\trapEndvC}
% No spoilers!
%\draw [->,>=latex] (trapC.east) to [out=0,in=0] (trapB.east);
}{}


\ifthenelse{\step>3}{
\begin{funFrame}[name=f,color=SeaGreen]{\frameFBeginvD}{\frameFEndvD}
\funReturnAddr{\frameFRetaddrvD}
\end{funFrame}
}{}


\end{stackDiagram}
\end{tikzpicture}
}%
\only<5>{\providecommand\step{2}\begin{tikzpicture}[font=\sffamily\tiny]
\input{../src/tailrec/ex1/gdb.out}

\begin{stackDiagram}[yFactor=0.35,showAddress=true]{\frameEntryBegin}{\frameFEndvD}
\callStack

\begin{funFrame}[name=entry,color=GreenYellow]{\frameEntryBegin}{\frameEntryEnd}
\funReturnAddr{\frameEntryRetaddr}
\end{funFrame}


\begin{funFrame}[name=f,color=YellowGreen]{\frameFBeginvA}{\frameFEndvA}
\funReturnAddr{\frameFRetaddrvA}
\end{funFrame}

\trapFrame[name=Trap,handler=\trapHandlervA,next=\trapNextvA,color=WildStrawberry]{\trapBeginvA}
\coordAtRight{trapA}{\trapEndvA}


\ifthenelse{\step>1}{
\begin{funFrame}[name=f,color=Green]{\frameFBeginvB}{\frameFEndvB}
\funReturnAddr{\frameFRetaddrvB}
\end{funFrame}

\trapFrame[name=Trap,handler=\trapHandlervB,next=\trapNextvB,color=OrangeRed]{\trapBeginvB}
\coordAtRight{trapB}{\trapEndvB}
% No spoilers!
%\draw [->,>=latex] (trapB.east) to [out=0,in=0] (trapA.east);
}{}


\ifthenelse{\step>2}{
\begin{funFrame}[name=f,color=JungleGreen]{\frameFBeginvC}{\frameFEndvC}
\funReturnAddr{\frameFRetaddrvC}
\end{funFrame}

\trapFrame[name=Trap,handler=\trapHandlervC,next=\trapNextvC,color=Red]{\trapBeginvC}
\coordAtRight{trapC}{\trapEndvC}
% No spoilers!
%\draw [->,>=latex] (trapC.east) to [out=0,in=0] (trapB.east);
}{}


\ifthenelse{\step>3}{
\begin{funFrame}[name=f,color=SeaGreen]{\frameFBeginvD}{\frameFEndvD}
\funReturnAddr{\frameFRetaddrvD}
\end{funFrame}
}{}


\end{stackDiagram}
\end{tikzpicture}
}%
\only<6>{\providecommand\step{3}\begin{tikzpicture}[font=\sffamily\tiny]
\input{../src/tailrec/ex1/gdb.out}

\begin{stackDiagram}[yFactor=0.35,showAddress=true]{\frameEntryBegin}{\frameFEndvD}
\callStack

\begin{funFrame}[name=entry,color=GreenYellow]{\frameEntryBegin}{\frameEntryEnd}
\funReturnAddr{\frameEntryRetaddr}
\end{funFrame}


\begin{funFrame}[name=f,color=YellowGreen]{\frameFBeginvA}{\frameFEndvA}
\funReturnAddr{\frameFRetaddrvA}
\end{funFrame}

\trapFrame[name=Trap,handler=\trapHandlervA,next=\trapNextvA,color=WildStrawberry]{\trapBeginvA}
\coordAtRight{trapA}{\trapEndvA}


\ifthenelse{\step>1}{
\begin{funFrame}[name=f,color=Green]{\frameFBeginvB}{\frameFEndvB}
\funReturnAddr{\frameFRetaddrvB}
\end{funFrame}

\trapFrame[name=Trap,handler=\trapHandlervB,next=\trapNextvB,color=OrangeRed]{\trapBeginvB}
\coordAtRight{trapB}{\trapEndvB}
% No spoilers!
%\draw [->,>=latex] (trapB.east) to [out=0,in=0] (trapA.east);
}{}


\ifthenelse{\step>2}{
\begin{funFrame}[name=f,color=JungleGreen]{\frameFBeginvC}{\frameFEndvC}
\funReturnAddr{\frameFRetaddrvC}
\end{funFrame}

\trapFrame[name=Trap,handler=\trapHandlervC,next=\trapNextvC,color=Red]{\trapBeginvC}
\coordAtRight{trapC}{\trapEndvC}
% No spoilers!
%\draw [->,>=latex] (trapC.east) to [out=0,in=0] (trapB.east);
}{}


\ifthenelse{\step>3}{
\begin{funFrame}[name=f,color=SeaGreen]{\frameFBeginvD}{\frameFEndvD}
\funReturnAddr{\frameFRetaddrvD}
\end{funFrame}
}{}


\end{stackDiagram}
\end{tikzpicture}
}%
\only<7>{\providecommand\step{4}\begin{tikzpicture}[font=\sffamily\tiny]
\input{../src/tailrec/ex1/gdb.out}

\begin{stackDiagram}[yFactor=0.35,showAddress=true]{\frameEntryBegin}{\frameFEndvD}
\callStack

\begin{funFrame}[name=entry,color=GreenYellow]{\frameEntryBegin}{\frameEntryEnd}
\funReturnAddr{\frameEntryRetaddr}
\end{funFrame}


\begin{funFrame}[name=f,color=YellowGreen]{\frameFBeginvA}{\frameFEndvA}
\funReturnAddr{\frameFRetaddrvA}
\end{funFrame}

\trapFrame[name=Trap,handler=\trapHandlervA,next=\trapNextvA,color=WildStrawberry]{\trapBeginvA}
\coordAtRight{trapA}{\trapEndvA}


\ifthenelse{\step>1}{
\begin{funFrame}[name=f,color=Green]{\frameFBeginvB}{\frameFEndvB}
\funReturnAddr{\frameFRetaddrvB}
\end{funFrame}

\trapFrame[name=Trap,handler=\trapHandlervB,next=\trapNextvB,color=OrangeRed]{\trapBeginvB}
\coordAtRight{trapB}{\trapEndvB}
% No spoilers!
%\draw [->,>=latex] (trapB.east) to [out=0,in=0] (trapA.east);
}{}


\ifthenelse{\step>2}{
\begin{funFrame}[name=f,color=JungleGreen]{\frameFBeginvC}{\frameFEndvC}
\funReturnAddr{\frameFRetaddrvC}
\end{funFrame}

\trapFrame[name=Trap,handler=\trapHandlervC,next=\trapNextvC,color=Red]{\trapBeginvC}
\coordAtRight{trapC}{\trapEndvC}
% No spoilers!
%\draw [->,>=latex] (trapC.east) to [out=0,in=0] (trapB.east);
}{}


\ifthenelse{\step>3}{
\begin{funFrame}[name=f,color=SeaGreen]{\frameFBeginvD}{\frameFEndvD}
\funReturnAddr{\frameFRetaddrvD}
\end{funFrame}
}{}


\end{stackDiagram}
\end{tikzpicture}
}%
\only<9>{\listlinear[firstline=2,highlightlines={12-17,26}]{../src/tailrec/ex2/camlExample\_\_f\_272.linear}{tailrec/ex2}}%
\end{column}
\end{columns}
\end{frame}

