\newif\ifHexadecimal
\newif\ifShowAddress
\newif\ifNextArrow
\pgfkeys{
/stackDiagram/.is family,
/stackDiagram,
default/.style={wordSize=8,xFactor=3.8,yFactor=1,stackOffset=8,colorOpacity=50,showAddress=false},
wordSize/.estore in=\wordSize,
xFactor/.estore in=\xFactor,
yFactor/.estore in=\yFactor,
stackOffset/.estore in=\stackOffset,
colorOpacity/.estore in=\colorOpacity,
showAddress/.is if=ShowAddress,
%
/stackFrame/.is family,
/stackFrame,
default/.style={color=GreenYellow,opacity=1},
color/.estore in=\sfColor,
opacity/.estore in=\sfOpacity,
%
/frameLocal/.is family,
/frameLocal,
default/.style={hex=false,opacity=1},
hex/.is if=Hexadecimal,
opacity/.estore in=\flOpacity,
%
/funFrame/.is family,
/funFrame,
default/.style={name={},color=GreenYellow,opacity=1,retaddr={}},
name/.estore in=\ffName,
color/.estore in=\ffColor,
opacity/.estore in=\ffOpacity,
retaddr/.estore in=\ffRetAddr,
%
/trapFrame/.is family,
/trapFrame,
default/.style={name={},color=RedOrange,opacity=1,handler={},next=0},
name/.estore in=\tfName,
color/.estore in=\tfColor,
opacity/.estore in=\tfOpacity,
handler/.estore in=\tfHandler,
next/.estore in=\tfNext,
nextarrow/.is if=NextArrow,
%
/showRegister/.is family,
/showRegister,
default/.style={color=Black,opacity=1,offset=0},
color/.estore in=\srColor,
opacity/.estore in=\srOpacity,
offset/.estore in=\srOffset,
}

\newcommand{\hexToDecInto}[2]{%
	\StrGobbleLeft{#2}{2}[\stepA]%
	\StrSubstitute{\stepA}{a}{A}[\stepB]%
	\StrSubstitute{\stepB}{b}{B}[\stepC]%
	\StrSubstitute{\stepC}{c}{C}[\stepD]%
	\StrSubstitute{\stepD}{d}{D}[\stepE]%
	\StrSubstitute{\stepE}{e}{E}[\stepF]%
	\StrSubstitute{\stepF}{f}{F}[\out]%
	\edef#1{\xintHexToDec{\out}}
}

\newcommand{\isBigNumber}[1]{\pdfmatch{^[0-9]+$}{#1}}
\newcommand{\isBigNum}[1]{\ifthenelse{\pdfmatch{^[0-9]+$}{#1} = 1}{red}{blue}}

\tikzset{
addr/.style={right,color=black!50,fill=white,font={\ttfamily\tiny},xshift=1}
}


\newenvironment{stackDiagram}[3][]{%
% #2 -> StackBegin
% #3 -> StackEnd
\pgfkeys{/stackDiagram,default,#1}%
%
\newcommand\stackBegin{\xinteval{#2 + \stackOffset}}
\newcommand\stackEnd{\xinteval{#3 - \stackOffset * 2}}
}%
{}

\newcommand{\debugAxis}{{
\newcommand\x{-1}
\draw [dashed,gray!50] (\x, 0) -- (\x, -10);
\foreach \i in {0, -1, ..., -10} {
	\draw (\x, \i) node [font=\tiny] {$\bullet$};
	\draw (\x, \i) node [left,font=\tiny] {\i};
}
}}


\newcommand{\yOffset}[1]{{
% #1 -> Addr
\xintfloateval{-(((\stackBegin - (#1)) // \wordSize) * \yFactor)}
}}


\newcommand{\debugAddr}[2][blue]{{
% #1 -> FontColor (optional)
% #2 -> Addr

\newcommand\debugAddrX{-1}
\newcommand\debugAddrY{\yOffset{#2}}

\draw (\debugAddrX, \debugAddrY) node [color=#1] {$\bullet$};
}}


\newcommand{\debugCoord}{{
\newcommand\debugCoordX{-1}
\newcommand\debugCoordDiff{\xinteval{\stackBegin - \stackEnd}}

\foreach \i in {0, 8, ..., \debugCoordDiff} {
	\newcommand\debugCoordAddrIt{\xinteval{\stackBegin - \i}}

	\debugAddr{\debugCoordAddrIt}

	\newcommand\debugCoordAddrY{\yOffset{\debugCoordAddrIt}}
	\draw (\debugCoordX, \debugCoordAddrY) node [left,font=\small]{\xintDecToHex{\debugCoordAddrIt} is at \debugCoordAddrY};
}
}}

% TODO optional pgfkey for left/right
\newcommand{\coordAtLeft}[2]{{
% #1 -> Name
% #2 -> Address

\tikzmath{
\y = \yOffset{#2};
\y = \y + 0.5 * \yFactor;
}

\coordinate (#1) at (0, \y);
}}

\newcommand{\coordAtRight}[2]{{
% #1 -> Name
% #2 -> Address

\tikzmath{
\y = \yOffset{#2};
\y = \y + 0.5 * \yFactor;
}

\coordinate (#1) at (\xFactor, \y);
}}

\newcommand{\colorWithOpacity}[1]{%
#1!\colorOpacity%
}

\newcommand\callStack{{
\tikzmath{
\Ax = 0; \Ay = \yOffset{\stackBegin};
\Bx = 1 * \xFactor; \By = \Ay;
\Cx = \Bx; \Cy = \yOffset{\stackEnd};
\Dx = \Ax; \Dy = \Cy;
}

\coordinate (stackTL) at (\Ax, \Ay);
\coordinate (stackTR) at (\Bx, \By);
\coordinate (stackBR) at (\Cx, \Cy);
\coordinate (stackBL) at (\Dx, \Dy);

\draw [dashed] (stackTL)
-- (stackTR) node [addr] {\ifShowAddress 0xFFFFFFFFFFFF \fi}
-- (stackBR) node [addr] {\ifShowAddress 0x0 \fi }
-- (stackBL)
-- cycle;

\ifnum \stackOffset > 0
\tikzmath{
\Ax = 0; \Ay = \yOffset{\stackBegin};
\Bx = 1 * \xFactor; \By = \Ay;
\Cx = \Bx; \Cy = \yOffset{\stackBegin - \stackOffset};
\Dx = \Ax; \Dy = \Cy;
}

\fill [pattern=crosshatch,opacity=0.4] (\Ax, \Ay) -- (\Bx, \By) -- (\Cx, \Cy) -- (\Dx,\Dy) -- cycle;
\fi
}}


\newenvironment{stackFrame}[3][]{
% #2 -> FrameBegin
% #3 -> FrameEnd
\pgfkeys{/stackFrame,default,#1}

\newcommand\sfBegin{#2}
\newcommand\sfEnd{#3}

\tikzmath{
\Ax = 0; \Ay = \yOffset{#2};
\Bx = 1 * \xFactor; \By = \Ay;
\Cx = \Bx; \Cy = \yOffset{#3};
\Dx = \Ax; \Dy = \Cy;
}

\filldraw[fill=\colorWithOpacity{\sfColor},opacity=\sfOpacity,draw=black]
	(\Ax, \Ay)
	-- (\Bx, \By) node [addr] {\ifShowAddress 0x\xintDecToHex{#2} \fi}
	-- (\Cx, \Cy) node [addr] {\ifShowAddress 0x\xintDecToHex{#3} \fi}
	-- (\Dx, \Dy)
	--cycle;

\newcommand\total{\xintiieval{#2 - #3 - 1}}
\foreach \i in {0, 8, ..., \total} {
\ifnum \i>0
	\draw [thin,dashed,gray,opacity=0.5] (0, \yOffset{#2 - \i}) -- (\xFactor, \yOffset{#2 - \i});
\fi
}
}{}


\newcommand\frameName[1]{{
\tikzmath{
\Lx = 1 * \xFactor;
\Lyt= \yOffset{\sfBegin}; \Lyb = \yOffset{\sfEnd};
\Ly = (\Lyt + \Lyb) / 2;
}

% FIXME fill=\colorWithOpacity{\sfColor}
\draw (\Lx, \Ly) node [above,rotate=90,opacity=\sfOpacity,yshift=1] {#1};
}}


\newcommand\frameLocalAt[3][]{{
% #2 -> localAddr
% #3 -> localValue
\pgfkeys{/frameLocal,default,#1}

\tikzmath{
\Ax = \xFactor * 0.5;
\Ay = \yOffset{#2}; \Ay = \Ay + 0.5 * \yFactor;
}

\ifHexadecimal
\draw (\Ax, \Ay) node [opacity=\flOpacity] {0x\xintDecToHex{#3}};
\else
\draw (\Ax, \Ay) node [opacity=\flOpacity] {#3};
\fi
}}


\newcommand\frameLocal[3][]{{
% #2 -> localOffset
% #3 -> localValue
\pgfkeys{/frameLocal,default,#1}

\newcommand\sfLocalAddr{%
\xintfloateval{\sfBegin - ((#2 + 1) * \wordSize)}%
}
\frameLocalAt[#1,opacity=\sfOpacity]{\sfLocalAddr}{#3}
}}


\newcommand\frameReturnAddr[2][]{{
% #2 -> frameReturnAddr
\pgfkeys{/frameLocal,default,#1}

\frameLocal[#1]{0}{#2}
}}


\newcommand{\pointerTo}[2]{{
% #1 -> ptrAddr
% #2 -> ptrValue

\frameLocalAt[hex=true]{#1}{#2}

\tikzmath{
\Ax = 0;
\Ay = \yOffset{#1}; \Ay = \Ay + 0.5 * \yFactor;
\Bx = 0;
\By = \yOffset{#2}; \By = \By + 0.5 * \yFactor;
\Cx = -0.5; \Cy = (\Ay + \By) / 2;
}
 
\draw [darkgray] (0.5, \Ay) -- (\Ax, \Ay);
\draw [darkgray,rounded corners] (\Ax, \Ay) -| (\Cx, \Cy);
\draw [darkgray,->,rounded corners] (\Cx, \Cy) |- (\Bx, \By);
}}


\newenvironment{funFrame}[3][]{
% #2 -> ffBegin
% #3 -> ffEnd
\pgfkeys{/funFrame,default,#1}

\begin{stackFrame}[color=\ffColor,opacity=\ffOpacity]{#2}{#3}
\IfStrEq{\ffName}{}
{}
{\funName{\ffName}}
\IfStrEq{\ffRetAddr}{}
{}
{\funReturnAddr{\ffRetAddr}}
}
{\end{stackFrame}}


\newcommand\funName[1]{{
\frameName{#1}
}}


\newcommand\funReturnAddr[2][]{{
% #2 -> funReturnAddr
\pgfkeys{/frameLocal,default,#1}

\frameReturnAddr[#1]{#2}
}}

\newcommand\showRegister[3][] {
% #2 -> regName
% #3 -> regAddr
\pgfkeys{/showRegister,default,#1}%

\tikzmath{
\Ax = -0.3 + \srOffset;
\Ay = \yOffset{#3}; \Ay = \Ay + 0.5 * \yFactor;
\Bx = 0;
\By = \yOffset{#3}; \By = \By + 0.5 * \yFactor;
}

\draw [->,>=stealth,color=\srColor] (\Ax, \Ay)
	node [draw,fill=white,rounded corners,left] {#2}
	-- (\Bx, \By);
}


\newcommand{\trapFrame}[2][]{{
% #2 -> trapAddr
\pgfkeys{/trapFrame,default,#1}

\begin{stackFrame}[color=\tfColor,opacity=\tfOpacity]{#2}{\xinteval{#2 - 2 * \wordSize}}
\frameName{\tfName}
\frameLocal{0}{\tfHandler}

\ifNextArrow
\pointerTo{\xinteval{#2 - \wordSize}}{\xinteval{\tfNext - \wordSize}}
\else
\IfStrEq{\tfNext}{}
{}

% FIXME pfgkeys should expand hex with commands
{\ifthenelse{\isBigNumber{\tfNext} = 1}
{\frameLocal[hex=true]{1}{\tfNext}}
{\frameLocal[hex=false]{1}{\tfNext}}}
\fi
\end{stackFrame}
}}
